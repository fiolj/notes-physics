\chapter{Generalization to arbitrary potentials}

In order to evaluate the cross sections for multiple electron atoms, we
will use numeric wavefunctions in the initial and final results
\parencite{Madison1973PRAp2449,Gulyas1995JPBp245,Fiori2001PRAp12705}. We will use the program of
Salvat and co-workers \parencite{Salvat1995CPCp151} To solve numerically the
partial-wave time-independent Schr\"{o}dinger equation and evaluate the
transition matrix by means of partial wave expansions.

The only difference between the present calculations and those
performed for hydrogenic atoms arise in the evaluation of matrix
elements between atomic wavefunctions, that will have the general form
\[
f = \langle \, \phi_{f}^{-} \, \mid  \, g \, \mid \, \phi_{i} \rangle
\]

\section{Definition of initial and final states}

The initial bound state of the atom is given by an eigenvector of the
angular momentum $L,L_{z}$ operators
%
\begin{equation}\label{Q:ap1}
\phi_{i}(\bm{r}) = \frac{u_{n_{i} \ell_{i}} (\bm{r})}{r} \, Y_{\ell_{i}
m_{i}}(\hat{r})
\end{equation}
%
where the $u_{n_{i} \ell_{i}}$ verifies the differential equation
\[
h_{\ell} u_{n \ell} \equiv \left[ - \frac{d^{2}}{d r^{2}} +
\frac{\ell (\ell + 1 )}{r^{2}} + \frac{2 m_{j}}{\hbar^{2}}\, V_{j}(r)
\right] u_{n \ell} = \varepsilon_{n \ell} \, u_{n \ell}
\]

The continuum state can be expanded in partial waves (convention used
in \cite{Salvat1995CPCp151})

\begin{equation}\label{Q:ap2}
\psi^{\pm}_{\bm{k}}(\bm{r}) = \frac{4 \pi}{(2 \pi)^{3/2}}\sum_{\ell,\, m}
i^{\ell} \, e^{\pm i \delta_{\ell}} \frac{P_{k,\,
\ell}(r)}{k\,r}\ Y^{*}_{\ell m}(\hat{k})\, Y_{\ell m}(\hat{r})
\end{equation}

\subsection{Normalization of the partial-wave functions}

The bound states of total and $z$ component of angular momentum
$\ell_{i}, m_{i}$ are normalized to unity
\[
\int_{0}^{\infty} d r \; u^{*}_{n_{i} \ell_{i}} \, u_{n'_{i}
\ell'_{i}} = \delta_{n_{i} n'_{i}} \delta_{\ell_{i} \ell'_{i}} \, .
\]

The continuum states are normalized in the momentum scale
\[
\int_{0}^{\infty} d r \; P^{*}_{k \ell} \, P_{k' \ell} =
\frac{\pi}{2}\; \delta(k-k') \, ,
\]
with asymptotic behavior
\[
\lim_{r \to \infty} P_{k \ell}(r) = \sin\left( kr - \nu \ln(2 k r) +
\frac{\ell \pi}{2} + \delta_{\ell} \right)
\]

However, normalization parameters can be chosen such that both bound
and continuous partial-wave states are real.

\section{Evaluation of the Form factor}

Here we evaluate the form factor

\begin{equation}\label{Q:ap3}
F_{if} = (2 \pi)^{-3/2} \int d \bm{r} \, \psi^{-*}_{\bm{k}}(\bm{r})
\ e^{i \bm{q} \cdot \bm{r}} \phi_{n_{i} \ell_{i}}(\bm{r}) \,.
\end{equation}

We will use the expansion of the exponential \cite[(A.42)]{Galindo1990_QMvI}

\begin{equation}\label{Q:ap4}
e^{i \bm{q} \cdot \bm{r}} = 4 \pi \, \sum_{\ell, m} i^{\ell}
j_{\ell}(q r) \ Y^{*}_{\ell m}(\hat{q})\, Y_{\ell m}(\hat{r})
\end{equation}
where
\[
j_{\ell}(z) = \sqrt{\frac{\pi}{2z}}\, J_{\ell +1/2}(z)
\]
is the spherical Bessel function of order $\ell$
\cite[(A.116)]{Galindo1990_QMvI} and $J_{\ell}$ is the $\ell$-order Bessel
function. The convention on these functions is such that
$j_{0}(z)=\sin(z)/z$.

Replacing \ref{Q:ap1},\ref{Q:ap2} and \ref{Q:ap4} in \ref{Q:ap3} we
obtain
\begin{eqnarray*}
\lefteqn{F_{if} = \frac{1}{(2 \pi)^{3/2}} \int r^{2} d r d
\hat{r} \frac{u_{n_{i}\ell_{i}}(r)}{r}\, Y_{\ell_{i} m_{i}}(\hat{r})
\left[ \frac{4 \pi}{(2 \pi)^{3/2}} \sum_{\ell,m} (-i)^{\ell}
e^{i \delta_{\ell}} \frac{P_{k \ell}(r)}{k r} \,\ Y_{\ell
m}(\hat{k})\, Y^{*}_{\ell m}(\hat{r}) \right]}
\\
&& \qquad \qquad {\times} \left[ 4 \pi  \sum_{\ell',m'} \, i^{\ell'}
j_{\ell'}(q r) \, \ Y^{*}_{\ell' m'}(\hat{q})\, Y_{\ell' m'}(\hat{r})
\right]\,
\\
&=& \frac{2}{\pi k} \sum_{{\ell,m \atop \ell',m'}}  (-i)^{\ell} \,
i^{\ell'} e^{i \delta_{\ell}} \,Y_{\ell m}(\hat{k})\,
Y^{*}_{\ell' m'}(\hat{q}) f_{\ell,\ell',\ell_{i}}(k,q) \int d
\hat{r}\, Y^{*}_{\ell m}(\hat{r}) \, Y_{\ell' m'}(\hat{r}) \,
Y_{\ell_{i} m_{i}}(\hat{r}) \, .
\end{eqnarray*}

The integral over the solid angle can be written in terms of the
Clebsch-Gordan coefficients or 3-$j$ symbols \cite[(A.44)]{Galindo1990_QMvI}
%
\begin{eqnarray*}
\int d \hat{r}\, Y^{*}_{\ell m}(\hat{r}) \, Y_{\ell' m'}(\hat{r}) \,
Y_{\ell_{i} m_{i}}(\hat{r}) = \sqrt{\frac{(2 \ell' + 1)(2 \ell_{i} +
1)}{4 \pi (2 \ell + 1)}} \ C(\ell' \, \ell_{i} \, \ell; m' \, m_{i} \,
m)\, C(\ell' \, \ell_{i} \, \ell;0 \, 0 \, 0)
\\
= (-1)^{m} \sqrt{\frac{(2 \ell + 1)(2 \ell' + 1)(2 \ell_{i} + 1)}{4 \pi
}} \ 
% \left( 
\begin{matrix}
\ell' & \ell_{i} & \ell \cr m' & m_{i} & - m 
\end{matrix}
% \right) 
% \left( 
\begin{matrix}
\ell' & \ell_{i} & \ell \cr 0 & 0 & 0
\end{matrix}
 % \right)
\end{eqnarray*}
%
where the relation \cite[(3.7.3)]{Edmonds1960_AMI} (appendix \ref{S:spfunc})
\begin{equation}\label{Q:ap5}
C(\ell_{1} \, \ell_{2} \, \ell_{3}; m_{1} \, m_{2} \, m_{3}) =
(-1)^{\ell_{1}-\ell_{2} + m_{3}} \, \sqrt{2 \ell_{3} + 1} 
% \left(
\begin{matrix}
\ell_{1} & \ell_{2} & \ell_{3} \cr m_{1} & m_{2} & -m_{3}
\end{matrix}
% \right)
\end{equation}
has been used. We note that at less
\[
m'+ m_{i}= m  \qquad \mbox{and} \qquad \mid \ell'-\ell_{i}\mid < \ell <
\ell' + \ell_{i}
\]
the Clebsch-Gordan and 3-$j$ symbols are null.

Introducing these results in the above expression of the form factor
$F_{if}$ and using the relation \ref{Q:spf3} we obtain

\begin{eqnarray*}
\lefteqn{ F_{if} = \frac{2}{\pi \, k}  \sum_{\ell, \ell',m}
(-i)^{\ell} \,(-1)^{m} \, i^{\ell'} e^{i \delta_{\ell}} \,
\sqrt{\frac{(2 \ell + 1)(2 \ell' + 1)(2 \ell_{i} + 1)}{4 \pi }} \
\left( 
\begin{matrix}\ell & \ell' & \ell_{i} \cr -m & m' & m_{i}
\end{matrix}
\right)}
\\
 && {\times} \left( \begin{matrix}\ell & \ell' & \ell_{i} \cr 0 & 0 & 0
   \end{matrix}
 \right) f_{\ell,\ell',\ell_{i}}(k,q) \ Y_{\ell m}(\hat{k})\,
Y^{*}_{\ell' m'} (\hat{q})
 \\
\\
&=& \frac{2}{\pi \, k} \sum_{\ell'} \sum_{\ell=\mid \ell'-\ell_{i}\mid
}^{\ell' + \ell_{i}} \, i^{\ell'} \, (-i)^{\ell} e^{i
\delta_{\ell}} \, \sqrt{\frac{(2 \ell + 1)(2 \ell' + 1)(2 \ell_{i} +
1)}{4 \pi }} \ \left( \begin{matrix}\ell & \ell' & \ell_{i} \cr 0 & 0 & 0
\end{matrix}
\right) \, f_{\ell,\ell',\ell_{i}}(k,q)
\\
&& {\times} \,\sum_{m'} \, (-1)^{m} \left( 
\begin{matrix}\ell & \ell' & \ell_{i} \cr
-m & m' & m_{i}
\end{matrix}
\right) \ Y_{\ell m}(\hat{k})\, Y^{*}_{\ell' m'}
(\hat{q})
\end{eqnarray*}
%
with $m'+ m_{i}= m$. We finally write
%
\begin{eqnarray}\label{Q:ap6}
F_{if}(\bm{k},\bm{q})&=& \frac{2}{\pi \, k} \sum_{\ell=0}^{\infty}
i^{\ell'_{min}} \, (-i)^{\ell} \, e^{i \delta_{\ell}} \,
\sum_{m=-\ell}^{\ell} \, (-1)^{m} \, B_{\ell m} \ \left[ Y_{\ell
m}(\hat{k})\, e^{-i m' \varphi_{q}} \right]
\\
\nonumber
\\
B_{\ell m} &=& \sum_{\ell'=\ell'_{min}}^{\ell + \ell_{i}} \,
i^{\ell'_{0}} \sqrt{\frac{(2 \ell + 1)(2 \ell' + 1)(2 \ell_{i} +
1)}{4 \pi }}
\\
&& \left( \begin{matrix}\ell & \ell' & \ell_{i} \cr -m & m-m_{i} & m_{i}
  \end{matrix}
\right) \left( \begin{matrix}\ell & \ell' & \ell_{i} \cr 0 & 0 & 0
  \end{matrix}
\right)
f_{\ell \, \ell'\, \ell_{i}}(k,q) \ \mathcal{Y}_{\ell',m'}(\cos
\theta_{q}) \nonumber
\\
\nonumber \\
f_{\ell,\ell',\ell_{i}} &=& \int_{0}^{\infty} d r  P^{*}_{k \ell}(r)
\, j_{\ell'}(q r) \, u_{n_{i}\ell_{i}}(r)
\\
\mathcal{Y}_{\ell,m}(\cos \theta_{q}) &=&  e^{i m \varphi_{q}}
Y^{*}_{\ell m} (\theta_{q},\varphi_{q}) \label{Q:ap9}
\end{eqnarray}
with $\ell'_{min} = \mid \ell - \ell_{i}\mid $ and $m' + m_{i} = m$.
We note that the terms in the sum over $\ell'$ are not null only when
$\ell + \ell' + \ell_{i}$ is a even number (see \ref{Q:spf4}). While
\[
\ell'_{min} + \ell + \ell_{i}= \mid \ell - \ell_{i}\mid  + \ell +
\ell_{i} = 2 L \, , \qquad L = \max(\ell,\ell_{i})
\]
the sum is carried out from $\ell'_{min}$ of two in two.

\subsection*{Alternative order in the summation}
%
\begin{eqnarray}\label{Q:ap6b}
F_{if}(\bm{k},\bm{q})&=& \frac{2}{\pi \, k} \sum_{\ell=0}^{\infty}
i^{\ell'_{min}} \, (-i)^{\ell} \, e^{i \delta_{\ell}} \,
\sum_{\ell'=\ell'_{min}}^{\ell + \ell_{i}} \, i^{\ell'_{0}} \,
\sqrt{\frac{(2 \ell + 1)(2 \ell' + 1)(2 \ell_{i} + 1)}{4 \pi }}
\nonumber
\\
&& {\times} \left( \begin{matrix}\ell & \ell' & \ell_{i} \cr 0 & 0 & 0
  \end{matrix}
\right) \
f_{\ell \ell' \ell_{i}}(k,q) \ B'_{\ell \ell'}(\hat{k}, \hat{q})
\\
\nonumber
\\
B'_{\ell \ell'}(\hat{k},\hat{q}) &=& \sum_{m=-\ell}^{\ell} \, (-1)^{m}
\,\left( \begin{matrix}\ell & \ell' & \ell_{i} \cr -m & m' & m_{i}
  \end{matrix}
\right)
\, Y_{\ell m}(\hat{k})\, Y^{*}_{\ell'\,m'}(\hat{q})
\\
\nonumber \\
f_{\ell \, \ell' \, \ell_{i}}(k,q) &=& \int_{0}^{\infty} d r
P^{*}_{k \ell}(r) \, j_{\ell'}(q r) \, u_{n_{i}\ell_{i}}(r)
\end{eqnarray}

\subsection*{Expansion in $m$-values}
%
\begin{eqnarray}\label{Q:ap6c}
F_{if}(\bm{k},\bm{q})&=& \frac{2}{\pi \, k} \sum_{m=-\infty}^{\infty}
(-1)^{m} \sum_{\ell=m}^{\infty} i^{\ell'_{min}} \, (-i)^{\ell} \,
e^{i \delta_{\ell}} \, \sum_{\ell'=\ell'_{min}}^{\ell + \ell_{i}}
\, i^{\ell'_{0}} \, \left( \begin{matrix}\ell & \ell' & \ell_{i} \cr 0 & 0
& 0
\end{matrix}
\right)\, f_{\ell \ell' \ell_{i}}(k,q)  \nonumber
\\
&& {\times}  \sqrt{\frac{(2 \ell + 1)(2 \ell' + 1)(2 \ell_{i} + 1)}{4 \pi }} \
\left( \begin{matrix}\ell & \ell' & \ell_{i} \cr -m & m' & m_{i}
  \end{matrix}
\right) \,
Y_{\ell m}(\hat{k})\, Y^{*}_{\ell'\,m'}(\hat{q}) \nonumber \\
&=& \sum_{m=-\infty}^{\infty} F_{if}^{m}
\end{eqnarray}



\subsection{Initial $s$ state}

\begin{eqnarray}\label{Q:ap10}
F_{if}&=& \frac{2}{\pi \, k} \sum_{\ell=0}^{\infty} \, e^{i
\delta_{\ell}} \,\frac{(2 \ell + 1)}{(4 \pi)^{3/2}}\
f_{\ell,\ell,0}(k,q)\; P_{\ell}(\cos{\theta_{kq}})
\end{eqnarray}

\subsection{Integration over the angular variables}

\textbf{Observe that only one integration is required in the evaluation
of cross sections}

In the partial wave expansion of the first Born approximation the
double differential cross section in the modulus of the momentum
transfer and the ejected electron is straightforward. The double
differential cross section is proportional to the integral of the
square modulus of the form factor over the angular variables. We start
with expression \ref{Q:ap6b}

\begin{eqnarray}\label{Q:ap6d}
\int d \hat{q} \, d \hat{k} \mid F_{if}(\bm{k},\bm{q})\mid ^{2}
&=& \frac{4}{\pi^{2} \, k^{2}}  \sum_{\ell_{1},\ell_{2}=0}^{\infty}
a_{\ell_{1}} a^{*}_{\ell_{2}}
\sum_{\ell_{1}',\ell_{2}'=\ell_{min}'}^{\ell_{max}'}
b_{\ell_{1},\ell'_{1}} b^{*}_{\ell_{2},\ell'_{2}} \nonumber
\\
&& {\times} \int d \hat{q} \;d \hat{k}  \ B'_{\ell_{1} \ell_{1}'}
(\hat{k}, \hat{q})  \ B^{*'}_{\ell_{2} \ell_{2}'}(\hat{k}, \hat{q}) \,
.
\end{eqnarray}

The integral  is easily performed
\begin{eqnarray*}
\lefteqn{\int d \hat{q} \;d \hat{k}  \ B'_{\ell_{1} \ell_{1}'}
(\hat{k}, \hat{q})  \ B^{*'}_{\ell_{2} \ell_{2}'}(\hat{k}, \hat{q})=}
\\
&&\sum_{m_{1},m_{2}} (-1)^{m_{1}+m_{2}} \,\left( \begin{matrix}\ell_{1} &
\ell_{1}' & \ell_{i} \cr -m_{1} & m'_{1} & m_{i}
\end{matrix}
\right) \, \left(
\begin{matrix}\ell_{2} &
\ell_{2}' & \ell_{i} \cr -m_{2} & m'_{2} & m_{i}
\end{matrix}
\right) \qquad \qquad
\\
&& {\times} \int d \hat{k} \,Y_{\ell_{1} m_{1}}(\hat{k})\, Y^{*}_{\ell_{2}
m_{2}}(\hat{k})\, \; \int d \hat{q} \,
Y^{*}_{\ell_{1}'\,m_{1}'}(\hat{q})
\, Y_{\ell_{2}'\,m_{2}'}(\hat{q}) \\
&=& \sum_{m=-\ell}^{\ell} \, \left( \begin{matrix}\ell & \ell' & \ell_{i} \cr
-m & m' & m_{i}
\end{matrix}
\right) \, \left( \begin{matrix}\ell & \ell' & \ell_{i} \cr
-m & m' & m_{i}
\end{matrix}
\right)
\end{eqnarray*}
The orthonormality of the harmonic spherical \ref{S:Spher-harmon} was
used in the last expression. By using the orthogonality of 3-$j$
symbols the summation can be simplified to
\[
\int d \hat{q} \;d \hat{k}  \ B'_{\ell_{1} \ell_{1}'} (\hat{k},
\hat{q})  \ B^{*'}_{\ell_{2} \ell_{2}'}(\hat{k}, \hat{q}) = 2 \,
\ell_{i} + 1
\]

The expression \ref{Q:ap6c} looks now
\begin{eqnarray*}
\int d \hat{q} \, d \hat{k} \mid F_{if}(\bm{k},\bm{q})\mid ^{2}
&=& \frac{4 ( 2 \ell_{i} + 1)}{\pi^{2} \, k^{2}}
\sum_{\ell=0}^{\infty} \mid a_{\ell}\mid ^{2}
\sum_{\ell'=\ell_{min}'}^{\ell_{max}'} \mid b_{\ell,\ell'}\mid ^{2}
\nonumber
\\
&& {\times} \int d \hat{q} \;d \hat{k}  \ B'_{\ell_{1} \ell_{1}'}
(\hat{k}, \hat{q})  \ B^{*'}_{\ell_{2} \ell_{2}'}(\hat{k}, \hat{q}) \,
.
\end{eqnarray*}
where
\begin{eqnarray}
% \nonumber to remove numbering (before each equation)
\mid a_{\ell}\mid ^{2} &=& 1 \\
\mid b_{\ell,\ell'}\mid ^{2} &=& \frac{(2 \ell + 1)(2 \ell' + 1)(2
\ell_{i} + 1)}{4 \pi} \left( \begin{matrix}\ell & \ell' & \ell_{i} \cr 0 & 0
& 0
\end{matrix}
\right)^{2} \ f_{\ell \ell' \ell_{i}}^{2}(k,q)
\end{eqnarray}
\section{Separation of the 0-order}

Now we will separate the plane wave approximation in order to speed up
the convergence of the partial wave series. In the plane wave
approximation (which in principle is valid only for short range
potentials) the form factor is given simply by the Fourier Transform of
the initial state
\begin{equation}\label{Q:FTwfi}
F^{0}_{if}(\bm{q}) = (2 \pi)^{-3/2} \int d \bm{r} \,
\frac{e^{-i \bm{k} \cdot \bm{r}}}{(2 \pi)^{3/2}} \ e^{i
\bm{q} \cdot \bm{r}} \phi_{n_{i} \ell_{i}}(\bm{r}) =
\frac{\tilde{\phi}_{n_{i} \ell_{i}}(\bm{q}-\bm{k})}{(2 \pi)^{3/2}}\, .
\end{equation}
%
Thus, the full form factor is given by $F_{if} = F^{0}_{if} + \Delta
F_{if}$, where
%
\[
\Delta F_{if} (\bm{q}) = (2 \pi)^{-3/2} \int d \bm{r} \,\left[
\psi_{\bm{k}}^{-*}(\bm{r}) - \frac{e^{-i \bm{k} \cdot \bm{r}}}{(2
\pi)^{3/2}} \right] \ e^{i \bm{q} \cdot \bm{r}} \phi_{n_{i}
\ell_{i}}(\bm{r}) \, .
\]

By using \ref{Q:ap2} and \ref{Q:ap4}  the calculation is
straightforward. The result is similar to $F_{if}$, but with $f_{\ell
\, \ell'\, \ell_{i}}(k,q)$ replaced by
\[
\Delta f_{\ell \, \ell'\, \ell_{i}}(k,q) = \int_{0}^{\infty} d r \,
\left[ P_{k\, \ell}(r) - kr \, e^{- i \delta_{\ell}(k)}
j_{\ell}(kr) \right] \, j_{\ell'}(q r) \, u_{n_{i} \ell_{i}}(r)
\]

\subsection{Evaluation of the Fourier Transform of the initial state}

Using the series expansion \ref{Q:ap4} and the explicit form
\ref{Q:ap1} we obtain

\begin{eqnarray*}
\tilde{\phi}_{n_{i} \ell_{i}}(\bm{q}) &=& \frac{4 \pi}{(2 \pi)^{3/2}}
\sum_{\ell,m} i^{\ell} Y_{\ell m}(\hat{q}) \int_{0}^{\infty} d r
\; r\, u_{n_{i} \ell_{i}}(r)\, j_{\ell}(q r) \; \int d \hat{r}
Y^{*}_{\ell
m}(\hat{r}) \, Y_{\ell_{i} m_{i}}(\hat{r}) \\
&=& \frac{4 \pi}{(2 \pi)^{3/2}} \; i^{\ell_{i}} \, Y_{\ell_{i}
m_{i}}(\hat{q}) \int_{0}^{\infty} d r \; r\, u_{n_{i} \ell_{i}}(r)\,
j_{\ell_{i}}(q r)
\end{eqnarray*}

\section{Evaluation of C3 approximation}

\begin{eqnarray} \label{Q:ap11}
t_{{\rm C3}} &=& \frac{Z_{P}\, \mathcal{N}_{P}
\mathcal{N}_{N}}{(2\pi)^{9/2}} \; \int d \bm{q} \Big[
\mathcal{Z}_{T} J_{0} \left( \lambda_{P}, \bm{p}- \textbf{q} -
\bm{Q},-\nu_{P},\bm{k}_{P} \right) J_{1}( \lambda_{N}, \bm{p}-\bm{q},
-\nu_{N}, \bm{k}_{N} )
  \nonumber \\
&& - \; J_{1} \left( \lambda_{P}, \bm{p}- \textbf{q} -
\bm{Q},-\nu_{P},\bm{k}_{P} \right) \; J_{0}( \lambda_{N},
\bm{p}-\bm{q}, -\nu_{N}, \bm{k}_{N} ) \, \Big] \; F_{if}(\bm{q} )
\nonumber
\\
&=& \frac{Z_{P}\, \mathcal{N}_{P} \mathcal{N}_{N}}{(2\pi)^{9/2}} \;
\int d \bm{q} F_{if}(\bm{q} ) \; \mathcal{G}(\bm{k},\bm{q},\bm{Q})
\end{eqnarray}
where $\bm{p}=(m_{T}/m)\bm{Q}$.

By using the expansions (\ref{Q:ap6}-\ref{Q:ap9}) we can write

\begin{eqnarray}\label{Q:ap13}
t_{{\rm C3}} &=& \frac{Z_{P} \, \mathcal{N}_{P}
\mathcal{N}_{N}}{(2\pi)^{9/2}} \frac{2}{\pi k_{T}}
\sum_{\ell=0}^{\infty} i^{\ell'_{min}} \, (-i)^{\ell} \,
e^{i \delta_{\ell}} \, \sum_{m=-\ell}^{\ell} \, (-1)^{m} \,
\mathcal{B}^{\mathrm{C3}}_{\ell m} \ Y_{\ell m}(\hat{k}_{T})
\\
\nonumber
\\
\mathcal{B}^{\mathrm{C3}}_{\ell m} &=& \sum_{\ell'=\ell'_{min}}^{\ell +
\ell_{i}} \, i^{\ell'_{0}} \sqrt{\frac{(2 \ell + 1)(2 \ell' + 1)(2
\ell_{i} + 1)}{4 \pi }}\left( \begin{matrix}\ell & \ell' & \ell_{i} \cr -m &
m-m_{i} & m_{i}
\end{matrix}
\right) \left( \begin{matrix}\ell & \ell' & \ell_{i} \cr 0 &
0 & 0
\end{matrix}
\right) \nonumber
\\
&{\times}& \int_{0}^{\infty} q^{2} d q \, f_{\ell,\ell',\ell_{i}}(k_{T},q)
\ \int_{-1}^{1}d (\cos \theta_{q}) \, \mathcal{Y}_{\ell',m'}(\cos
\theta_{q}) \int_{0}^{2 \pi} d \varphi_{q} e^{-i m'
\varphi_{q}} \mathcal{G}(\bm{k},\bm{q},\bm{Q}) \;.  \nonumber
\end{eqnarray}

For a $s$ initial state it is given by

\begin{eqnarray}\label{Q:ap12}
t_{{\rm C3}} &=& \frac{Z_{P} \, \mathcal{N}_{P}
\mathcal{N}_{N}}{(2\pi)^{9/2}} \frac{2}{\pi k_{T}}
\sum_{\ell=0}^{\infty} \, e^{i \delta_{\ell}} \,\frac{(2 \ell +
1)}{(4 \pi)^{3/2}} \,
\\
&{\times}&
 \int_{0}^{\infty} q^{2} d q \, f_{\ell,\ell,0}(k_{T},q) \
\int_{-1}^{1}d (\cos \theta_{q}) \,  \int_{0}^{2 \pi} \,d
\varphi_{q}\ \mathcal{G}(\bm{k},\bm{q},\bm{Q}) P_{\ell}(\cos
\theta_{k_{T}q}) \qquad \nonumber
\end{eqnarray}


\section{Numerical Fourier transform of the potential}

The Fourier transform of an arbitrary potential is given by

\[
\tilde{V}(\bm{q}) = \frac{1}{(2 \pi \hbar)^{3/2}} \int e^{i
\bm{q} \cdot \bm{r}} V(r) d \bm{r}
\]
is simplified in cases of spherical symmetry to
\begin{eqnarray*}
(2 \pi \hbar)^{3/2} \tilde{V}(\bm{Q}) &=& 2 \pi \int r^{2} V(r) d r
\int_{-1}^{1} e^{i q r x} d x \\
&=& \frac{4 \pi}{q} \int \left[ r V(r) \right] \, \sin{(q r)}\, d r
\\
&=& \frac{4 \pi}{q} \int \mathcal{V}(r) \, \sin{(q r)}\, d r
\end{eqnarray*}

\subsection{Linear local approximation to the potential}

If we have a table $\mathcal{V}_{i}= \mathcal{V}(r_{i}) $ ($i=1,
\ldots, N_{P}$) and approximate its value at intermediate points
between $r_{i}$ and $r_{i+1}$ by a linear interpolation
$\mathcal{V}(r)= a_{i} + b_{i}\,r$ with local coefficients for each
$i$, the integral is easily performed. We separate the domain

\[
(2 \pi \hbar)^{3/2} \tilde{V}(\bm{Q}) = \frac{4 \pi}{q} \int_{0}^{R}
\mathcal{V}(r) \, \sin{(q r)}\, d r + \int_{R}^{\infty}
\mathcal{V}(r) \, \sin{(q r)}\, d r
\]
where $R$ is a point large enough such that the potential has converged
to its asymptotic value $\lim_{R \to \infty} \mathcal{V}(R) =
\mathcal{V}_{\infty}$. The asymptotic part of the integral is carried
out analytically (after using the Abel's regularization)
\begin{equation}\label{Q:apFTV1}
\lim_{\alpha \to 0^{+}} \int_{R}^{\infty} \mathcal{V}(r) \,e^{-
\alpha r} \sin{(q r)}\, d r = \mathcal{V}(R) \frac{\cos{(q R)}}{q}
\, .
\end{equation}

The inner part of the integral can be simplified with the above
approximation

\begin{equation}
I = \int_{0}^{R} \mathcal{V}(r) \, \sin{(q r)}\, d r =
\sum_{i=1}^{N-1} \int_{r_{i}}^{r_{i+1}} \mathcal{V}(r) \, \sin{(q r)}\,
d r = \sum_{i=1}^{N-1} I_{i}
\end{equation}
where, using the chain rule (inverted) we obtain
\begin{eqnarray*}
I_{i} &=& \int_{r_{i}}^{r_{i+1}} \mathcal{V}_{i}(r) \, \sin{(q r)}
\,d r
\\
&=& \int_{r_{i}}^{r_{i+1}}  \left\{ \left[\mathcal{V}_{i}(r)
\,\frac{-\cos{(q r)}}{q} \right]' - \mathcal{V}'_{i}(r) \frac{-\cos{(q
r)}}{q} \right\} \,d r
\\
&=& \left. \mathcal{V}(r) \,\frac{-\cos{(q r)}}{q}  + b_{i}
\frac{\sin{(q r)}}{q^{2}} \; \right| _{r_{i}}^{r_{i+1}}
\end{eqnarray*}



Summing up the result for all the intervals we get
\begin{eqnarray}
I &=& \sum_{i=1}^{N-1}\left[  \mathcal{V}(r_{i}) \,\frac{\cos{(q
r_{i})}}{q} - b_{i} \frac{\sin{(q\, r_{i})}}{q^{2}} \right] -
\sum_{i=1}^{N-1}\left[ \mathcal{V}(r_{i+1}) \,\frac{\cos{(q
r_{i+1})}}{q}  - b_{i} \frac{\sin{(q\, r_{i+1})}}{q^{2}} \right]
 \\
&=&  \frac{\mathcal{V}(0)}{q} - \frac{\mathcal{V}(r_{N})}{q} \cos{(q \,
r_{N})} + \sum_{i=1}^{N-1}  \frac{b_{i}}{q^{2}} \Big[ \sin{(q\,
r_{i+1})}-\sin{(q \, r_{i})} \Big]
\end{eqnarray}

The final expression is
\begin{equation}\label{Q:apFTV2}
\tilde{V}(\bm{Q}) =\frac{4 \pi}{(2 \pi \hbar)^{3/2} \, q^2} \left\{
\mathcal{V}(0) + \sum_{i=1}^{N-1}  \frac{b_{i}}{q} \Big[ \sin{(q \,
r_{i+1})}-\sin{(q \, r_{i})} \Big] \right\}
\end{equation}

\subsection*{Case of a pure Coulomb potential}

In the case of a Coulomb potential  $\mathcal{V}(r) = Z$ constant.
Thus, the coefficient $b_{i}= \mathcal{V}'(r_{i})$ is null for all
$i=1, \ldots, N$. The Fourier transform is simply given by

\[
\tilde{V}(\bm{Q}) =\frac{1}{(2 \pi \hbar)^{3/2} }\; \frac{4 \pi \, Z}{
q^2} .
\]

\subsection{Spline approximation of the potential}

If the potential can be locally approximated by a fourth grade
polynomial

\begin{equation}\label{Q:apTFspl}
\mathcal{V}_{i}(r) = A_{i} + B_{i}\,r +C_{i}\,r^{2}+D_{i}\,r^{3} \, .
\end{equation}%
Thus,
%
\begin{eqnarray*}
I_{i} &=& \frac{1}{q^{4}}\Big[ q\,\left( -6\,D_{i}\,r_{i} +
C_{i}\,\left( -2 + {q^2}\,{r_{i}^2} \right)  + {q^2}\,\left( A_{i} +
B_{i}\,r_{i} + D_{i}\,{r_{i}^3}
\right)  \right) \,\cos (q\,r_{i}) \nonumber \\
&&- q\,\left( -2\,C_{i} - 6\,D_{i}\,r_{i+1} + C_{i}\,{q^2}\,{r_{i+1}^2}
+ D_{i}\,{q^2}\,{r_{i+1}^3} + {q^2}\,\left( A_{i} + B_{i}\,r_{i+1}
\right) \right)
\, \cos (q\,r_{i+1})  \nonumber \\
&&- \left[     \left( {q^2}\,\left( B_{i} + 2\,C_{i}\,r_{i} \right) +
3\, D_{i}\,\left( -2 + {q^2}\,{r_{i}^2} \right) \right) \,\sin
(q\,r_{i}) \right.
\nonumber  \\
&& + \left( {q^2}\,\left( B_{i} + 2\,C_{i}\,r_{i+1} \right)  +
3\,D_{i}\,\left( -2 + {q^2}\,{r_{i+1}^2} \right) \right) \,\sin
(q\,r_{i+1}) \Big]
\end{eqnarray*}

Which can be further simplified by using \ref{Q:apTFspl} to
%
\begin{eqnarray*}
I_{i} &=& \frac{1}{q}\left\{ \left( \mathcal{V}_{i}(r_{i}) -
\frac{\mathcal{V}_{i}''(r_{i})}{q^{2}}\, \right) \right. \,\cos
(q\,r_{i}) \nonumber
\\
&& \;\, - \left( \mathcal{V}_{i}(r_{i+1}) -
\frac{\mathcal{V}_{i}''(r_{i+1})} {q^{2}}\, \right) \, \cos
(q\,r_{i+1})
\nonumber \\
&&- \left[ \left( \mathcal{V}'_{i}(r_{i}) - \frac{6\, D_{i} }{q^{2}}
\right) \,\frac{\sin (q\,r_{i})}{q} \right.
\nonumber \\
&&- \left. \left. \left( \mathcal{V}'_{i}(r_{i+1}) - \frac{6\, D_{i}
}{q^{2}} \right) \,\frac{\sin (q\,r_{i+1})}{q} \right] \right\}
\end{eqnarray*}

The complete integral is obtained by summing up all terms $i=1, \ldots,
N-1$. We use that the function and its first and second derivatives are
continuous
\begin{eqnarray*}
\mathcal{V}_{i}(r_{i+1}) &=& \mathcal{V}_{i+1}(r_{i+1})
\\
\mathcal{V}'_{i}(r_{i+1}) &=& \mathcal{V}'_{i+1}(r_{i+1})
\\
\mathcal{V}''_{i}(r_{i+1}) &=& \mathcal{V}''_{i+1}(r_{i+1})
\end{eqnarray*}
%
to obtain
\begin{eqnarray*}
q. I &=& \mathcal{V}(r_{1})  - \mathcal{V}(r_{N})\, \cos (q\,r_{N}) +
\frac{\mathcal{V}'(r_{N}) \sin{(q\,r_{N})}}{q}
-\frac{\mathcal{V}''(r_{N}) \cos{(q\,r_{N})} -
\mathcal{V}''(r_{1})}{q^{2}} \nonumber
\\
&&-  \frac{6}{q^{3}}\sum_{i=1}^{N-1} D_{i} \Big[\sin{ (q r_{i+1})} -
\sin{ (q r_{i})}\Big]
\end{eqnarray*}

The Form factor is then given by
\begin{equation}\label{Q:apspl2}
\tilde{V}(\bm{Q}) =\frac{4 \pi}{(2 \pi \hbar)^{3/2} \, q^2} \left\{
\mathcal{V}(0) + \frac{\mathcal{V}''(0)}{q^{2}} - \frac{6}{q^{3}}
\sum_{i=1}^{N-1} D_{i} \Big[\sin{ (q r_{i+1})} - \sin{ (q r_{i})}\Big]
\right\}
\end{equation}


\section{Evaluation of simple Nordsieck-like integrals}
In this section we are interested in the evaluation of integrals of the kind:
\begin{equation}\label{Q:J-Nords-Numer} 
  J_{n}(\bm{p},\bm{k},\alpha) =
  \int d \bm{r} \, e^{i \bm{p}\cdot\bm{r}} \, \frac{e^{-\alpha \, r}}{r^{n}} \, \psi_{\bm{k}}^{+*}(\bm{r})
\end{equation}

In order to evaluate the integral we insert the partial-wave expansions for the plane-wave and the scattering state (\ref{Q:ap2}) and (\ref{Q:ap4})
%
\begin{eqnarray*}
e^{i \bm{p} \cdot \bm{r}} &=& 4 \pi \, \sum_{\ell', m'} i^{\ell'}
j_{\ell'}(p r) \ Y_{\ell' m'}(\hat{p})\, Y^{*}_{\ell' m'}(\hat{r}) 
\\  
\psi^{\pm}_{\bm{k}}(\bm{r}) &=& \frac{4 \pi}{(2 \pi)^{3/2}}\sum_{\ell,\, m}
i^{\ell} \, e^{\pm i \delta_{\ell}} \frac{\mathcal{P}_{k,\,
\ell}(r)}{k\,r}\ Y^{*}_{\ell m}(\hat{k})\, Y_{\ell m}(\hat{r})\, ,
\end{eqnarray*}
%
obtaining
%
\begin{eqnarray*}
  J_{n}(\bm{p},\bm{k},\alpha) &=& \sum_{\ell, m, \ell', m'} \frac{(4 \pi)^{2}}{(2\pi)^{3/2}} \times \left(Y^{*}_{\ell m}(\hat{k}) Y_{\ell' m'}(\hat{p}) \right) \nonumber
\\
&&\times \, \left(\int d r \; j_{\ell'}(pr) \frac{r^{2} e^{-\alpha r}}{r^{n}} \frac{\mathcal{P}_{k,\ell}(r)}{kr} \right) \;  \left( \int d \hat{r} Y^{*}_{\ell' m'}(\hat{r})  Y_{\ell m}(\hat{r})\right) 
\\
&=& \frac{1}{k} \sqrt{\frac{2}{\pi}} \, \sum_{\ell=0}^{\infty} (2 \ell +1) (-1)^{\ell} e^{-i \delta_{\ell}} \; \left(\int  d r \; j_{\ell'}(pr) \frac{r e^{-\alpha r}}{r^{n}} \, \mathcal{P}_{k,\ell}(r) \right) \, P_{\ell}(\cos{\theta})
\end{eqnarray*}

In order to accelerate the convergence of the series we calculate it as the difference between the general problem and the corresponding Coulomb problem $J_{n}= J^{c}_{n} + (J_{n} - J^{c}_{n}) = J^{c}_{n} + \Delta{J_{n}}$. The difference can be expanded in partial-waves to read
\begin{equation}\label{eq:4}
  \Delta J_{n}(\bm{p},\bm{k},\alpha) = \frac{1}{k} \sqrt{\frac{2}{\pi}} \, \sum_{\ell=0}^{\infty} (2 \ell +1) (-1)^{\ell} e^{-i \delta^{c}_{\ell}} \; g(p,k)\, P_{\ell}(\cos{\theta})
\end{equation}
where the function $g(p,k)$ is given by
\[
g(p,k)= \int d r \; j_{\ell'}(pr) \frac{r e^{-\alpha r}}{r^{n}}\left( \mathcal{P}_{k ,\ell}(r)e^{-i \delta^{s}_{\ell}}  - \mathcal{P}^{c}_{k ,\ell}(r)\right)  \, .
\]
Here we have separated the phase shifts corresponding to the Coulomb and short-range potentials as $\delta^{c}_{\ell}$ and $\delta^{s}_{\ell}$, respectively.





\section{Evaluation of $\bi{K}$ factors}

\begin{equation}\label{Q:K_if-factor}
\bm{K}_{if}(\bm{p}) = \frac{1}{(2 \pi)^{3}} \int d \bm{r}_{T} \;
e^{\, i \bm{p} \cdot \bm{r}_{T}} \, \phi_{i}(\bm{r}_{T}) \,
e^{- i \bm{k}_{T} \cdot \bm{r}_{T}} \,\nabla_{\bm{r}_{T}} \left[
D^{-*}(-\nu_{T},\bm{k}_{T},\bm{r}_{T}) \right]
\end{equation}
which can be rewritten as
\begin{equation}\label{Q:Kfact1}
\bm{K}_{if}(\bm{p}) = \frac{1}{(2 \pi)^{3}} \int d \bm{r}_{T} \;
e^{\, i \bm{p} \cdot \bm{r}_{T}} \, \phi_{i}(\bm{r}_{T}) \,
\left[ -i \bm{k}_{T} \psi_{\bm{k_{T}}}^{-*} + \nabla_{\bm{r}_{T}}
\psi^{-*}_{\bm{k_{T}}} \right]
\end{equation}
The first term in the square brackets is proportional to the form
factor $F_{if}$ (\ref{Q:ap6b}).

By introducing the series expansions \ref{Q:ap2} and \ref{Q:ap4} we can
write for the term involving the gradient
\begin{eqnarray*}
\bm{K}_{if}(\bm{p}) &=& \frac{1}{(2 \pi)^{3}} \int r^{2} \, d r
\end{eqnarray*}


We use \cite[B.105]{Galindo1990_QMvI} for the term involving a gradient can be
written in terms of irreducible vector tensors (see \ref{S:spfunc})

\begin{eqnarray*}
\nabla \left[ R(r)\, Y_{LM}(\hat{r}) \right] &=& - \sqrt{\frac{L+1}{2L
+ 1}} \left[ \frac{d R (r)}{d r} - \frac{L}{r} \, R(r) \right]
\boldsymbol{\mathcal{Y}}^{L+1}_{LM}(\hat{r}) \nonumber \\
&&+ \sqrt{\frac{L}{2L + 1}} \left[ \frac{d R (r)}{d r} +
\frac{L+1}{r} \, R(r) \right]
\boldsymbol{\mathcal{Y}}^{L-1}_{LM}(\hat{r}) \, .
\end{eqnarray*}

%%% Local Variables: 
%%% mode: latex
%%% TeX-master: "mainxs"
%%% End: 
