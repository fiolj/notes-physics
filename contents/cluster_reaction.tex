\chapter{Reaction rates for cluster formation}
\label{C:react-rates-cluster}

\section{Excited (unstables) dimers}
\label{S:excit-unst-dimers}

\subsection{Notation, involved magnitudes}
\label{S:cant-de-inter}

The macroscopic magnitudes governing the reactions in the gas will be:
\begin{itemize}
\item The concentration $x$ of \ce{XF6} in the gas.
\item The total pressure $P$.
\item Temperature $T$. May be important to distinguish between translational temperature $T_{tr}$ and roto-vibrational temperature $T_{rv}$.
\end{itemize}

Microscopically, important magnitudes will be:
\begin{itemize}
\item Initial collision velocity $v_{i}$.
\item Initial state of the molecule: angular momentum ($L_{i}$) and vibrational level $\nu_{i}$.
\item Final state of the dimer, characterized by its excess energy $\varepsilon^{*}$.
\item Time duration $\tau^{*}$ of survival of the aggregate.
\end{itemize}

\subsection{Statement of the problem}
\label{S:statement-problem}

We will be interested in the formation rates for excited dimers, $k_{d}^{*} \equiv k_{d}^{*}(v_{i}, L_{i}, \nu_{i} ; \varepsilon^{*}, \tau^{*})$, where we made explicit the dependence with the microscopic parameters. The number of dimers formed by unit of time is
\begin{equation}
  \label{Q:def-dimer-rate}
  k_{d}^{*}(v_{i}, L_{i}, \nu_{i} ; \varepsilon^{*}, \tau^{*}) = \sigma_{d}^{*}(v_{i}, L_{i}, \nu_{i} ; \varepsilon^{*}, \tau^{*}) \; J_{i}(v_{i}, L_{i}, \nu_{i} )
\end{equation}
where the cross section is obtained from Classical Trajectory Monte-Carlo (CTMC) simulations. The initial flux depends on the experimental conditions
\begin{equation}
  \label{Q:rates-flux}
  J_{i} =\frac{N_{i}}{A \,\Delta t} = \frac{N_{i}}{A \, v_{i} \Delta t} v_{i} = \rho(v_{i}, L_{i}, \nu_{i})\, v_{i}
\end{equation}
%
where $\rho_{i}$ is the collision distribution in the experimental conditions of the gas mixture,
\begin{equation}
  \label{Q:rates-def-collis-distrib}
  \rho(v_{i}, L_{i}, \nu_{i}) = N_{A} \, N_{M} f_{T_{rv}}(L_{i},\nu_{i}) \, f_{T_{tr}}(v_{i})
\end{equation}

\subsubsection{Cross section}
\label{S:cross-section}

The cross section for dimerization may be calculated from individual collisions between one atom and molecule. 
From the definition of cross section we have
\begin{equation*}
  \sigma_{d}^{*} = \frac{N_{d}^{*}}{J_{i}}= \frac{\text{Number of dimers per unit of time}}{\text{Incident flux}}
\end{equation*}

An alternative expression may be obtained in terms of the probability of dimerization in a collision at a given impact parameter $\vect{b}$:
\begin{equation}
  \label{Q:rates-def-probab}
  \sigma_{d}^{*} = p(\vect{b}) \rmd \vect{b} 
\end{equation}

\subsubsection{Molecular initial state distribution}

We will assume, at this time, that the probability of finding a given molecular initial state is given by a classical canonical distribution with ``roto-vibrational'' temperature $T_{rv}$,
\begin{equation}
  \label{Q:rates-rotovib-distrib}
  f_{T_{tr}}(v) = \left( \frac{\beta M}{\pi} \right)^{3/2} \, \rme^{- \beta M v^{2}}
\end{equation}
\subsubsection{Collision frequency}


%%% Local Variables:
%%% mode: latex
%%% TeX-master: "main"
%%% End:
