\chapter{Interactions with electromagnetic fields}
\label{C:Inter-with-elect-field}

We discuss the interaction of electromagnetic fields with matter. The
presentation follows closely that of references
\autocite{Bethe1997AC_IQM,Townsen2000_AMA}. Some parts are still missing and many need further elaboration.

\section{The Semiclassical approach}
\label{S:Semic-appro}
A widely used semiclassical treatment of the interaction of radiation with matter consists in a quantum-mechanical description of the particles and a classical description of the field \autocite{Bethe1997AC_IQM}.

\subsection{Classical Electromagnetic fields}

In a classical treatment, the electromagnetic field is obtained from the Maxwell equations
\begin{eqnarray}\label{Q:Maxwe-eq}
\nabla \cdot \bm{B} = 0 &,& \nabla {\times} \bm{E} +
  \frac{1}{c}\frac{\partial \bm{B}}{\partial t} = 0 \nonumber \\ \\
  \nabla  \cdot \bm{E} = 4 \pi \rho_{e} &,&
\nabla {\times} \bm{B} - \frac{1}{c} \frac{\partial \bm{E}}{\partial t} =
\frac{4 \pi}{c} \bm{J}_{\mathrm{e.m.}} \nonumber
\end{eqnarray}
Here, the charge density $\rho_{e}$ and electric current $J_{\mathrm{e.~m.}}$ are related by the continuity relation
\[
\frac{\partial \rho_{e}}{\partial t} + \nabla \cdot
\bm{J}_{\mathrm{e.m.}}= 0
\]

The first equation implies that the magnetic field can be expressed as the divergence of a certain vector potential $\bm{A}$. Similarly, the second one allows us to write the electric field in term of a scalar potential $\phi$
\[
\bm{B} = \nabla {\times} \bm{A} \quad , \qquad \bm{E} = - \left( \frac{1}{c} \frac{\partial \bm{A}}{\partial t} + \nabla \phi \right) .
\]
%
However, these equations do not fix the potential since every potential
that differs from these in the form
%
\begin{align*}
  \phi'(\bm{r},t) = \phi(\bm{r},t) - \frac{1}{c} \frac{\partial \xi(\bm{r},t)}{\partial t} \,,\\
  \bm{A}'(\bm{r},t) = \bm{A}(\bm{r},t) +  \nabla \xi(\bm{r},t)
\end{align*}
do not alter the magnetic and electric fields. The choice of the particular vector and scalar potential is called the ``gauge''.

The observable magnitudes are independent of the gauge chosen \autocite{Galindo1990_QMvI}. In many cases it is useful to choose the so-called Coulomb gauge in which
\begin{equation}\label{Q:Coul-Gaug}
  \nabla \bm{A} = 0 \, , \qquad \phi = 0 \,.
\end{equation}
%
Thus, the Maxwell equation result in
\begin{eqnarray}\label{Q:Maxwe-EZ-2}
\bm{B} &=& \nabla {\times} \bm{A} \label{Q:Maxwe-EZ-1}\\
\bm{E} &=& -\frac{1}{c} \frac{\partial \bm{A}}{\partial t}
\\
0 &=& \nabla^{2} \bm{A} - \frac{1}{c^{2}} \frac{\partial^{2}
\bm{A}}{\partial t^{2}} \label{Q:Maxwe-EZ-3}
\\
  0 &=& \nabla  \cdot \bm{A} \label{Q:Maxwe-EZ-4}
\end{eqnarray}

A specific solution of the wave equation \ref{Q:Maxwe-EZ-3} is
\begin{equation}\label{Q:Monoc-field}
\bm{A} = \bm{A}_{0}\, e^{i (\bm{k} \cdot \bm{r} - \omega t)} 
\end{equation}
%
with $\omega = k c$. Additionally, the Coulomb gauge (expressed by \ref{Q:Maxwe-EZ-4}) imposes the constraint $\bm{k} \cdot \bm{A}_{0} =
0$. Thus, the wave is transverse to the direction of propagation $\bm{k}$.

The most general solution of the Maxwell equation is a superposition of these monochromatic waves
\begin{equation}\label{Q:Gener-field}
\bm{A} = \frac{1}{\sqrt{V}}\sum_{\bm{k}} \left[ \sum_{\lambda} c_{\bm{k},\lambda} \, \hat{\epsilon}_{(\bm{k},\lambda)} e^{i (\bm{k} \cdot \bm{r} - \omega t)} + c^{*}_{\bm{k},\lambda} \, \hat{\epsilon}_{(\bm{k},\lambda)} e^{- i (\bm{k} \cdot \bm{r} - \omega t) } \right] 
\end{equation}
%
where the vector $\hat{\epsilon}$ is perpendicular to the propagation vector $\bm{k}$. $\lambda=1,2$ are the two components of $\hat{\epsilon}$ in this perpendicular plane such that $\bm{k} \cdot \hat{\epsilon}_{(\bm{k},\lambda)}=0$. $V$ is the volume of a box of finite dimensions $L_{x},L_{y},L_{z}$ used to normalize the field. The conjugate term in \ref{Q:Gener-field} is included in order to get a real potential vector.

From the Maxwell equation the electric and magnetic fields are
\begin{subequations}
  \begin{align}\label{Q:Elect-Magne-field}
    \bm{E} =& \frac{1}{\sqrt{V}}\sum_{\bm{k},\lambda}\frac{\omega}{c} \; \hat{\epsilon}_{(\bm{k},\lambda)} \left[ c_{\bm{k},\lambda} \,e^{i (\bm{k} \cdot \bm{r} - \omega t)} + c^{*}_{\bm{k},\lambda} \, e^{- i (\bm{k} \cdot \bm{r} - \omega t) } \right]
    \\
    \bm{B} =& \frac{1}{\sqrt{V}}\sum_{\bm{k},\lambda} \left(\bm{k} {\times} \hat{\epsilon}_{(\bm{k},\lambda)} \right) \left[ c_{\bm{k},\lambda} \, e^{i (\bm{k} \cdot \bm{r} - \omega t)} + c^{*}_{\bm{k},\lambda} \, e^{- i (\bm{k} \cdot \bm{r} - \omega t) } \right]
  \end{align}

\end{subequations}
Observe that in absence of charges and currents, the electromagnetic field energy is given by
\begin{equation}\label{Q:Energ-EM-free}
H_{\mathrm{E.M.}} = \frac{1}{8 \pi} \int d \bm{r} \left[ \left(
-\frac{1}{c} \frac{\partial \bm{A}}{\partial t}\right)^{2} + \left(
\nabla {\times} \bm{A} \right)^{2} \right] = \frac{1}{8 \pi} \int d \bm{r}
\left( E^{2} + B^{2} \right)
\end{equation}


\subsection{Linear, circular and elliptically polarizations}

\section{Hamiltonian of a particle in an E.M. field}

Lets consider a spinless particle of chage $Z$ immersed in an E. M.  field characterized by potentials $\phi, \bm{A}$ and an external potential $V(r)$. The classical hamiltonian is given by \textcite[][App.~E]{Townsen2000_AMA}
\begin{equation}\label{Q:Clas-Hamil-EM}
H = \frac{1}{2 m} \left( \bm{p} - \frac{Z \bm{A}}{c} \right)^{2} + Z \, \phi + V(r)
\end{equation}
%

Thus, the quantum mechanical time-dependent Schr\"{o}dinger equation (TDSE) is straightforwardly obtained
\begin{equation}\label{Q:QM-Hamil-EM}
i \hbar \frac{\partial \Phi}{\partial t} = \left[ \frac{1}{2 m} \left( - i \hbar \nabla - \frac{Z}{c}\bm{A} \right)^{2} + Z \phi + V(\bm{r})\right] \Phi \,.
\end{equation}


\subsection{Gauge Invariance and Gauge Choice}
\label{S:gauge-invar-choice}

This equation is invariant to gauge transformations. This means that for a given function $\xi(\bm{r},t)$ the above Schr\"{o}dinger equation does not change its form if we replace
\begin{subequations}
  \begin{align}
    \label{Q:ph-gauge-inv-phi}  \phi'(\bm{r},t) &= \phi(\bm{r},t) - \frac{1}{c} \frac{\partial \xi(\bm{r},t)}{\partial t} \,,\\
    \label{Q:ph-gauge-inv-A}  \bm{A}'(\bm{r},t) &= \bm{A}(\bm{r},t) +  \nabla \xi(\bm{r},t)\\
    \label{Q:ph-gauge-inv-wf} \Phi'(\bm{r},t) &= \Phi(\bm{r},t) \, e^{i\, Z/\hbar c\,
      \xi(\bm{r},t)}
  \end{align}
\end{subequations}
Two commonly used gauges are the velocity and length gauges. The Hamiltonian keeps its form (\ref{Q:Clas-Hamil-EM}) with the fields
\begin{subequations}
  \begin{align}
    &\text{velocity} & &\text{length} \notag \\
    \xi^{(v)}(\bm{r},t)&= (Z/2 m c) \int A^{2}(t') d t' & \xi^{(l)}(\bm{r},t)&= - \bm{A}(t) \cdot \bm{r} \\
    \bm{A}^{(v)}(\bm{r},t)&= \bm{A}(\bm{r},t) & \bm{A}^{(l)}(\bm{r},t) &= 0
    \\
    \phi^{(v)}(\bm{r},t)&= \phi(\bm{r},t) - \frac{Z}{2 m c^{2}} A^{2}(\bm{r},t) &
    \phi^{(l)}(\bm{r},t)&= \phi(\bm{r},t) + \frac{1}{c} \frac{\partial \bm{A}(t)}{\partial
      t} \cdot \bm{r}
  \end{align}
\end{subequations}

\subsection{Detailed calculations for velocity and length gauges}
\label{S:deta-calc-veloc-lengt-gauge}

Although is easy to introduce the above fields in the Schr\"{o}dinger equation and obtain the hamiltonian for the two gauges, we will show do the calculations.
In order to rewrite the Schr\"{o}dinger equation in terms of a (time dependent) force we write the wavefunction as:
\begin{equation}
  \label{Q:ph-volkov-1}
  \Psi(\bm{r},t) = e^{-i \alpha(\bm{r},t)} \, \Phi(\bm{r},t)
\end{equation}
and insert it in the Schr\"{o}dinger equation (\ref{Q:QM-Hamil-EM}). On the right hand side we have:
\begin{equation*}
\hat{H} \Phi(\bm{r},t) = \hat{H}  e^{i \alpha(\bm{r},t)} \, \Psi(\bm{r},t) = \left[ \frac{1}{2 m}
\left( - i \hbar \nabla - \frac{Z}{c}\bm{A} \right)^{2} + Z \phi + V(\bm{r})\right]  e^{i \alpha(\bm{r},t)} \, \Psi(\bm{r},t) 
\end{equation*}
%
First, let's investigate the left side of the Schr\"{o}dinger equation:
\begin{equation}
  \label{Q:ph-lhs-sch-eq}
    i \hbar \frac{\partial \Phi}{\partial t} = e^{i \alpha(\bm{r},t)} \, \left(   i \hbar \frac{\partial \Psi(\bm{r},t)}{\partial t} - \hbar \frac{d \alpha}{d t} \Psi(\bm{r},t) \right) 
\end{equation}

The operation of the ``momentum'' on this wavefunction is:
\begin{align}\label{Q:ph-momentum-action}
  \left( - i \hbar \nabla - \frac{Z}{c}\bm{A} \right)\, e^{i \alpha(\bm{r},t)} \, \Psi(\bm{r},t)&= e^{i \alpha(\bm{r},t)} \,  \left\{  - i \hbar  \left[ i (\nabla \alpha) \Psi(\bm{r},t) +  \nabla \Psi(\bm{r},t) \right] - \frac{Z}{c}\bm{A} \Psi(\bm{r},t)  \right\} \nonumber \\
  &= \left[ \hbar (\nabla \alpha) - \frac{Z}{c}\bm{A} \right]\,e^{i \alpha(\bm{r},t)}
  \Psi(\bm{r},t) - i \hbar \, e^{i \alpha(\bm{r},t)} \, \nabla \Psi(\bm{r},t)
\end{align}
 %
The choice of the exponent $\alpha$ is what it is called \textbf{gauge} choice.



\subsubsection{Velocity Gauge}
\label{S:velocity-gauge}
The \textbf{velocity gauge} is chosen such that the term quadratic in the vector field, $A^{2}$ is removed from the Schr\"{o}dinger equation. It is easy to show that this is accomplished by choosing
\begin{equation}
\label{Q:ph-alpha-vel} \alpha(\bm{r},t)\equiv \alpha_{(v)}= \frac{-Z^{2}}{2 m \hbar c^{2}}  \,\int  A^{2}(t') \, dt' \,.
\end{equation}

With this choice we have:
\begin{align}
  \label{Q:ph-alpha-vel-deriv-t}
  \frac{d \alpha}{d t} &= \frac{-Z^{2}}{2 m \hbar c^{2}} \, A^{2}(t)& \text{and} &&  \nabla \alpha &= 0
\end{align}
and the momentum acting on the wavefunction may be written from (\ref{Q:ph-momentum-action}) as

\begin{align*}
  \left( - i \hbar \nabla - \frac{Z}{c}\bm{A} \right)^{2} \, e^{i \alpha(\bm{r},t)} \, \Psi(\bm{r},t) &= \left( - i \hbar \nabla - \frac{Z}{c}\bm{A} \right)\left[ - \frac{Z}{c}\bm{A} \,e^{i \alpha(\bm{r},t)} \Psi(\bm{r},t) - i \hbar \, e^{i \alpha(\bm{r},t)} \, \nabla \Psi(\bm{r},t) \right] \\
&=  e^{i \alpha(\bm{r},t)} \,\left( \frac{Z^{2}}{c^{2}} A^{2} + 2 i \hbar \frac{Z}{c} \bm{A} \cdot \nabla \Psi - \hbar^{2} \nabla^{2} \Psi  \right)
\end{align*}
%
while replacing (\ref{Q:ph-lhs-sch-eq}) in the L.H.S gives
\begin{equation*}
  i \hbar \frac{\partial \Phi}{\partial t} = e^{i \alpha(\bm{r},t)} \, \left( i \hbar \frac{\partial \Psi(\bm{r},t)}{\partial t} + \frac{Z^{2}}{2 m  c^{2}} \, A^{2}(t)  \Psi(\bm{r},t) \right) \,.
\end{equation*}

Replacing these expresions in the Schr\"{o}dinger equations we obtain
\begin{equation}
  \label{Q:ph-schrod-gauge-vel}
    i \hbar \frac{\partial \Psi(\bm{r},t)}{\partial t} = \left[ \frac{\hat{\bm{p}}^{2}}{2m} +  Z \phi +  V(\bm{r}) + \frac{i \hbar Z}{m c} \bm{A}(t) \cdot \nabla \right]  \Psi(\bm{r},t)  \,.
\end{equation}


\subsubsection{Length Gauge}
\label{S:length-gauge}
We can make vanish the first term if we choose $\alpha$ such that $\nabla \alpha = (Z/\hbar c) \bm{A}$. For a simple potential vector that does not depend on $\bm{r}$ this gives
\begin{equation}
\label{Q:ph-alpha-len}  \alpha(\bm{r},t) \equiv \alpha_{(l)}= \frac{Z}{\hbar c}\, \bm{A}(t) \cdot \bm{r}
\end{equation}
%
With this choice we obtain
\begin{eqnarray*}
  \left( - i \hbar \nabla - \frac{Z}{c}\bm{A} \right)^{2}\, e^{i \alpha(\bm{r},t)} \, \Psi(\bm{r},t)&=& -i \hbar \left( - i \hbar \nabla - \frac{Z}{c}\bm{A} \right) e^{i \alpha(\bm{r},t)} \, \left( \nabla \Psi(\bm{r},t) \right) \\ 
&=&  e^{i \alpha(\bm{r},t)} \, (-i \hbar)^{2}  \nabla^{2} \Psi(\bm{r},t) =  e^{i \alpha(\bm{r},t)} \, \hat{\bm{p}}^{2} \Psi(\bm{r},t)
\end{eqnarray*}
and the right hand side of the Hamltonian (\ref{Q:QM-Hamil-EM}) reads
\begin{equation}
  \label{Q:ph-Hamilt-EM-1}
  \hat{H} \Phi(\bm{r},t)=  e^{i \alpha(\bm{r},t)} \,\left[ \frac{\hat{\bm{p}}^{2}}{2m} +  Z \phi + V(\bm{r})\right] \Psi(\bm{r},t)
\end{equation}

On the left hand side we obtain
\begin{equation*}
  i \hbar \frac{\partial \Phi}{\partial t} = i \hbar \frac{\partial \left( e^{i \alpha(\bm{r},t)} \, \Psi(\bm{r},t) \right)}{\partial t} = \hbar \,  e^{i \alpha(\bm{r},t)}\, \left(  i \frac{\partial \Psi(\bm{r},t)}{\partial t} - \frac{d \alpha}{d t} \Psi(\bm{r},t) \right) 
\end{equation*}

Now, combining this equation with eqs.~(\ref{Q:ph-alpha-len},~\ref{Q:ph-Hamilt-EM-1}) we obtain
\begin{equation}
  \label{Q:ph-hamil-EM-force}
  i \hbar \frac{\partial \Psi(\bm{r},t)}{\partial t} = \left[ \frac{\hat{\bm{p}}^{2}}{2m} +  Z \phi +  V(\bm{r}) + \bm{F}(t) \cdot \bm{r} \right]  \Psi(\bm{r},t) 
\end{equation}
where
\begin{equation*}
  \bm{F}(t) \equiv - Z \bm{E}(t) = -Z\, \left(- \frac{1}{c} \frac{\partial \bm{A}(t)}{\partial t} \right)
\end{equation*}


\subsection{Free particle immersed in a E.M. field: length gauge}
\label{S:Free-parti-immer-E.M.-field}

We first start studying the effect of a radiation field in a free particle. As expressed by \ref{Q:QM-Hamil-EM} the hamiltonian of a quantum system is obtained by replacing $\hat{\bm{p}} \to \hat{\bm{p}}-( Z / c ) \bm{A}$. As shown in the previous section, in the length gauge this Hamiltonian may in turn be converted to
\begin{equation}\label{Q:Schro-eq-free-parti}
  \hat{H} = \frac{\hat{\bm{p}}^{2}}{2m} + \bm{F}(t) \cdot \bm{r}  \quad  , \qquad \qquad  \bm{F}(t)= Z\, \frac{1}{c} \frac{\partial \bm{A}(t)}{\partial t} 
\end{equation}
%
A solution for the Time Dependent Schr\"{o}dinger Equation (TDSE) can be proposed as
\[
\psi(\bm{r},t) = \exp{\left\{ (i/\hbar)\left(  \bm{K}(t) \cdot \bm{r} - \int_{t_{0}}^{t}\frac{\bm{K}(t')^{2}}{2m} dt' \right) \right\}}
\]
where $\bm{K}(t)$ is still a function to determine. Applying the hamiltonian to this wavefunction we obtain
%
 \begin{eqnarray*}
  i \hbar \frac{\partial \psi}{\partial t} &=& \left(  \frac{\hat{\bm{p}}^{2}}{2m} + \bm{F}(t) \cdot \bm{r}  \right) \psi(\bm{r},t) \\
\left[ \frac{K(t)^{2}}{2m} - \frac{\partial \bm{K}(t)}{\partial t} \cdot \bm{r} \right] \psi(\bm{r},t)  &=& \left[ \frac{K(t)^{2}}{2m}  + \bm{F}(t) \cdot \bm{r} \right] \psi(\bm{r},t) 
\end{eqnarray*}
Thus, the unknown function must verify
\begin{equation}
\label{Q:ph-volkov-K}  \frac{\partial \bm{K}(t)}{\partial t} = - \bm{F}(t)= Z\, \frac{1}{c} \frac{\partial \bm{A}(t)}{\partial t}  \qquad \Rightarrow \qquad  \bm{K}(t)= - \frac{Z}{c} \bm{A}(t) + \mathrm{cte} \, .
\end{equation}
%
The constant must be determined from boundary (initial or final) conditions. For instance if we require that for earlier times the field is not active $A(-\infty)=$0, the state must be a plane wave with momentum $\bm{k}$, giving the so-called Volkov states
\begin{equation} \label{Q:ph-volkov-state}
\psi(\bm{r},t) = \exp{\left\{ (i/\hbar)\left[\left( \bm{k} - \frac{Z}{c} \bm{A}(t) \right)\cdot \bm{r} - \frac{1}{2m} \,\int_{t_{0}}^{t} \left( \bm{k} - \frac{Z}{c} \bm{A}(t) \right)^{2} dt' \right] \right\}}  \,.
\end{equation}

\subsection{Free particle in the velocity gauge}
\label{S:free-part-veloc}

In the velocity gauge the Hamiltonian of a ``free-particle'' is
\begin{equation*}
  \hat{H} = \frac{\hat{\bm{p}}^{2}}{2m} + i \frac{Z \hbar}{m c}\, \bm{A}(t) \cdot \nabla
\end{equation*}
By using that the solutions in different gauges are related we can obtain the solution to this equation
\begin{align}
  \psi^{(v)}(\bm{r},t) &= e^{i (\alpha_{(l)} - \alpha_{(v)})} \psi^{(l)} = \exp{\left[i \frac{Z}{\hbar c} \bm{A} \cdot \bm{r} + i \frac{Z^{2}}{2 m \hbar c^{2}}  \right]} \times \psi^{(l)}(\bm{r},t) \\
&= e^{i \bm{k}\cdot \bm{r} - E_{k}t/\hbar} \, \exp{\left[ i \frac{Z}{\hbar m c} \bm{k} \cdot \int_{t_{0}}^{t} \bm{A}(t') \, dt' \right]}
\end{align}

It is straightforward to verify the this wavefunction verifies the time-dependent Schr\"{o}dinger equation.

\section{Interaction of photons with atoms}

Having obtained an expression for the external field $\bm{A}$, we consider now its effect on a quantum system. The hamiltonian for the system can be decomposed in the sum of three terms: the atomic $H_{At}$, the electromagnetic $H_{EM}$ (given by \ref{Q:Energ-EM-free}) and a final term describing the interaction between these two systems $H_{I}$. Expanding the kinetic energy term in eq. \ref{Q:QM-Hamil-EM} we obtain for the interaction term
%
\begin{eqnarray}\label{Q:H_int-EM}
H_{I} = \frac{i \hbar Z}{m c}\, \bm{A} \cdot \nabla +
\frac{Z^{2}}{c^{2}} \, A^{2}
\end{eqnarray}
%
where we have used that in the Coulomb gauge $\nabla \bm{A}=0$.

Let's consider a system (an atom for instance) initially in a state $n_{i}$. The probability of finding it in a state $n_{f}$ at a posterior time $t$ is, at first order,
\begin{equation}\label{Q:first-order-trans-ampli}
a_{fi}^{(1)}(t) = \frac{1}{i \hbar} \int_{0}^{t} \langle
f|H_{I}(t')| i \rangle \; e^{i \omega_{fi} t'} d t'
\end{equation}
where $\omega_{fi} = (E_{f}-E_{i})/\hbar$ and the states $| i\rangle, | f\rangle$ are eigenvectors of the isolated hamiltonian $H_{0} = H_{At} + H_{EM}$. Using expression \ref{Q:Gener-field} the transition probability can be written for each monochromatic wave as
\begin{equation}\label{Q:first-order-trans-ampli-1}
a_{fi}^{(1)}(t) = - \frac{T'_{fi}}{\hbar} \frac{e^{i (\omega_{fi}-\omega)t} -1}{\omega_{fi} - \omega} \, e^{i \theta}
- \frac{T''_{fi}}{\hbar} \frac{e^{i (\omega_{fi} + \omega)t} -1}{\omega_{fi} + \omega} \, e^{- i \theta}
\end{equation}
%
where
\begin{eqnarray} \label{Q:Tif-EM-1}
T'_{fi} &=& \frac{i Z \hbar}{m c} |c_{\bm{k},\lambda}| \int u^{*}_{f} \; e^{i \bm{k} \cdot \bm{r}} \;
\hat{\epsilon}_{(\bm{k},\lambda)}  \cdot \nabla u_{i} \; d \tau
\\
T''_{fi} &=& \frac{i Z \hbar}{m c} |c_{\bm{k},\lambda}| \int u^{*}_{f} \; e^{- i \bm{k} \cdot \bm{r}} \;
\hat{\epsilon}_{(\bm{k},\lambda)}^{*} \cdot \nabla u_{i} \; d \tau
\end{eqnarray}
where, for simplicity we have included the phase of $c_{k,\lambda}$ to the polarization vectors $\hat{\epsilon}_{(\bm{k},\lambda)}$.

Each of the two terms in expression \ref{Q:first-order-trans-ampli} has a maximum when $\omega_{fi} = \omega $, indicating that the change in the energy of the quantum system is quantized
%
\[ E_{f} = E_{i} \pm \hbar \omega \,. \]
%
Observe that the quantization of the field is not necessary because the atomic system only can exchange quanta of energy. The two terms in equation \ref{Q:Tif-EM-1} correspond to absorbtion ($T'$) and induced emission ($T''$) of one photon from the radiation field.

Since the density of states in a box is given by
\begin{equation}\label{Q:Densi-state}
\rho ( k_{f}) = \frac{V}{(2 \pi \hbar)^{3}}\frac{k_{f}^{2 d
k_{f}}}{V} d \Omega_{f}
\end{equation}
%
the rate of transition from initial state $i$ to a final $f$ by absorption of radiation is
\begin{equation}\label{Q:Trans-Rate}
W = \frac{Z^{2}}{(2\, \pi \, \hbar \, c)^{2}} \frac{k_{f}}{m}
|c_{\bm{k}, \lambda}|^{2}
  \left|\int
u_{f}^{*} e^{i \bm{k} \cdot \bm{r}} \, \hat{\epsilon} \cdot
\nabla u_{i} d \tau \right|^{2} \, V \,d \Omega_{f}
\end{equation}

In the case of non-monochromatic radiation we must sum over all the terms expressed in \ref{Q:Gener-field}.

\section{Quantization of the radiation field}


\section{Multipole expansion}
\label{S:Multi-expan}

We expand the exponential in the computation of $T'$ and keep only the few first terms leading to a non-vanishing integral. The limit of validity and justification of this approximation can be obtained by estimating the values of $r$ at which the integrand in \ref{Q:first-order-trans-ampli} contributes significantly.  
%
Because the initial wavefunction $u_{i}$ corresponds to an bound atomic state, the typical values of $r$ are of the order of 1~a.~u. Since $k = 2 \pi/\lambda$, $k\,r = 2 \pi r/ \lambda$ and while the order of magnitude of the wavelength of radiation available in laboratories is at about a hundred nanometers (100~nm = 1000~\AA $\approx$ 2.~10$^{4}$~a.~u.), typical values for the argument in the exponential are $k\,r\approx 10^{-3}$. For these small values, the exponential can be expanded in a Taylor series
\[
e^{-i \bm{k} \cdot \bm{r}} = 1 - i \bm{k} \cdot \bm{r} + \frac{\left( i \bm{k} \cdot \bm{r}\right)^{2}}{2!} - \frac{\left(
i \bm{k} \cdot \bm{r}\right)^{3}}{3!} + \dots
\]

\subsection{Electric dipole approximation}

The electrical dipole approximation is obtained by retaining the first (constant) term
\[ e^{-i \bm{k} \cdot \bm{r}} \to 1 \, . \]
%
Thus, the integral in equations \ref{Q:Trans-Rate} can be written as
%
\[
I_{fi} = \hat{\epsilon}  \cdot \, \int u^{*}_{f}(\bm{r}) \, \nabla u_{i}(\bm{r}) d \bm{r}
\]
%
which, we will inmediately see, is equivalent to the action of a perturbation potential depending on an electric dipole
\begin{equation}\label{Q:dipol-approx-pertur-1}
  H_{I} = - \vec{\bmu}_{e}  \cdot \breve{\bm{E}} \, .
\end{equation}

In order to derive this result, lets remember that the gradient is
proportional to the momentum operator $\breve{\bm{p}} \equiv {(i /
\hbar)} \nabla $ and its commutator with the position operator is
simply $ [ \breve{\bm{r}}, \breve{\bm{p}} ] = i \hbar $. Then, the
commutator of $\breve{\bm{r}}$ and the hamiltonian $\breve{H} =
\breve{\bm{p}}^{2}/2 m + V(\breve{\bm{r}})$ is simply
%
\[
[\breve{H},\breve{\bm{r}}] = \frac{1}{2 m} \left[ \breve{\bm{p}}^{2} ,
\breve{\bm{r}} \right] = \frac{\breve{\bm{p}}}{m} [ \breve{\bm{p}} ,
\breve{\bm{r}} ] = - \frac{i \hbar}{m} \, \breve{\bm{p}} \,.
\]

The above integral can be written
\begin{eqnarray*}
I_{fi} &=& \frac{\hbar }{i} \hat{\epsilon}  \cdot \langle u_{f} |
\breve{\bm{p}}| u_{i}\rangle \\
&=& - \frac{m}{i \hbar} \frac{\hbar }{i} \hat{\epsilon} \cdot
\langle
u_{f} | [\breve{H},\breve{\bm{r}}] | u_{i}\rangle \\
&=& m\, \hbar \, \frac{\left(E_{f} - E_{i}\right)}{\hbar}
\hat{\epsilon}
\cdot \langle u_{f} |\bm{r} | u_{i}\rangle \\
\end{eqnarray*}
%
This last result corresponds to make the replacement $H_{I} \to q
\bm{r} \cdot \bm{E} \equiv - \vec{\bmu}_{e} \cdot \bm{E}$, giving a
transition rate proportional to the vector product of the polarization
and the matrix element
\[
\hat{\epsilon} \cdot \langle u_{f}|\bm{r}| u_{i}\rangle
\]

\subsubsection{Classical interpretation}

To do, see \citet{Bethe1997AC_IQM}.

\subsubsection{Line breadth of atoms}

Example for 2p state of hydrogen \autocite[see][]{Bethe1997AC_IQM,Townsen2000_AMA}


\subsection{Magnetic dipole and Electric Quadrupole Transitions}



\subsection{Antihydrogen formation}

%%% Local Variables: 
%%% mode: latex
%%% TeX-master: "mainxs"
%%% End: 
