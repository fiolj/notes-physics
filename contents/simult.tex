\chapter{Simultaneous ionization from different centers}

\section{Hamiltonian of the problem}

We start by considering a collision between two atomic species, each of them
having one or more electrons. The corresponding Hamiltonian is given by
\cite{Fiol2001JPBpL503}
\begin{eqnarray}\label{Q:simH1}
H &=& H_{0} + V_{N_{P},e_{P}} + V_{N_{T},e_{T}} +  V_{N_{P},N_{T}} +
V_{N_{T},e_{P}} + V_{N_{P},e_{T}} + V_{e_{P}, e_{T}} \nonumber \\
&=& H_{P} + H_{T} + V_{N_{P},N_{T}} + V_{N_{T},e_{P}} + V_{N_{P},e_{T}} +
V_{e_{P}, e_{T}} \nonumber \\
&=& H_{P} + H_{T} + V_{N_{P},N_{T}}
\\
&+& \left[ V_{N_{T},e_{P}} + \langle V_{e_{P}, e_{T}} \rangle_{T} \right] +
\left[ V_{N_{P},e_{T}} + \left\langle V_{e_{P}, e_{T}} \right\rangle_{P}
\right] + \left[  V_{e_{P}, e_{T}} - \langle  V_{e_{P}, e_{T}} \rangle_{T} -
\langle  V_{e_{P}, e_{T}} \rangle_{P} \right] \nonumber
\end{eqnarray}
Here $H_{P,T}$ are the hamiltonian of each center, $\langle V_{e_{T},e_{P}}
\rangle $  is the interaction of projectile electrons with the mean field
produced by the target electrons. The first term between square brackets
model the screened nuclear interaction felt by the projectile electrons. A
similar interpretation can be made of the second set of square brackets. It
is the average potential felt by the target electrons due to the screened
projectile. The last term accounts by the difference between mean fields and
the real interactions. It can be interpreted as the dynamical contribution of
the e-e interaction, since the static part is contained in the two first
brackets.

We model the screened interaction of each center with each electrons in the
other center through a single potential $\mathcal{V}_{T,P}$. Thus the
hamiltonian is
\[
H = H_{P} + H_{T} + V_{N_{T},N_{P}} + \sum_{i_{P}} \mathcal{V}_{T,e_{P}} +
\sum_{i_{T}} \mathcal{V}_{P,e_{T}} + \left[  V_{e_{P}, e_{T}} - \langle
V_{e_{P}, e_{T}} \rangle_{T} - \langle  V_{e_{P}, e_{T}} \rangle_{P} \right]
\]

\section{Modelling of simultaneous target and projectile ionization}


We are interested in the simultaneous target and projectile ionization by the
collision of two atomic/ionic species at large relative velocities. In this
range of impact energy the First Born approximation (FBA) is valid for
describing single ionization in three-body problems.

In the three-body FBA the internuclear potential can be safely switched-off
for describe cross sections related to the electron momentum. The final state
in this approximation include the (free) motion of the projectile, while the
electron does not interact with the projectile in the final state.

Now we will try to generalize the FBA for the process of interest including,
at least, four bodies. Additionally we will incorporate some of the
interactions neglected in the FBA calculations. We consider first the
simplest case in which the target and projectile contain only one electron.
The transition matrix element for the simultaneous ionization in a
perturbative approach is given by $t_{if} = \langle \Psi^{*-}_{f} \mid V\mid
\Phi_{i} \rangle$

\begin{eqnarray}\label{Q:Sim1}
\Psi^{\pm}_{f} &=& e^{i (\bm{k}_{T} \cdot \bm{r}_{T} + \bm{k}_{T,2}
\cdot \bm{r}_{T,2} + \bm{K}_{T} \cdot \bm{R}_{T})} \, D^{\pm}_{\nu_{T,1}}\,
D^{\pm}_{\nu_{P,2}}\,\left[  D^{\pm}_{\nu_{1,2}} \right] \, \left[
D^{\pm}_{\nu_{T,P}}\, D^{\pm}_{\nu_{T,2}}\, D^{\pm}_{\nu_{P,1}} \right]
\nonumber \\
\Phi_{i} &=& e^{i \bm{P}_{T} \cdot \bm{R}_{T}} \;
\phi^{f}_{T}(\bm{r}_{T,1}) \phi^{f}_{P}(\bm{r}_{P,2})
\nonumber \\
\\
V &=&  V_{N_{T},e_{P}} + V_{N_{P},e_{T}} + V_{e_{P}, e_{T}} + \left[
V_{N_{P},N_{T}} \right] \, . \nonumber
\end{eqnarray}

The terms between square brackets will be neglected in the following
calculations. This approximation is the equivalent to the FBA for four
bodies. While the final velocity of the projectile electron (electron 2) is
expected to be close to its initial velocity (projectile incident velocity)
and the target electron  (electron 1) velocity is expected to be small, the
interaction of each electron with the opposite center is neglected in the
final state.


In the case of atoms with more than one electron, the final state has to be
modified by the inclusion of wavefunctions describing the internal state of
the projectile and target (discarding the terms between brackets)
\[
\Psi^{\pm}_{f} = e^{i (\bm{k}_{T} \cdot \bm{r}_{T} + \bm{k}_{T,2} \cdot
\bm{r}_{T,2} + \bm{K}_{T} \cdot \bm{R}_{T})} \, D^{\pm}_{\nu_{T,1}}\,
D^{\pm}_{\nu_{P,2}}\, D^{\pm}_{\nu_{1,2}} \; \phi^{f}_{T} \; \phi^{f}_{P}
\]


\subsection{Evaluation of the transition matrix}

The transition matrix for a process in which the projectile and target change
from initial states $\mid \Phi_{P,\,i} \rangle $, $\mid \Phi_{T,\,i} \rangle
$ to final states $\mid \Phi_{P,\,f} \rangle$ and $\mid \Phi_{T,\,f} \rangle$
is given by
\begin{eqnarray}\label{Q:Sim2}
t_{if} = \langle \Phi_{P,\,f} \mid
\end{eqnarray}



\section{Stripping cross sections at high energy}

\subsection{First Born Approximation (FBA)}

The simplest approximation in the evaluation of total cross sections is given
by the FBA. In a first order we will neglect most of multi electron effects
together with the effect of dynamical electron-electron interactions.

Within this model the problem simplifies to the projectile ionization by the
collision with a structureless target. The interaction between the two
projectile particles and the target is given by a screened potential
modelling the average effect of the nucleus and all target electrons.

The transition matrix is given by

\[
t_{if} = \langle \mid
\]


%%% Local Variables: 
%%% mode: latex
%%% TeX-master: "main"
%%% End: 
