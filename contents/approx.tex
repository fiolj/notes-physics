\chapter{A time-dependent approximation for fast ion-atoms collisions}


We want to reproduce the experimental and CTMC triply differential
cross sections (see paper) for the collision $\mathrm{C}^{6+} +
\mathrm{He}$ at $100~\mathrm{MeV/u}$. The CTMC results indicate that
the electron is `kicked' by the projectile. In a very good
approximation the projectile transfer momentum only during a short time
($\Delta t \approx 0.2-0.4${~a.~u.}).  While the projectile acts,
the electron momentum changes but its position is not sensibly
modified.

Seems like the above mechanism should be well described by very simple
approximations, including the First Born Approximation. However,
neither the FBA, the C3 are able to reproduce the experimenta/CTMC
results.
%
We think that the use of a target continuum wavefunction in the final
state implies a scattering state (particles impinging from infinite).

Let's consider the fast collision as a three step process. In the
initial stage the electron and its parent nucleus are bound and evolve
in the field of their mutual interaction. Then, the fast, heavy and
highly-charged projectile comes into scene dominating the dynamics of
the electron motion. In the third step the projectile interaction
vanishes as suddenly as it appeared. Thus the electron and recoil ion
evolve in their mutual interaction as in the initial state.
%
This simple mechanism seems to correspond to the description given by
the First Born Approximation. However as stated before, in the final
state --corresponding to the third step-- is implicitly supposed that
the electron evolving in the target nucleus field comes from infinity.

Moreover, in order to obtain an electron out of the scattering
($\bm{Q},\bm{v}$) plane, it must be dispersed at $90^{\circ}$ by the
target nucleus. Classically the impact parameter $b_{T}$ between the
electron and the recoil in this last step (for the helium model
potential) must be larger than the size of the atom if the electron
comes from infinity. On the contrary, if we allow it to start at a
finite distance ($\approx \langle r \rangle_{\mathrm{He}}\equiv$ the
expectation value of the initial state), the impact parameter is only
0.83~{a.~u.}

\medskip

Following is the first attempt to model the above behavior. I am still
trying to connect it to the off-shell states. I expect to find
something very similar to the FBA but with the final state replaced by
the off-shell state with an energy shift $\Delta E \approx V(\approx
\langle r \rangle_{\mathrm{He}})$.

We will next to investigate this scenario within a time-dependent
framework.

The wavefunction in the three stages just described can be written as
\begin{align}\label{Q:wave-funct-stage}
\Psi_{1}&= e^{i \mu_{T} \bm{v} \cdot \bm{R}_{T}} \,
\phi_{i,T}(\bm{r}_{T}) e^{- i E t}
\nonumber \\
\Psi_{2} &= e^{i \bm{K}_{P} \cdot \bm{R}_{P}} \sum_{\alpha_{P}}
b_{\alpha_{P}}(t) \; \phi_{\alpha_{P}, P}(\bm{r}_{P})
% e^{i E_{\alpha_{P}} t}
\\
\Psi_{3} &= e^{i \bm{K}_{T} \cdot \bm{R}_{T}} \sum_{\alpha_{T}}
c_{\alpha_{T}}(t) \; \phi_{\alpha_{T}, T}(\bm{r}_{T}) \nonumber
%e^{i E_{\alpha_{T}} t}
\end{align}
where $E$ is the total center of mass energy and the time dependence
has been included in the coefficients $b_{\alpha_{P}}$ and
$c_{\alpha_{T}}$.

We choose time $t=0$ when the projectile appears and time $t=\tau $
when its potential stop acting. At both instants, the concurrent pieces
of the wavefunction must coincide
%
\begin{eqnarray}\label{Q:bound-cond}
\Psi_{1}(0) &=& \Psi_{2}(0) \nonumber
  \\
\Psi_{2}(\tau) &=& \Psi_{3}(\tau) \,.
\end{eqnarray}

The probability of the electron being ionized with momentum
$\bm{k}_{e}$ while the projectile ends up with momentum $\bm{K}$ is
given by

\begin{equation}\label{Q:Prob-ion-1}
\mathcal{P}_{if} = \lim_{t \to \infty} \left|\langle \Psi_{3} |
\Phi_{f} \rangle\right|^{2}
\end{equation}
%
where
\begin{equation} \label{Q:fi-f-a}
\Phi_{f} = e^{i \left( \bm{K}_{T} \cdot \bm{R}_{T} + \bm{k}_{T}
\cdot \bm{r}_{T}\right)}
\end{equation}
%
or, by considering the ionization as excitation to the continuum we
must use
%
\begin{equation}\label{Q:fi-f-b}
\Phi_{f} = e^{i \bm{K}_{f} \cdot \bm{R}_{T}} \,
\phi_{\bm{k}_{T},T}(\bm{r}_{T})
\end{equation}

We will use the usual approximations regarding the large masses of the
projectile and target-nucleus compared to the electron mass. Then, the
coordinates $\bm{R}_{P}, \bm{R}_{T}$ can be identified with the
internuclear distance $\bm{r}_{N}$. We compare the wavefunctions
\ref{Q:wave-funct-stage} and use the boundary conditions
\ref{Q:bound-cond}. Within the mass' approximation the projectile part
of $\Psi_{1}$ is equal to the one in $\Psi_{2}$. The factors can be
obtained simply from the electron part
%
\begin{equation}\label{Q:b_alpha}
b_{\alpha_{P}}(t=0) = \langle \phi_{\alpha_{P},P} | \phi_{i,T} \rangle
\, .
\end{equation}
%
At later times, the system evolves in the field of the projectile as
%
\begin{equation}\label{Q:Time-evol-fi2}
\Psi_{2}(t) = e^{- i (H_{0}+V_{P}+V_{N}) t} \; \Psi_{2}(0) \, .
\end{equation}
%
The wavefunction $\Psi_{2}$ is an exact solution of the hamiltonian
involved in \ref{Q:Time-evol-fi2} with energy $E = E_{\alpha_{P}}+
k_{N}^{2}/2 \mu_{P}$. At the time $t=\tau$, we have
\begin{eqnarray} \label{Q:Time-evol-fi2a}
\Psi_{2}(\tau) &=& e^{i\left(\bm{k}_{N} \cdot \bm{r}_{N} -
k_{N}^{2}/2 \mu_{P} \tau \right)} \sum_{\alpha_{P}} b_{\alpha_{P}}(0)
\; \phi_{\alpha_{P},P}(\bm{r}_{P}) \; e^{-i E_{\alpha_{P}} \tau}
\nonumber
\\
b_{\alpha_{P},P}(\tau) &=& b_{\alpha_{P},P}(0) \; e^{- i
E_{\alpha_{P}} \tau} \, .
\end{eqnarray}

  \noindent
By using the second condition in \ref{Q:bound-cond} we found

\begin{eqnarray} \label{Q:c-alfaT-exact}
c_{\alpha_{T}}(\tau) &=& \sum_{\alpha_{P}} b_{\alpha_{P}}(\tau) \;
\langle \phi_{\alpha_{T},T} |\phi_{\alpha_{P},P} \rangle \nonumber
\\
&=& \sum_{\alpha_{P}} \langle \phi_{i,T} | \phi_{\alpha_{P},P} \rangle
\; \langle \phi_{\alpha_{P},P} | \phi_{\alpha_{T},T} \rangle \;
e^{-i E_{\alpha_{P}} \tau}
\end{eqnarray}
where we have used \ref{Q:b_alpha} and \ref{Q:Time-evol-fi2a} to get
the last expression.

The coefficient of the last piece $c_{\alpha_{T}}$ only changes the
phase in our approximation. Then we redefine
\[
c'_{\alpha_{T}} = c_{\alpha_{T}} e^{i E_{\alpha_{T}} \tau}
\]
such that its evolution is given simply by
\[
 c'_{\alpha_{T}} = c'_{\alpha_{T}} e^{-i E_{\alpha_{T}} t}
\]
%
By using the projection on wavefunction \ref{Q:fi-f-b}, we get the
transition probability
\[
\mathcal{P}_{if} = \lim_{t \to \infty} c_{\alpha_{T}}(t) =
|c_{\alpha_{T}}(\tau)| = |c'_{\alpha_{T}}(\tau)| \,,
\]
since the total wavefunction has been expanded into (approximated)
eigenfunctions of the final-state channel.

\section{Approximated evaluation of the transition probability}
\label{S:Appro-evalu-trans-proba}

In our first attempt to evaluate the probability of transition we will
use the expression \ref{Q:c-alfaT-exact} for $c_{\alpha}$ and use that
the final state of interest corresponds to low-energy ionized
electrons. We will suppose that only contribute to the summation those
intermediate states where the electron is ionized and its momentum
$\bm{q}$ is small. Then, $\bm{q}_{\alpha_{P}} \approx m_{P} \bm{v} -
\bm{q}$ with $|\bm{q}| \ll |m_{P} \bm{v}|$, implying
%
\begin{equation}\label{Q:Energ-approx-alfaP}
E_{\alpha_{P}} = \frac{q_{P}^{2}}{2 m_{P}} \approx \frac{v^{2} + 2 v
q_{\|}}{2} \, .
\end{equation}
%
This approximation allows us to write
\begin{eqnarray} \label{Q:c-alfaT-appro-1a}
c_{\alpha_{T}}(\tau) &=& \sum_{\alpha_{P}} \langle \phi_{i,T} |
\phi_{\alpha_{P},P} \rangle \; \langle \phi_{\alpha_{P},P} |
\phi_{\alpha_{T},T} \rangle \; e^{-i E_{\alpha_{P}} \tau}
\\ \label{Q:c-alfaT-appro-1b}
& \approx & e^{-i v^{2} \tau/2} \sum_{\alpha_{P}} \langle
\phi_{i,T} | \phi_{\alpha_{P},P} \rangle \; \langle \phi_{\alpha_{P},P}
| \phi_{\alpha_{T},T} \rangle \; e^{-i q_{\|} v \tau}
\\ \label{Q:c-alfaT-appro-1c}
& \approx & e^{-i v^{2} \tau/2} \int d \bm{q} \langle
\phi_{i,T} | \phi_{\bm{q}_{P}} \rangle \; \langle \phi_{\bm{q}_{P}} |
\phi_{\bm{k}_{T}} \rangle \; e^{-i q_{\|} \tau/v}
\end{eqnarray}
%
where in the last equation we have explicitly excluded intermediate
bound states. In our case the $\phi_{\bm{q}_{P}}$ are Continuum Coulomb
states of the ($\mathrm{C}^{6+}$ ion) projectile and
$\phi_{\bm{k}_{T}}$ is the target continuum wavefunction.

Several approximations are possible here.

\subsection{Intermediate plane-wave approximation}
\label{S:Inter-plane-appro}

First, we can consider that the momentum $|\bm{q}_{P}| \gg Z_{P}$ and
use plane waves for the projectile states
\begin{eqnarray} \label{Q:c-alfaT-appro-2}
|c_{\alpha_{T}}| &\approx& \int d \bm{q} \int d \bm{r}_{1} \int
d \bm{r}_{2} \phi_{i,T}(\bm{r}_{1}) e^{i (\bm{v}-\bm{q}) \cdot
(\bm{r}_{1} - \bm{v}\tau)} \; e^{-i (\bm{v}-\bm{q}) \cdot
(\bm{r}_{2} - \bm{v}\tau)} \phi_{\bm{k}_{T}}(\bm{r}_{2}) \;
e^{-i q_{\|}  \cdot \bm{v}\tau} \nonumber \\
&\approx& \int d \bm{r}_{1} \int d \bm{r}_{2}
\phi_{i,T}(\bm{r}_{1}) e^{i \bm{v} \cdot (\bm{r}_{1}-\bm{r}_{2})}
\phi_{\bm{k}_{T}}(\bm{r}_{2}) \;
\delta(\bm{r}_{2}-\bm{r}_{1}-\bm{v}\tau)
\\
&\approx& e^{-i v^{2} \tau} \int d \bm{r}_{1} \,
\phi_{i,T}(\bm{r}_{1}) \phi_{\bm{k}_{T}}(\bm{r}_{1} + \bm{v}\tau)
\end{eqnarray}


\subsection{Contribution from resonant energies}
\label{S:Contr-from-reson-energ}

A further approximation would be to consider only those states that the
two-body target-electron energy at time $\tau$ is the same that at $t
\to \infty$,
\begin{equation}\label{Q:Energ-shift}
q = \sqrt{2 [E_{\infty} + V(r_{T}(\tau))]} \approx \sqrt{k^{2} + 2
V(\langle r_{T} \rangle)}
\end{equation}
%
where we simplify the position to $\langle r_{T}\rangle$ is the
expectation value of the initial state

\section{Limit of infinite time-duration of projectile action}
\label{S:Limit-infin-time--proje-actio}

If the electron evolves in the projectile field during an infinite
amount of time we must take the limit $\tau \to \infty$. By using the
Bochner's integral
%
\begin{equation}\label{Q:Bochn-integ}
\lim_{t \to \infty} f(t) = \lim_{\varepsilon \to 0^{+}}
\frac{\varepsilon^{\gamma}}{\Gamma(\gamma)} \int_{0}^{\infty}
t^{\gamma-1} e^{- \varepsilon t} f(t) d t
\end{equation}
%
and choosing $\gamma = 1$ we obtain
%
\begin{eqnarray}\label{Q:limit-infinit-time}
\lim_{\tau \to \infty} c'_{\alpha_{T}}(\tau)\!\! &=&\!\!
\lim_{\varepsilon \to 0^{+}} \varepsilon \sum_{\alpha_{P}} \langle
\phi_{i,T} | \phi_{\alpha_{P},P} \rangle  \langle \phi_{\alpha_{P},P} |
\phi_{\alpha_{T},T} \rangle \;
\nonumber \\
&& \quad \quad \int_{0}^{\infty} d \tau \; e^{- \varepsilon \tau}
e^{- i (E_{\alpha_{P}} - E_{\alpha_{T}}) \tau}  \nonumber
\\
&=& \sum_{\alpha_{P}} \langle \phi_{i,T} | \phi_{\alpha_{P},P} \rangle
\nonumber \\
&&\quad \quad \left\langle \phi_{\alpha_{P},P} \left|\lim_{\varepsilon
\to 0^{+}} \varepsilon  \int_{0}^{\infty} d \tau \; e^{-
\varepsilon \tau} e^{- i (E_{\alpha_{P}} - h_{T}) \tau} \right|
\phi_{\alpha_{T},T} \right\rangle \nonumber
\\
&=& \sum_{\alpha_{P}} \langle \phi_{i,T} | \phi_{\alpha_{P},P} \rangle
\;  \lim_{\varepsilon \to 0^{+}} \left\langle \phi_{\alpha_{P},P}
\left|i \, \varepsilon G_{T}(E_{\alpha_{P}}-i \varepsilon)
\right| \phi_{\alpha_{T},T} \right\rangle  \nonumber
\\
&=& \sum_{\alpha_{P}} \langle \phi_{i,T} | \phi_{\alpha_{P},P} \rangle
\;  \lim_{\varepsilon \to 0^{+}} \left\langle \phi_{\alpha_{P},P}
\left|1 + G_{T}(E_{\alpha_{P}}-i \varepsilon) V_{T} \right|
\phi_{\alpha_{T},T} \right\rangle   \nonumber
\end{eqnarray}
%
where $G_{T}$ is the Green function of the two-body target-electron
system, with interaction $V_{T}$.
