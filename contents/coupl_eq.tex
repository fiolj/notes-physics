% \documentclass[english,oneside]{book}
%   \RequirePackage{amssymb}[1995/01/01]

% %\DeclareMathAlphabet{\bi}{OML}{cmm}{b}{it}
% %\DeclareMathAlphabet{\bcal}{OMS}{cmsy}{b}{n}
% %% MATH DEFINITIONS
% %%
% %% Define the roman letters to use in math mode (from iopart.cls)
% %\newcommand{e}{\ensuremath{\mathrm{e}}}
% %\newcommand{i}{\ensuremath{\mathrm{i}}}
% %\newcommand{d}{\ensuremath{\mathrm{d}}}
% %\newcommand{\Real}[1]{\ensuremath{\mathrm{Re}\left[ #1 \right]}}
% %\newcommand{\Imag}[1]{\ensuremath{\mathrm{Im}\left[ #1 \right]}}
% %
% \begin{document}

\chapter{Coupled equations}
\section{Wavefunctions}

We assume at $t=-\infty$ the wavefunction is given by
\[
\Psi(\mathbf{r},\mathbf{R}) = \psi_{i,T} (\mathbf{r_{T}})
\]

At an arbitrary posterior time $t$ we expand the wavefunction as:
\begin{eqnarray}
\Psi(\mathbf{r},\mathbf{R},t) &=&
  \int c_{\mathbf{k},\mathbf{K}}(t) \psi_{\mathbf{k},\mathbf{K}}(\mathbf{r},\mathbf{R})\,
  e^{i E t}  d \mathbf{k} d \mathbf{K}  \\
&+& \sum_{j} a_{j}(t)) \Phi_{j,T}(\mathbf{r},\mathbf{R}) e^{i E t}
\nonumber \\
&+& \sum_{j}  b_{j}(t) \Phi_{j,P}(\mathbf{r},\mathbf{R}) e^{i E t}
\nonumber
\end{eqnarray}
%
where the states describe the continuum, target and projectile centered
states. The explicit expressions are:
\begin{eqnarray}\label{Q:CE-fi-T}
  \Phi_{j,T}(\mathbf{r},\mathbf{R}) &=& e^{i \mathbf{K}_{T} \cdot \mathbf{R}_{T}} \,
  \phi_{j,T}(\mathbf{r}_{T}) \,  \mathbb{E}(\mathbf{r}_{P},\mathbf{k}_{P})
  \\
  \Phi_{j,P}(\mathbf{r},\mathbf{R}) &=& e^{i \mathbf{K}_{P} \cdot \mathbf{R}_{P}} \,
  \phi_{j,P}(\mathbf{r}_{P}) \,  \mathbb{E}(\mathbf{r}_{T},\mathbf{k}_{T})
  \\
  \psi_{\mathbf{k},\mathbf{K}}(\mathbf{r},\mathbf{R}) &=& \frac{e^{i
(\mathbf{k}_j\cdot \mathbf{r}_j+\mathbf{K}_j\cdot \mathbf{R}_j)}}{(2
\pi)^3} \,
 D^{\pm}(\nu_T,\mathbf{k}_T,\mathbf{r}_T) \, D^{\pm}(\nu_P,\mathbf{k}_P,\mathbf{r}_P)
 \,D^{\pm}(\nu_N,\mathbf{k}_N,\mathbf{r}_N) \nonumber \\
\end{eqnarray}
Here $phi_{j,T(P)}$ are atomic functions of one center. In the case of
continuum states with Coulomb interactions the distortion factor is
given by \ref{Q:DFactCoul}
%
\[
D^{\pm}(\nu_j,\mathbf{k}_{j},\mathbf{r}_{j})= N^{\pm}(\nu_{j}) \,{_1F_1}\left(
\mp i \nu_{j};1; {\pm} i (k_{j} r_{j} \mp \mathbf{k}_{j}
\cdot\mathbf{r}_{j} ) \right) \, ,
\]
%
$N^{\pm}(\nu_j)= \Gamma(1 {\pm} i\nu_j) e^{-\pi \nu_j/2}$ and Sommerfeld's
parameter is defined by $\nu_j = m_j Z_j/ k_j$.

The energy can be written as $E=\frac{k_{T}^{2}}{2 m_{\alpha}}+
\frac{K_{T}^{2}}{2 m_{\alpha}} $


\section{Evolution of the wavefunction}
The wavefunction must obey the time-dependent Schr\"{o}dinger equation
\begin{equation}\label{Q:CE-tdse}
  \left( H - i\frac{\partial }{\partial t} \right)\Psi = 0\, , \qquad
  \qquad H=H_{0} + V_{T} + V_{P}+ V_{N} \,.
\end{equation}
%
We obtain
\begin{eqnarray}
i \int \dot{c}(\mathbf{k},\mathbf{K})(t)
\psi_{\mathbf{k},\mathbf{K}}(\mathbf{r},\mathbf{R}) d \mathbf{k} d
\mathbf{K}
  =
\int c_{\mathbf{k},\mathbf{K}}(t) \left( H - E \right)
\psi_{\mathbf{k},\mathbf{K}}(\mathbf{r},\mathbf{R})  d \mathbf{k} d \mathbf{K} \\
  \sum_{j}  a_{j}(t) \left( H - E \right) \Phi_{j,T}
  - i \dot{a}_{j}(t) \Phi_{j,T}
  \nonumber \\
+ \sum_{j} b_{j}(t) \left( H - E \right) \Phi_{j,P} -
  i \dot{b}_{j}(t) \Phi_{j,P}
 \nonumber
\end{eqnarray}

Then, we project on each base state. Using the continuum wavefunction
$\psi_{\mathbf{k}',\mathbf{K}'}$ we get
\[
\int  d \mathbf{k} d \mathbf{K}  \,  c_{\mathbf{k},\mathbf{K}}(t) \,
W^{c,c}_{\mathbf{k}',\mathbf{k}}  + \sum_{j} a_{j}
W^{c,T}_{\mathbf{k}',j} - i \dot{a}_{j} S^{c,T}_{\mathbf{k}',j} +
\sum_{l} b_{l} W^{c,P}_{\mathbf{k}',l} - i \dot{b}_{l}
S^{c,P}_{\mathbf{k}',l} = i \dot{c}_{\mathbf{k}',\mathbf{K}'}(t)
\]

Identically, when projecting on bound states,
\begin{eqnarray*}
\int  d \mathbf{k} d \mathbf{K}  \Big( c_{\mathbf{k},\mathbf{K}}(t) \,
W^{T,c}_{m,\mathbf{k}} - i \dot{c}_{\mathbf{k},\mathbf{K}}(t)
\,S^{T,c}_{m,\mathbf{k}} \Big) + \sum_{j} a_{j} W^{T,T}_{m,j} +
\sum_{l} b_{l} W^{T,P}_{m,l} - i \dot{b}_{l} S^{T,P}_{m,l} = i
\dot{a}_{m}\nonumber
  \\ %\\
\int  d \mathbf{k} d \mathbf{K}  \Big( c_{\mathbf{k},\mathbf{K}}(t) \,
W^{P,c}_{m,\mathbf{k}} - i \dot{c}_{\mathbf{k},\mathbf{K}}(t)
\,S^{P,c}_{m,\mathbf{k}} \Big) + \sum_{j} a_{j} W^{P,T}_{m,j} - i
\dot{a}_{j} S^{P,T}_{m,j} + \sum_{l} b_{l} W^{P,P}_{m,l} = i
\dot{b}_{m} \nonumber
\end{eqnarray*}
or, in matrix form:
%\begin{equation}\label{Q:CE-diff-equat}
%  \begin{array}{cccccc}
%     &  &  &  &  &  \\
%     &  &  &  &  &  \\
%     &  &  &  &  &  \\
%     &  &  &  &  &  \\
%     &  &  &  &  &  \\
%     &  &  &  &  & 
%  \end{array}
%\end{equation}


\noindent Where the coefficients are matrix elements given by
\begin{eqnarray}
W^{c,c} &=& \langle
\psi_{\mathbf{k}',\mathbf{K}'}|H-E|\psi_{\mathbf{k},\mathbf{K}} \rangle
 \\
W^{T,T} &=& \langle \Phi_{P,j'}|H-E|\Phi_{P,j'} \rangle
 \\
W^{P,P} &=& \langle \Phi_{P,j'}|H-E|\Phi_{P,j'} \rangle
 \\
W^{T,P} &=& \langle \Phi_{T,j'}|H-E|\Phi_{P,j'} \rangle
 \\
W^{P,c} &=& \langle \Phi_{P,j'}|H-E|\psi_{\mathbf{k},\mathbf{K}}
\rangle
 \\
W^{T,c} &=& \langle \Phi_{T,j}|H-E|\psi_{\mathbf{k},\mathbf{K}} \rangle
 \end{eqnarray}
%
and
%
\begin{eqnarray}
S^{T,P} &=& \langle \Phi_{T,j'}|\Phi_{P,j'} \rangle
 \\
S^{P,c} &=& \langle \Phi_{P,j'}|\psi_{\mathbf{k},\mathbf{K}} \rangle
 \\
S^{T,c} &=& \langle \Phi_{T,j}|\psi_{\mathbf{k},\mathbf{K}} \rangle
\end{eqnarray}

\subsection{Evaluation of coefficients}
\[
\mathbb{K}_{j} = \frac{\nabla_{\mathbf{r}_{j}}
D(\nu_{j};\mathbf{k}_{j},\mathbf{r}_{j})}{D(\nu_{j};\mathbf{k}_{j},\mathbf{r}_{j})}
\]
% \end{document}
