\chapter{Definition of functions and Integrals}
\label{C:Defin-funct-Integ}
\section[Nordsieck's integrals]
{Nordsieck's integrals \cite{Nordsie1954PRp785}}

We will denote the different Nordsieck-like integrals as:
%
\begin{eqnarray} \label{Q:Nm}
N_{m} \left(Z, \vect{Q} ; a_{1}, \vect{k}_{1} ; a_{2}, \vect{k}_{2} \right)
\!\!&=&\!\! \int d \vect{r} \frac{e^{i \vect{Q} \cdot \vect{r} - Z
r}}{r^{m}} \, {_{1}F_{1}} \left[ i a_{2} ; 1 ;i \left( k_{2} r +
\vect{k}_{2}\cdot \vect{r} \right)\right] \; {_{1}F_{1}} \left[ i a_{1}
; 1 ;i \left(k_{1} r + \vect{k}_{1}\cdot \vect{r} \right) \right]
 \nonumber \\ \\
\vect{I}_{m} \left(Z, \vect{Q}; a_{1}, \vect{k}_{1} ; a_{2}, \vect{k}_{2}
\right) &=&\!\! \int d \vect{r} \frac{e^{i \vect{Q} \cdot \vect{r} -
Z r}}{r^{m}} {_{1}F_{1}} \left[ i a_{2} ; 1 ;i \left(k_{2} r +
\vect{k}_{2}\cdot \vect{r} \right)\right] \; \nabla_{\vect{r}} \,{_{1}F_{1}}
\left[ i a_{1} ; 1 ;i \left(k_{1} r + \vect{k}_{1}\cdot \vect{r}
\right)\right]  \nonumber
\\   \label{Q:Im}
\\ \label{Q:NEm}
N^{E}_{m} \left(Z, \vect{Q} ; a_{1}, \vect{k}_{1} ; a_{2}, \vect{k}_{2}
\right) &=& \int d \vect{r} \frac{e^{i \vect{Q} \cdot \vect{r} - Z
r}}{r^{m}} \, e^{i a_{2} \log{ \left( k_{2} r + \vect{k}_{2}\cdot
\vect{r} \right)}} \; {_{1}F_{1}} \left[ i a_{1} ; 1 ;i \left(k_{1}
r + \vect{k}_{1}\cdot \vect{r} \right) \right]
\\  \label{Q:NEEm}
N^{EE}_{m} \left(Z, \vect{Q} ; a_{1}, \vect{k}_{1} ; a_{2}, \vect{k}_{2}
\right) &=& \int d \vect{r} \frac{e^{i \vect{Q} \cdot \vect{r} - Z
r}}{r^{m}} \, e^{i a_{2} \log{ \left( k_{2} r + \vect{k}_{2}\cdot
\vect{r} \right)}} \; e^{ i a_{1} \log{\left(k_{1} r +
\vect{k}_{1}\cdot \vect{r} \right)}}
\\    \label{Q:IEm}
\vect{I}^{E}_{m} \left(Z, \vect{Q}; a_{1}, \vect{k}_{1} ; a_{2}, \vect{k}_{2}
\right) &=& \int d \vect{r} \frac{e^{i \vect{Q} \cdot \vect{r} - Z
r}}{r^{m}}\,e^{i a_{2} \log{ \left( k_{2} r + \vect{k}_{2}\cdot
\vect{r} \right)}} \; \nabla_{\vect{r}}\, {_{1}F_{1}} \left[ i a_{1} ; 1
;i \left(k_{1} r + \vect{k}_{1}\cdot \vect{r} \right)\right]
\end{eqnarray}

We will state the result for several of these integrals.

\subsection{Evaluation of the $N_{1}$ Nordsieck integral}

All the quantities are related to the basic integral
%
\begin{eqnarray} \label{Q:N1}
N_{1} \left(Z,  \vect{Q} ; a_{1}, \vect{k}_{1} ; a_{2}, \vect{k}_{2} \right)
  &=& \int d \vect{r} \frac{e^{i \vect{Q} \cdot \vect{r} - Z
r}}{r} {_{1}F_{1}} \left[ i a_{2} ; 1 ;i \left(k_{2} r +
\vect{k}_{2}\cdot \vect{r} \right)\right] \; {_{1}F_{1}} \left[ i a_{1}
; 1 ;i \left(k_{1} r + \vect{k}_{1}\cdot \vect{r} \right)\right]
  \nonumber \\
 &=& \frac{4 \pi}{D} \; \left( 1 + u_{1} \right)^{-i
a_{1}} \, \left( 1 + u_{2} \right)^{-i a_{2}} \, {_{2}F_{1}}\left(
i a_{1} , i a_{2} ; 1 ; X \right)
\end{eqnarray}
with
\begin{eqnarray} \label{Q:XuD}
X &=& \frac{u_{1} \, u_{2} +u_{3}}{(1 + u_{1}) (1 + u_{2})} \nonumber
\\
u_{1} &=& 2 \left( \vect{k}_{1}\cdot \vect{Q} - i Z k_{1}
\right)/D \nonumber \\
u_{2} &=& 2 \left( \vect{k}_{2}\cdot \vect{Q} - i Z k_{2}
\right)/D \\
u_{3} &=& 2 \left( k_{1} k_{2} - \vect{k}_{1}\cdot \vect{k}_{2} \right)/D
 \nonumber \\
D &=& Q^{2} + Z^{2} \nonumber
\end{eqnarray}
%

\subsection{Evaluation of the $N_{0}$ Nordsieck-like integral}

For evaluate this integral we use the relation
%
\[
N_{0}(Z,\vect{q}; a_{1},\vect{k}_{1}; a_{2},\vect{k}_{2}) = - \frac{d
}{d Z}\left( N_{1}(Z,\vect{q}; a_{1},\vect{k}_{1}; a_{2},\vect{k}_{2})
\vstretch \right)
\]
%
and obtain
%
\begin{eqnarray*}
N_{0}(Z,\vect{q}; a_{1},\vect{k}_{1}; a_{2},\vect{k}_{2}) &=& \left[\frac{1}{D} \frac{d D}{d Z} + \frac{i a_{1}}{1 + u_{1}}
\frac{d u_{1}}{d Z} + \frac{i a_{2}}{1 + u_{2}}\frac{d
u_{2}}{d Z} - \Big({{_{2}F'_{1}}(i a_{1}, i a_{2}; 1;X) }/{{_{2}F_{1}}(i a_{1}, i a_{2}; 1; X)}\Big) \frac{d X}{d Z}
\right]
\\
&&  {\times} N_{1}(Z,\vect{q};a_{1},\vect{k}_{1}; a_{2},\vect{k}_{2})
\end{eqnarray*}
%
where $u_{j},D,X$ are defined in \ref{Q:XuD} and the derivatives
respect to $Z$ are given by
\begin{eqnarray*}
D' &=& 2 Z \nonumber \\
u'_{j}&=& - 2 \left(Z u_{j} + \,i k_{j} \right)/D \qquad \qquad
(j=~1,2)
\nonumber \\
u'_{3} &=& -2 Z u_{3}/D
\nonumber \\
X' &=& \frac{ u_{1}' u_{2} + u_{1} u_{2}' + u'_{3} - X \left[ u_{1}'
(1+u_{2}) + (1+u_{1}) u_{2}' \right]}{(1 + u_{1})(1+u_{2})}
\nonumber \\
\\
{_{2}F'_{1}}(a_{1}, a_{2}; a_{3}; x) &=& \frac{a_{1} a_{2}}{a_{3}} \;
{_{2}F_{1}}(1 + i a_{1}, 1 + i a_{2};1+a_{3};x) \nonumber
\end{eqnarray*}

\subsection{Expressions involving derivatives}
The $\vect{I}_{m} \left(Z, \vect{Q}; a_{1}, \vect{k}_{1} ; a_{2}, \vect{k}_{2}
\right) $ given by \ref{Q:Im} is easily obtained. By using
%
\[
\nabla_{\vect{r}_{1}} \Big[ \,{_{1}F_{1}} \left[ i a_{1} ; 1 ;i
\left(k_{1} r + \vect{k}_{1}\cdot \vect{r} \right)\right] \Big] =
\frac{k_{1}}{r} \nabla_{\vect{k}_{1}}\Bigl[\,_{1}F_{1} \left[ i
a_{1} ; 1 ;i \left(k_{1} r + \vect{k}_{1}\cdot \vect{r} \right)\right]
\Bigr] \, .
\]
%
For instance, we obtain for $\vect{I}_{0}$
%
\begin{align*}
\vect{I}_{0} \left(Z, \vect{Q}; a_{1}, \vect{k}_{1} ; a_{2}, \vect{k}_{2}
\right) &= k_{1} \nabla_{\vect{k}_{1}} \left[ N_{1}(Z,\vect{q};
a_{1},\vect{k}_{1}; a_{2},\vect{k}_{2})  \vstretch \right]
\\
&= k_{1}  \left[ \frac{-i a_{1}}{1 + u_{1}} \nabla_{\vect{k}_{1}}
u_{1}  + \frac{\,{_{2}F'_{1}}(i a_{1}, i a_{2}; 1;
X)}{\,{_{2}F_{1}}(i a_{1}, i a_{2}; 1; X)} \nabla_{\vect{k}_{1}} X
\right] \nonumber
\\
& {\times} \; N_{1}(Z,\vect{Q}; a_{1},\vect{k}_{1}; a_{2},\vect{k}_{2})
\end{align*}
%
where
\begin{align*}
\nabla_{\vect{k}_{1}} X &= \frac{\partial X}{\partial u_{1}}
\nabla_{\vect{k}_{1}} u_{1} + \frac{\partial X}{\partial u_{3}}
\nabla_{\vect{k}_{1}} u_{3}
 \\
\\
\frac{\partial X}{\partial u_{1}}  &= \frac{u_{2}}{ ( 1 + u_{1}) (1 +
u_{2})} - \frac{u_{1} u_{2} + u_{3}}{(1 + u_{1})^{2} (1 + u_{2})}
\nonumber \\
\frac{\partial X}{\partial u_{3}}  &= \frac{1}{ ( 1 + u_{1})
(1 + u_{2})} \nonumber \\
\nabla_{\vect{k}_{1}} u_{1} &=\frac{2\left( \vect{Q} - i Z \vect{k}_{1}
/k_{1} \right)}{D}
\nonumber \\
\nabla_{\vect{k}_{1}} u_{3} &= \frac{2 \left( k_{2} \vect{k}_{1}/k_{1} -
\vect{k}_{2} \right)}{D} \nonumber
\end{align*}

\subsection{Eikonal versions}


By recalling the definition of the $E^{\pm}$ in terms of the asymptotic
form of the hypergeometric functions (equation \ref{Q:Eikonal-wf}) we
can write the integrals \ref{Q:NEm}-\ref{Q:IEm} from the corresponding
\ref{Q:Nm} and \ref{Q:Im} by keeping the asymptotic behavior of the
latter expressions. We obtain:
%
\begin{eqnarray*}
N_{m} \left(Z, \vect{Q} ; a_{1}, \vect{k}_{1} ; a_{2}, \lambda \vect{k}_{2}
\right) &\bajo{\approx}{\lambda \to \infty}& N^{-1}(a_{2}) \; N^{E}_{m}
\left(Z, \vect{Q}; a_{1}, \vect{k}_{1} ; a_{2}, \lambda \vect{k}_{2} \right)
\\
\vect{I}_{m} \left(Z, \vect{Q} ; a_{1}, \vect{k}_{1} ; a_{2}, \lambda
\vect{k}_{2} \right) &\bajo{\approx}{\lambda \to \infty}& N^{-1}(a_{2})
\; \vect{I}^{E}_{m} \left(Z, \vect{Q} ; a_{1}, \vect{k}_{1} ; a_{2}, \lambda
\vect{k}_{2} \right)
\end{eqnarray*}
where $N(a_{2})$ is the normalization factor defined in \ref{Q:NNorm}

\subsubsection{Eikonal N-like integrals}

For the $N$-like integrals we obtain:
%
\begin{eqnarray}\label{Q:N-integ}
N^{E}_{1} \left(Z, \vect{Q} ; a_{1}, \vect{k}_{1} ; a_{2}, \vect{k}_{2}
\right) &=& N(a_{2})\;\frac{4 \pi}{D} \; \left( 1 + u_{1}
\right)^{-i a_{1}} \, u_{2}^{-i a_{2}} \, {_{2}F_{1}}\left( i
a_{1} , i a_{2} ; 1 ; X^{E} \right)
\\
N^{E}_{0} \left(Z, \vect{Q} ; a_{1}, \vect{k}_{1} ; a_{2}, \vect{k}_{2}
\right) &=& \left[ \frac{1}{D} \frac{d D}{d Z} + \frac{i
a_{1}}{1 + u_{1}} \frac{d u_{1}}{d Z} + \frac{i a_{2}}{u_{2}}
\frac{d u_{2}}{d Z} - \frac{\,{_{2}F'_{1}}(i a_{1}, i a_{2};
1; X)}{\,{_{2}F_{1}}(i a_{1}, i a_{2}; 1; X^{E})} \frac{d
X^{E}}{d Z} \right] N^{E}_{1} \nonumber
\\
\end{eqnarray}
%
where
\begin{eqnarray*}
X^{E} &=& \frac{u_{1} u_{2} + u_{3}}{(1+u_{1}) \, u_{2}}  \nonumber \\
{X^{E}}' &=& \frac{ u_{1}' u_{2} + u_{1} u_{2}' + u'_{3} - X^{E} \left[
u_{1}' u_{2} + (1+u_{1}) u_{2}' \right]}{(1 + u_{1})u_{2}}
\end{eqnarray*}
%
and $D',u_{j},u'_{j}$ are defined in \ref{Q:XuD}


For the eikonal-eikonal integrals we get:
\begin{eqnarray}\label{Q:NEE}
N^{EE}_{1} \left(Z, \vect{Q} ; a_{1}, \vect{k}_{1} ; a_{2}, \vect{k}_{2}
\right) &=&  N(a_{1}) N(a_{2})\;\frac{4 \pi}{D} \; u_{1}^{-i a_{1}}
\, u_{2}^{-i a_{2}} \, {_{2}F_{1}}\left( i a_{1} , i a_{2} ; 1
; X^{EE} \right)
\\
N^{EE}_{0} \left(Z, \vect{Q} ; a_{1}, \vect{k}_{1} ; a_{2}, \vect{k}_{2}
\right) &=& \left[ \frac{1}{D} \frac{d D}{d Z} + \frac{i
a_{1}}{u_{1}} \frac{d u_{1}}{d Z} + \frac{i a_{2}}{u_{2}}
\frac{d u_{2}}{d Z} - \frac{\,{_{2}F'_{1}}(i a_{1}, i a_{2};
1; X^{EE})}{\,{_{2}F_{1}}(i a_{1}, i a_{2}; 1; X^{EE})} \frac{d
X^{EE}}{d Z} \right] N^{EE}_{1}
\nonumber \\
\end{eqnarray}
%
where
\begin{eqnarray*}
X^{EE} &=& \frac{u_{1} u_{2} + u_{3}}{u_{1} \, u_{2}}  \nonumber \\
{X^{EE}}' &=& \frac{ u_{1}' u_{2} + u_{1} u_{2}' + u'_{3} - X^{EE}
\left[ u_{1}' u_{2} + u_{1} u_{2}' \right]}{u_{1} u_{2}}
\end{eqnarray*}
%
\subsubsection{Eikonal $\bi{I}$-like integrals}

The $\vect{I}$-like integrals are given by

\begin{eqnarray}\label{Q:I_Eik}
\vect{I}^{E}_{0} \left(Z, \vect{Q}; a_{1}, \vect{k}_{1} ; a_{2}, \vect{k}_{2}
\right) &=& k_{1}  \left[ \frac{- i a_{1}}{1 + u_{1}}
\nabla_{\vect{k}_{1}} u_{1}  + \frac{\,{_{2}F'_{1}}(i a_{1}, i a_{2};
1; X^{E})}{\,{_{2}F_{1}}(i a_{1}, i a_{2}; 1; X^{E})}
\nabla_{\vect{k}_{1}} X^{E} \right] \nonumber
\\
&& {\times} \; N^{E}_{1}(Z,\vect{Q}; a_{1},\vect{k}_{1}; a_{2},\vect{k}_{2})
\end{eqnarray}
%
with
\begin{eqnarray*}
\nabla_{\vect{k}_{1}} X^{E} &=& \frac{\partial X^{E}}{\partial u_{1}}
\nabla_{\vect{k}_{1}} u_{1} + \frac{\partial X^{E}}{\partial u_{3}}
\nabla_{\vect{k}_{1}} u_{3}
\\
\frac{\partial X^{E}}{\partial u_{1}}  &=& \frac{1-X^{E}}{( 1 + u_{1})}
\nonumber \\
\frac{\partial X^{E}}{\partial u_{3}}  &=& \frac{1}{ ( 1 + u_{1})
u_{2}} \nonumber
\end{eqnarray*}


\subsection{Simplified integrals}
From the $N_{m}$ and $\vect{I}_{m}$ expressions it is possible to derive
simplified expressions by taking the limit $a_{2} \to 0$
%
\begin{eqnarray}\label{Q:Jm}%
 J_{m}( Z, \vect{p}, a_{1}, \vect{k}_{1} ) &=&
\int d \vect{r} \; e^{i \vect{p} \cdot \vect{r}} \; \frac{e^{- Z
\, r}}{r^{m}} \, {_{1}F_{1}} \left( i a_{1} ; 1 ;i \left(k r +
\vect{k}_{1}\cdot \vect{r} \right)\right) \\
\vect{G}_{m}( Z, \vect{p}, a_{1}, \vect{k}_{1} ) &=& \int d \vect{r} \;
e^{i \vect{p} \cdot \vect{r}} \; \frac{e^{- Z \, r}}{r^{m}} \,
\nabla {_{1}F_{1}} \left( i a_{1} ; 1 ;i \left(k r +
\vect{k}_{1}\cdot \vect{r} \right)\right)
\end{eqnarray}

  \noindent
We obtain
\begin{eqnarray}\label{Q:Jint}
J_{1}( Z, \vect{Q}, a_{1}, \vect{k}_{1} ) &=& \frac{4 \pi}{D} (1+a_{1})^{-
i a_{1}}
\\
J_{0}( Z, \vect{Q}, a_{1}, \vect{k}_{1} ) &=& \left( \frac{1}{D} \frac{d
D}{d Z} + \frac{i a_{1}}{1 + u_{1}} \frac{d u_{1}}{d Z}
\right) J_{1}( Z, \vect{Q}, a_{1}, \vect{k}_{1} ) \\
\vect{G}_{0}( Z, \vect{Q}, a, \vect{k} ) &=&  i  \frac{ a_{1}\, k_{1}}{1 +
u_{1}} \left[ \nabla_{\vect{k}_{1}} u_{1} \right] \; J_{1}(Z,\vect{Q};
a_{1},\vect{k}_{1})
\end{eqnarray}

\begin{equation}\label{Q:G_if}
\vect{G}_{0}( Z, \vect{Q}, a, \vect{k} ) + i\, \vect{Q} J_{0}( Z, \vect{Q}, a,
\vect{k} ) = Z \int \,\hat{r} \,e^{i \vect{Q} \cdot \vect{r}} \;
e^{- Z \, r} \, {_{1}F_{1}} \left[\, i a_{1} ; 1 ;i \left(k r
+ \vect{k}_{1}\cdot \vect{r} \right) \vstretch \right]\, d \vect{r}
\end{equation}
%

\subsection{Relations between quantities (for $Z \to 0$)}
%
\begin{eqnarray} \label{Q-N1-Ja}
\vect{Q} \, J_{0}(Z , \vect{Q}, a , \vect{k}) &=& i \, \vect{G}_{o} (Z ,
\vect{Q}, a , \vect{k}) \qquad \qquad
 \\
N_{1} \left(Z,  \vect{Q} ; a_{1}, \vect{k}_{1} ; a_{2}, \vect{k}_{2} \right)
&=&  \frac{1}{(2 \pi)^{3}} \int J_{1}(Z_{1}, \vect{Q}
-\vect{q},a_{1},\vect{k}_{1}) \, J_{0}(Z_{2},  \vect{q} ,a_{2},\vect{k}_{2} )
\, d \vect{q}
 \nonumber \\ \\ \label{Q-N1-Jb}
&=& \frac{1}{(2 \pi)^{3}} \int J_{1}(Z_{1},\vect{q} - \vect{Q}, a_{1},-
\vect{k}_{1}) \, J_{0}(Z_{2}, \vect{q} ,a_{2},\vect{k}_{2} ) \, d \vect{q}
 \nonumber \\ \\ \label{Q-N1-Jc}
&=& \frac{1}{(2 \pi)^{3}} \int J_{1}( Z_{1}, \vect{q} + \vect{Q}, a_{1},
\vect{k}_{1} ) \, J_{0}(Z_{2}, \vect{q}  ,a_{2},-\vect{k}_{2} ) \, d
\vect{q}
\nonumber \\
\end{eqnarray}

\begin{eqnarray} \label{Q-I1-Ja}
\vect{I}_{1}(Z,\vect{Q}; a_{1},\vect{k}_{1}; a_{2},\vect{k}_{2} ) &=&\frac{-
i }{(2 \pi)^{3}}\,\int \vect{q} \, J_{0}(Z_{2}, \vect{q},
a_{2},\vect{k}_{2}) \,
J_{1}(Z_{1}, \vect{Q}- \vect{q}, a_{1}, \vect{k}_{1} ) \; d \vect{q}
\nonumber \\ \\
\label{Q-I1-Jb} \mbox{(verificar!)} &=^{*}&  \frac{i }{(2
\pi)^{3}}\,\int \vect{q} \, J_{0}(Z_{2}, \vect{q}, a_{2}, - \vect{k}_{2}) \,
J_{1}(Z_{1}, \vect{Q} + \vect{q}, a_{1}, \vect{k}_{1} ) \; d \vect{q}
\nonumber \\  \\  \label{Q-I1-Jc} &=&\frac{- i }{(2 \pi)^{3}}\,\int
\vect{q} \, J_{0}(Z_{2}, \vect{q}, a_{2},\vect{k}_{2}) \, J_{1}(Z_{1},
\vect{q}- \vect{Q}, a_{1}, - \vect{k}_{1} ) \;
d \vect{q}    \nonumber \\
\end{eqnarray}
with $Z = Z_{1} + Z_{2}$.

%%% Local Variables: 
%%% mode: latex
%%% TeX-master: "mainxs"
%%% End: 
