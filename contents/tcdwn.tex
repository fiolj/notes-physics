\section{Semiclassical inclusion of internuclear interaction}

In the usual mass' approximation the internuclear interaction can be included
by means of a phase factor in a semiclassical approximation.  For Coulomb
interactions this factor is given by $e^{i \nu_{N}\ln(\rho)}$ and the
transition matrix can be written as
%
\begin{eqnarray}\label{Q:Npot-semic}
t_{if}(\bm{k},\bm{Q}) &=& \frac{1}{(2 \pi)^{2}} \int d \brho \; e^{i
\, \bm{Q} \cdot \brho } \rho^{2 \, i \, \nu} \int d \bm{Q}_{\perp}' \;
e^{- i \, \bm{Q}' \cdot \brho } \;
t^{\circ}_{if}(\bm{k},\bm{Q}) \nonumber \\
  &=&
\frac{1}{(2 \pi)^{2}} \int d \bm{Q}_{\perp}' \; t^{\circ}_{if}(\bm{k},\bm{Q})
\; \int_{0}^{\infty}  \rho^{ 1 + 2 \, i \, \nu_{N}} \; d \rho
\int_{0}^{2 \pi} e^{i \, (\bm{Q}- \bm{Q}') \cdot \brho} \; d
\varphi_{\rho}.
\end{eqnarray}
%
where $t^{\circ}_{if}$ is evaluated neglecting the internuclear interaction. The
integral on the impact parameter can be carried out analytically
\begin{eqnarray}\label{Q:Npot-semic1}
I_{1}=\int_{0}^{\infty} d \rho \; \rho^{ 1 + 2 \, i \, \nu}
\int_{0}^{2 \pi} e^{i \, \eta\, \rho \cos{\varphi_{\rho}}} \; d
\varphi_{\rho} &=& 2 \pi \int_{0}^{\infty} d \rho \; \rho^{1 + 2 \, i
\, \nu} J_{0}(\rho \,\eta)
  \\ &=&
\frac{{4^{1 + i\,\nu_{N}}}\,\pi \, \eta^{-2 - 2\, i\,\nu_{N}}\,
\Gamma(1 + i\,\nu_{N})} {\Gamma\left(- i\,\nu_{N} \right)} \nonumber
\end{eqnarray}
where $J_{0}$ and $\Gamma$ are the order zero-Bessel and gamma functions,
respectively. We have defined the two-dimensional vector
$\bm{\eta}=\bm{Q}_{\perp}-\bm{Q}_{\perp}'$. The first integration is valid
provided than $\textrm{Im}\left(\eta \, \rho \right)=0$. The second equality
requires that $\textrm{Im}\left(\sqrt{\eta^{2}} \right)=0$, $\eta^{2} > 0$
and $\textrm{Re}\left( i \nu_{N} \right) > -1$.

The transition matrix can finally be written as
\begin{eqnarray}\label{Q:tifNsemic}
t_{if}(\bm{k},\bm{Q}) &=& \frac{2^{1 + 2\,i\,\nu_{N}}}{2 \pi}
\frac{\Gamma(1 + i\,\nu_{N})} {\Gamma\left(- i\,\nu_{N} \right)} \int
d \bm{Q}'_{\perp} \; \frac{\; t^{\circ}_{if}(\bm{k},\bm{Q}')} {\mid
\bm{Q}_{\perp}-\bm{Q}'_{\perp}\mid ^{2 + 2\, i\,\nu_{N}}}\, \nonumber
\\
&=& \frac{2^{1 + 2\,i\,\nu_{N}}}{2 \pi} \frac{\Gamma(1 + i\,\nu_{N})}
{\Gamma\left(- i\,\nu_{N} \right)} \int d \bm{\eta} \; \frac{\;
t^{\circ}_{if}(\bm{k},(\bm{\eta}+\bm{Q},Q_{\parallel}))} {\eta^{2 + 2\,
i\,\nu_{N}}}\,
\end{eqnarray}

This expression presents the difficulty that in the case of null internuclear
interaction does not converge to the identity relation $t_{if}=t^{\circ}_{if}$.

\subsection{Regularization of the integrals}

In order to define the Fourier transforms we use a regularization based on an
exponential decreasing of the integrand. Equation \ref{Q:Npot-semic1} can be
evaluated as the limit $\alpha \to 0$ of the integral
\begin{eqnarray}\label{Q:Npot-sreg}
I_{1} &=&\int_{0}^{\infty} d \rho \; \rho^{ 1 + 2 \, i \, \nu} \,
e^{- \alpha \rho} \int_{0}^{2 \pi} e^{i \, \eta\, \rho
\cos{\varphi_{\rho}}} \; d \varphi_{\rho} \nonumber
  \\
&=& 2 \pi \int_{0}^{\infty} d \rho \; \rho^{1 + 2 \, i \, \nu}\,
e^{- \alpha \rho} \, J_{0}(\rho \,\eta) \nonumber
  \\
&=& 2 \pi \, \frac{\Gamma \left( 2 + 2 i \nu_{N} \right)}{ \alpha^{2 + 2
i \nu_{N}}} \; _{2}F_{1}\left[ 1 + i \nu_{N}, \frac{3}{2} + i
\nu_{N} ; 1; -  \frac{\eta^{2}}{\alpha^{2}} \right] \nonumber
  \\
&=& 2 \pi \, \frac{\Gamma \left( 2 + 2 i \nu_{N} \right)}{ \alpha^{2 + 2
i \nu_{N}}} \; \left[ 1 + \left( \frac{\eta}{\alpha} \right)^{2}
\right]^{-(1+i \nu)} P^{0}_{1 + 2 i \nu} \left[
\frac{\alpha}{\sqrt{\alpha^{2}+ \eta^{2}}} \right] \nonumber
  \\
&=&  \frac{2 \pi \,\Gamma \left( 2 + 2 i \nu_{N} \right)}{
\left({\alpha^{2} + \eta^{2}} \right)^{1+i \nu}} \;  P^{0}_{1 + 2 i
\nu} \left[ \frac{\alpha}{\sqrt{\alpha^{2}+ \eta^{2}}} \right] \nonumber
\end{eqnarray}
%
where $_{2}F_{1}$ and $P^{\mu}_{\lambda}(x)$ are the hypergeometric and
Legendre functions, respectively.

The transition matrix is evaluated as
\begin{eqnarray}\label{Q:tifNsreg}
t_{if} &=&   \frac{\Gamma \left( 2 + 2 i \nu_{N} \right)}{2 \pi} \; \int
d \bm{\eta} \; P^{0}_{1 + 2 i \nu} \left(
\frac{\alpha}{\sqrt{\alpha^{2}+ \eta^{2}}}\right) \;
\frac{t^{\circ}_{if}(\bm{k},(\bm{\eta}+\bm{Q}_{\perp},Q_{\parallel}))} {
\left( {\alpha^{2} + \eta^{2}} \right)^{1+i \nu}} \nonumber \\ \\
&=&  \frac{\Gamma \left( 2 + 2 i \nu_{N} \right)}{2 \pi} \;
\int_{0}^{\infty} P^{0}_{1 + 2 i \nu} \left(
\frac{\alpha}{\sqrt{\alpha^{2}+ \eta^{2}}} \right) \; \frac{\eta \, d
\eta}{\left( {\alpha^{2} + \eta^{2}} \right)^{1+i \nu}} \;\int_{0}^{2 \pi}
d \varphi_{\eta} \; t^{\circ}_{if}(\bm{k},(\bm{\eta}+\bm{Q}_{\perp},Q_{\mid
}))\nonumber
\\
&=&  \frac{\Gamma \left( 2 + 2 i \nu_{N} \right)}{4 \pi} \;
\int_{0}^{\infty} P^{0}_{1 + 2 i \nu} \left(
\frac{\alpha}{\sqrt{\alpha^{2}+ \eta^{2}}} \right) \; \frac{d \eta^{2}}{
\left( {\alpha^{2} + \eta^{2}} \right)^{1+i \nu}} \;\int_{0}^{2 \pi} d
\varphi_{\eta} \; t^{\circ}_{if}(\bm{k},(\bm{\eta}+\bm{Q}_{\perp},Q_{\mid
}))\nonumber
\end{eqnarray}
%
where the limit $\alpha \downarrow 0$ must be understood. If we suppose that
the region $\eta \gg \alpha $ most contributes to the above integral, we can
use the expansion of the Legendre function around $x=0$
%
\[
P_{1 + 2 i \nu} (x) \bajo{\approx}{x \to 0}
\frac{\sqrt{\pi}}{\Gamma\left(\frac{3}{2} + i \nu \right) \,
\Gamma\left(-i \nu\right)} + \frac{(1 + 2 i \nu) (1 + i \nu)
\sqrt{\pi}}{\Gamma\left(\frac{1}{2} - i \nu \right) \, \Gamma\left(2+ i
\nu \right)}\, x + \mathcal{O}(x^{2}) \, .
\]


Equivalently to \ref{Q:tifNsreg} we can write the transition matrix in terms
of the hypergeometric function
%
\begin{eqnarray}
t_{if} & =&\frac{\Gamma \left( 2 + 2 i \nu_{N} \right)}{2 \pi \, \alpha^{2
 + 2 i \nu_{N}}} \int_{0}^{\infty} {_{2}F_{1}}\left[ 1 + i \nu_{N},
\frac{3}{2} + i \nu_{N} ; 1; -  \frac{\eta^{2}}{\alpha^{2}} \right] \eta
\, d \eta \nonumber \\ &&\int_{0}^{2 \pi} d \varphi_{\eta} \;
t^{\circ}_{if}(\bm{k},(\bm{\eta}+\bm{Q}_{\perp},Q_{\mid  \mid }))
\end{eqnarray}

\subsection*{Limit without internuclear interaction $\nu_{N}=0$}

In the case that the internuclear interaction is negligible ($\nu_{N}=0$) the
Legendre function is $P_{0}(x)=x$ and the expression \ref{Q:tifNsreg} can be
rewritten as

\begin{eqnarray*}
t_{if} &=& \frac{1}{2 \pi} \int_{0}^{\infty} \frac{\alpha \eta \, d
\eta}{\sqrt[3]{\alpha^{2}+ \eta^{2}}}  \;\int_{0}^{2 \pi} d \varphi_{\eta}
\;
t^{\circ}_{if}(\bm{k},(\bm{\eta}+\bm{Q}_{\perp},Q_{\mid  \mid })) \\
(u = \alpha^{2}+ \eta^{2}) \quad &=&  \frac{\alpha}{4 \pi} \int_{0}^{2 \pi}
d \varphi_{\eta} \; \int_{\alpha^{2}}^{\infty} \, \frac{d u
}{\sqrt[3]{u}}
\;\, t^{\circ}_{if}(\bm{k},(\bm{\eta}(u,\alpha) + \bm{Q}_{\perp},Q_{\mid  \mid }))
\\
&\approx&  \frac{\alpha}{4 \pi} \int_{0}^{2 \pi} d \varphi_{\eta} \;
t^{\circ}_{if}(\bm{k},\bm{Q}) \int_{\alpha^{2}}^{\infty} \, \frac{d u
} {\sqrt[3]{u}} \\
&=&  \frac{\alpha}{4 \pi} \; t^{\circ}_{if}(\bm{k}, \bm{Q}) \, \int_{0}^{2 \pi}
d \varphi_{\eta} \left[ - 2 u^{-1/2}\right]_{\alpha^{2}}^{\infty} \\
&=&  \frac{\alpha}{4 \pi} \; t^{\circ}_{if}(\bm{k}, \bm{Q}) \; 2 \pi \,
\frac{2}{\alpha} \\
&=& t^{\circ}_{if}
\end{eqnarray*}
%
where the natural expression is recovered.

\[
P_{1 + 2 i \nu} (x) \bajo{\approx}{x \to 1} 1 + (1+ 2 i \nu)(1 + i
\nu) (x-1) + \mathcal{O}(x^{2})
\]

%
\section{Capture to bound states of the projectile in the CDW approximation}

In the case of final states where the electron is bound to the projectile,
the distortion $D_{P}$ \ref{Q:DFact} in the $C_{3}$ wavefunction \ref{Q:C3}
must be defined in terms of a bound state similar to the initial state
\ref{Q:wf-i}, $D_{P}=(2 \pi)^{3/2}\ \exp{\left(- i \bm{k}_{P} \cdot
\bm{r}_{P} \right)} \, \varphi_{f} (\bm{r}_{P})$.

The evaluation of the transition matrix is carried out in similar form that
in section \ref{S:appro-post-form} for ionization. For instance, the only
term that contributes in the approximation of masses for heavy ions
\begin{eqnarray} \label{Q:tif-N-cpt}
t^{N}_{if} &=& \frac{-1}{(2 \pi)^{3}}  \int d \bm{r}_{T} \,d \bm{r}_{P}
\, e^{i \, ( \mu_{T}\bm{v} \cdot \bm{R}_{T} - \bm{K}_{P}  \cdot
\bm{R}_{P} ) }\, \phi_{i}(\bm{r}_T) \,\left[ \nabla_{\bm{r}_{T}}
D^{-*}_{T}(\bm{k}^{\circ}_{T},\bm{r}_T) \right] {\times}
\\
&&  \left[ \vstretch \nabla_{\bm{r}_{P}} \phi_{f}(\bm{r}_{P}) \right] \,
D^{\circ +}_{P} (\bm{k}_{P}^{\circ},\bm{r}_{P}) \, \left[ D^{-\,\ast
\circ}_{N}(\bm{k}_{N}, \bm{r}_{N} ) \, D^{+ \circ }_{N}
(\bm{k}_{N}^{\circ},\bm{r}_{N}) \right] \; , \nonumber
\end{eqnarray}
where
\[
\bm{k}_{P}^{\circ} = - m_{P} \bm{v} \; , \qquad \bm{k}^{\circ}_{T} = m_{T} \bm{v}
\]


Now, expressing coordinates and momenta in terms of $\bm{r}_{P}$,
$\bm{r}_{T}$ and $\bm{K}_{R}=\bm{Q}$, the argument of the exponential is
\[
\bm{K}_{R} \cdot \bm{r}_{T} - \left( m_{P} \bm{v} + \frac{m_{P}}{m}
\bm{K}_{R} \right)\bm{r}_{P}
\]


In the simplest case, the internuclear interaction can be expressed as an
exponential factor which is switched-off in total cross sections calculations
\begin{eqnarray}\label{Q:tif-cpt-noNN-f}
t_{if}^{N} &=& \frac{-1}{(2 \pi)^{3/2}}  \int d \bm{r}_{T} \, e^{i
\, \bm{K}_{R} \cdot \bm{r}_T} \, \phi_{i}(\bm{r}_T) \,\left[
\nabla_{\bm{r}_{T}} D^{-*}_{T}(\bm{k}^{\circ}_{T},\bm{r}_T) \right]  \cdot
\nonumber \\
&& \int d \bm{r}_{P} \, e^{- i \, (m_{P} \bm{v} + \frac{m_{P}}{m}
\bm{K}_{R} + \bm{k}_{P}^{\circ}) \cdot \bm{r}_{P}} \left[\nabla_{\bm{r}_{P}}
\phi_{f}(\bm{r}_{P}) \right] \, \psi^{+}_{\bm{k}_{P}^{\circ}} (\bm{r}_{P}) \nonumber
\\
&=& (2 \pi)^{3} \bm{K}^{(T)}_{if}\left[ \bm{K}_{R} + \bm{k}_{T}^{\circ} \right]
\cdot \bm{G}^{(P)}_{if}\left[-\bm{K}_{R},-\bm{k}^{\circ}_{P}\right]
\end{eqnarray}
where we have used that
\[
\frac{m_{P}}{m} \bm{K}_{R} + m_{P}\bm{v} + \bm{k}_{P}^{\circ} = \frac{m_{P}}{m}
\bm{K_{R}} \approx \bm{K}_{R}
\]

Note that this result must be equal to the obtained by applying the
perturbation on the initial state,
%
\begin{eqnarray}\label{Q:tif-cpt-noNN-i}
t_{if}^{N} &=&  \int d \bm{r}_{T} \, e^{i \,(
\bm{K}_{R}+\bm{k}_{T}^{\circ}) \cdot \bm{r}_T} \,\left[ \nabla_{\bm{r}_{T}}
\phi_{i}(\bm{r}_T) \right] \, \psi^{-*}_{\bm{k}^{\circ}_{T}}(\bm{r}_T)   \cdot
\nonumber \\
&&\frac{-1}{(2 \pi)^{3/2}} \int d \bm{r}_{P} \, e^{- i \, \left(
\frac{m_{P}}{m} \bm{K}_{R} + \bm{k}_{P}^{\circ} \right) \cdot \bm{r}_{P}}
\phi_{f}(\bm{r}_{P}) \, \left[\nabla_{\bm{r}_{P}}
D^{+}_{P}(\bm{r}_{P},\bm{k}_{P}^{\circ}) \right]  \nonumber
\\
&=& (2 \pi)^{3} \bm{G}^{(T)}_{if}\left[ \bm{K}_{R} + \bm{k}_{T}^{\circ} \right]
\cdot \bm{K}^{(P)}_{if}\left[-\bm{K}_{R},-\bm{k}^{\circ}_{P}\right]
\end{eqnarray}

\subsection{Eikonal initial or final states}

The eikonal state is defined as the solution of the exact two-body problem
asymptotically. Within this approximation the distortion factor $D^{\pm}$ must
be replaced by its asymptotic limit as expressed in \ref{Q:Eikonal-wf}
\[
E^{\pm}_{\alpha}(\bm{k},\bm{r}) = \lim_{r \to \infty}
D^{\pm}_{\alpha}(\bm{k},\bm{r})
\]

In the case of Coulomb interactions, lets write the limit
\begin{eqnarray*}
\lim_{\lambda \to \infty} D^{\pm}(\nu_{k}, \bm{k},\lambda \bm{r}) &=& e^{{\pm}
i \, \nu_{k} \ln{\lambda \left( k r \mp \bm{k}\cdot \bm{r} \right)}}
\\
&=& \lim_{\lambda \to \infty} e^{\pm i \, \nu_{k} \ln{\lambda}} \, e^{
\pm i \, \nu_{k} \ln{\left( k r \mp \bm{k} \cdot \bm{r} \right)}}
\end{eqnarray*}
%
Then, we can evaluate the eikonal state as
%
\begin{equation}\label{Q:Eikonal-wf-Coul}
E^{\pm} (\bm{k}_{\alpha}, \bm{r}) = \lim_{\lambda \to \infty} e^{\mp i
\nu \ln{ \lambda}} \, D^{\pm}(\nu , \lambda \bm{k}_{\alpha}, \bm{r})
\end{equation}
where we have used that $D^{\pm} (\bm{k},\lambda \bm{r})= D^{\pm} (\lambda
\bm{k}, \bm{r})$.
%
The factors $\bm{G}_{if}$ in \ref{Q:tif-cpt-noNN-f} and
\ref{Q:tif-cpt-noNN-i} must be computed with the replacement $D \to E$.
