\chapter{Matrix elements between initial and final target states}

\section{Definitions}
\label{S:definitions}


\label{S:FGKL-if}

\begin{equation}\label{Q:Fif-def}
  F_{if}(\bm{p}) = \langle \phi_{f}\mid e^{i \bm{p}\cdot \bm{r}}\mid \phi_{i} \rangle = \frac{1}{(2 \pi)^{3/2}} \int d  \bm{r}_{T} \; e^{\, i \bm{p} \cdot \bm{r}_{T}} \, \phi_{i}(\bm{r}_{T}) \, \phi^{-\, \ast}_{f}(\bm{r}_T)
\end{equation}

\begin{equation}\label{Q:Lif-def}
\bm{L}_{if} = \langle \phi_{f}\mid  \bm{r} e^{i \bm{p}\cdot
\bm{r}}\mid \phi_{i} \rangle = \frac{1}{(2 \pi)^{3/2}} \int \bm{r}_{T}
\, d \bm{r}_{T} \; e^{\, i \bm{p} \cdot \bm{r}_{T}} \,
\phi_{i}(\bm{r}_{T}) \,
 \phi^{-\, \ast}_{f}(\bm{r}_T)
\end{equation}

\begin{equation}\label{Q:Fif-Lif-relat}
\bm{\nabla}_{\bm{p}} F_{if} = i \bm{L}_{if} =i \langle \phi_{f}\mid  \bm{r}
e^{i \bm{p}\cdot \bm{r}}\mid \phi_{i} \rangle
\end{equation}
%

We have defined also the vector $\bm{K}_{if}$ (eq. \ref{Q:Kif}),

\begin{equation}\label{Q:Kif-def}
  \bm{K}_{if}(\bm{p})= \frac{1}{(2 \pi)^{3}} \int d \bm{r}_{T} \;
e^{\, i \bm{p} \cdot \bm{r}_{T}} \, \phi_{i}(\bm{r}_{T}) \,
e^{- i \bm{k}_{T} \cdot \bm{r}_{T}} \,\nabla_{\bm{r}_{T}} \left[
D^{-*}(-\nu_{T},\bm{k}_{T},\bm{r}_{T}) \right]
\end{equation}

For a Coulomb final state it can be written as
\begin{eqnarray*}
\bm{K}_{if}(\bm{p}) &=& \frac{{N^{-}}^{*}(-\nu_{T})}{(2 \pi)^{3}} \int
d \bm{r}_{T} \; e^{\, i \left( \bm{p}- \bm{k}_{T} \right)
\cdot \bm{r}_{T}} \, \phi_{i}(\bm{r}_{T}) \,\nabla_{\bm{r}_{T}} \left[
\,{_{1}F_{1}}(- i \nu_{T}; 1; i \left( k_{T} r_{T} + \bm{k}_{T}
\cdot
\vect{r}_{T} \right) \right] \nonumber \\
\end{eqnarray*}

\section{Simple cases}
\label{S:simple-cases}

Let's consider the simple case of a Coulomb final state and an $1s$ initial state of the form 
\begin{equation*}
  \phi_{i}(\bm{r})= \frac{1}{\sqrt{\pi a^{3}}}\, e^{-\alpha r} 
\end{equation*}

Then, the form factor is 
\begin{subequations}
  \begin{align}
    \label{Q:Fif-hyd}
    F_{if}(\bm{p})&= \frac{1}{\sqrt{\pi a^{3}} \, (2\pi)^{3/2}} \int d\bm{r} e^{i \bm{p}\cdot \bm{r}} r^{n}\, e^{-\alpha r}  e^{-i \bm{k}\cdot \bm{r}} N^{-\,*}(\nu)~ _{1}F_{1}\left(-i\nu; 1; i (k r + \bm{k}\cdot\bm{r}) \right)\\
    &= \left(\frac{2}{a}\right)^{3/2} \frac{1}{(2 \pi)^{2}}\, N^{-\,*}(\nu) \, J_{0}(\alpha, \bm{p}-\bm{k}, -\nu, \bm{k}) \\
  N^{\pm}(\nu)&=\Gamma(1 \pm i \nu)\, e^{-\pi \nu/2} \qquad , \qquad \qquad  \nu= \frac{m\, Z_{1}Z_{2}}{\hbar k}
  \end{align}
\end{subequations}
where
\begin{align*}
  J_{0}(Z, \bm{p}, a_{1}, \bm{k})&= \frac{8\,\pi}{v_{1}\, D^{2} }\,\left(Z + a_{1}\frac{\left(k - i {Z\,u_{1} }\right) }{u_{1} +1}\right)\\
D&= Z^{2}+ p^{2}\\
u_{1} &= 2 \left( \bm{k} \cdot \bm{p} - i Z k \right)/D \\
v_{1}&= \left( u_{1}+1\right) ^{i\,a_{1}}
\end{align*}

Similarly, for $1s$ states of hydrogenic atoms the $\bm{K}$ factor may be written as:
\begin{equation}
  \label{Q:G-K-relation-1s}
  \bm{K}(\bm{p}) = \frac{{N^{-}}^{*}(-\nu_{T})}{(2 \pi)^{3}}\,\left(\frac{2}{a}\right)^{3/2}\, \bm{G}_{0}(\alpha, \bm{p}-\bm{k}_{T}, - \nu_{T}, \bm{k}_{T})
\end{equation}

For the $\bm{L}$-factor we obtain
\begin{equation}
  L_{if}(t) = -i \bm{\nabla}_{\bm{A}} \,F_if = -i \left(\frac{2}{a}\right)^{3/2} \frac{1}{(2 \pi)^{2}} \,N^{+}(\nu) \, \bm{\nabla}_{\bm{A}} \left( J_{0}(1/a, -(\bm{A}(t)+\bm{k}), -\nu, \bm{k}) \right)
\end{equation}
where the gradient may be explicitly written as:
%
\begin{align*}
\frac{d J_{0}}{d p_{z}} &= \left[ \bm{\nabla}_{\bm{p}} \,J_{0}(Z, \bm{p}, a_{1}, \bm{k}) \right]_{z} \\
&-\frac{16  i \pi a_{1}}{D^{3} v_{1} (u_{1}+1)} (k_{z}- p_{z} u_{1})\left[Z + a_{1} \frac{k - i Z u_{1}}{u_{1}+1} \right]
\\
&-\frac{16 \pi p_{z} }{D^{3} v_{1}}  \left[Z + a_{1} \frac{k - i Z u_{1}}{u_{1}+1} \right]
\\
&+ \frac{-16 \pi}{D^{3} v_{1}} \left\{  p_{z} Z + \frac{a_{1}}{u_{1}+1} (i k_{z} Z +  k p_{z} - 2 i p_{z} Z u_{1})  + (\frac{a_{1}}{u_{1}+1} (k_{z} - p_{z} u_{1}) ( k- i Z u_{1}))/(u_{1}+1) \right\} 
\end{align*}
Now, we can rearrange the last line as
\begin{align*}
& \frac{-16 \pi}{D^{3} v_{1}} \left\{  p_{z} Z + \frac{a_{1}}{u_{1}+1} \left( i Z (k_{z} - p_{z} u_{1}) +  p_{z} (k- i Z u_{1})  \right)  + (\frac{a_{1}}{u_{1}+1} (k_{z} - p_{z} u_{1}) ( k- i Z u_{1}))/(u_{1}+1) \right\} \\
& \frac{-16 \pi}{D^{3} v_{1}} \left\{  p_{z}  \left[Z + a_{1} \frac{k - i Z u_{1}}{u_{1}+1} \right] +  i Z a_{1} \frac{k_{z} - p_{z} u_{1}}{u_{1}+1} + a_{1} \frac{k_{z} - p_{z} u_{1}}{u_{1}+1}\frac{ k- i Z u_{1}}{u_{1}+1} \right\} 
  \end{align*}
Thus, defining the following variables:
\begin{subequations}
  \begin{align}
    D&= p^{2}+ Z^{2}   &&  u_{1}=2 \frac{\left( \bm{k} \cdot \bm{p} - i Z k \right)}{D} &&  v_{1}=(u_{1}+1)^{i a_{1}} \\
    A&= -\frac{16 \pi}{D^{3} v_{1}} && t_{1}=\frac{ k- i Z u_{1}}{u_{1}+1} &&
    t_{2}=\frac{k_{z} - p_{z} u_{1}}{u_{1}+1} && t_{3}= Z + a_{1}\,t_{1}
  \end{align}
\end{subequations}
and adding the three lines above we get:
\begin{equation} \label{Q:grad-J0-z}
  \left[ \bm{\nabla}_{\bm{p}} \,J_{0}(Z, \bm{p}, a_{1}, \bm{k}) \right]_{z} = A \left[ i a_{1}\, t_{2}\, t_{3} + 2 p_{z}\,t_{3} + i Z \, a_{1}\, t_{2} + a_{1}\, t_{1}\, t_{2} \right]
\end{equation}




% \begin{align*}
% &-(4  i \pi a1 ((2 kz)/D-(4 pz D u1/2)/D^2) (v_{2}^{-1}) (
%   ( i a1 (-(2  i k)/D-(4 Z D u1/2)/D^2))/(u1+1)+(2 Z)/D))/D
% \\
% &-(8 \pi pz (( i a1 (-(2  i k)/D-(4 Z D u1/2)/D^2))/(u1+1)+(2 Z)/D))/(D^2 (u1+1)^( i a1))
% \\
% &+ 4 \pi (\\ 
%   &         ( i a1 (-(4 kz Z)/D^2+(4  i k pz)/D^2+(16 pz Z D u1/2)/D^3))/(u1+1) \\
%   &        -( i a1 ((2 kz)/D-(4 pz D u1/2)/D^2) (-(2  i k)/D-(4 Z D u1/2)/D^2))/(u1+1)^2 \\
%   &        -(4 pz Z)/D^2 \\
%   &        )/(D (u1+1)^( i a1)) \\
% \end{align*}

%%% Local Variables: 
%%% mode: latex
%%% TeX-master: "main"
%%% End: 
