\chapter{Semiclassical approach to multiple electron processes}
\label{C:Semic-appro-multi-elect-proce}

\section{Single and multiple ionization of complex atoms}
\label{S:Single-multipl-ionizat-complex-atoms}

In a semiclassical impact parameter approach, within the independent
particle approximation, the single and multiple ionization cross
sections can be evaluated from the one electron transition
probabilities. We will suppose that:
\begin{itemize}
\item The electrons evolve independently  in the (model) field of the
nuclei,
\item The projectile can be described classically (its trajectory is
well defined),
\item The  heavy particles movement can be decoupled from the electron's
  trajectories.
\end{itemize}

If the target has N electrons in a given shell, the probability of
ionizing one electron with momentum $\bm{k}$ by a projectile that
impinged with impact parameter $\rho$ in the independent electron model
(IEM) $p_{i} \equiv p_{i}(\bm{k},\rho) $ determines the cross section
for emission of $q$ electrons. The probability of single-ionization
from a two-electron atom is equal to the probability of ionize on
electron while the other is not ionized $\mathcal{P}_{SI} = p(1-p)$. In
rigor, if we are evaluating the probability of emission of only one
electron (which will have momentum $\bm{k}$), we have to ensure that
the remaining electron is not emitted at all. The probability of single
ionization would be
\begin{equation}\label{Q:SionPrb}
\mathcal{P}_{SI}(\bm{k},\rho) = 2\, p_{i}(\bm{k},\rho)\, \int \left[1 -
p_{i}(\bm{k}',\rho) \right] \, d \bm{k}' = 2\, p_{i}(\bm{k},\rho)\,
\left[1 - P_{i}(\rho) \right] \, ,
\end{equation}
where $P_{i}(\rho)$ is the probability of ionizing one electron. The
total probability is obtained by integrating over the momenta of the
``detected'' electron
\[
\mathcal{P}_{SI}(\rho) = 2 \, \left[ 1-P_{i}(\rho) \right] \, \int d
\bm{k} \, p_{i}(\bm{k},\rho)\, = 2 \, \left[ 1-P_{i}(\rho) \right] \,
P_{i}(\rho)\,
\]
Integrating over the impact parameter we obtain the corresponding cross
section
\[
\sigma_{SI} = \int_{0}^{\infty} \rho \,\mathcal{P}_{SI}(\rho) d \rho
= 2 \, \int_{0}^{\infty} \rho \,\left[ 1-P_{i}(\rho) \right] \,
P_{i}(\rho)\, d \rho
\]


It is possible generalize this analysis to the ionization of q
electrons from an atom, initially with N electrons, where one of them
is emitted with momentum $\bm{k}$. This reasoning lead us to the
well-known binomial analysis \autocite[see, for instance,][]{Mcguire1991AAMOP_MEIp217}. The probability of
$q_{m}$ electrons ionized from a given shell $m$ with initially $N_{m}$
electrons  is given by
\begin{equation} \label{Q:qionshllPrb}
\mathcal{P}_{q_{m}}(\bm{k},\rho) =  {N_{m} \choose q_{m}}
p_{m}(\bm{k},\rho) \left[ P_{m} (\rho) \right]^{q_{m}-1} \, \left[ 1 -
P_{m}(\rho) \right]^{N_{m}-q_{m}}.
\end{equation}

When several shells have appreciable probability of being ionized we
have to add up all the contributions that may give to the total $q$ of
ionized electrons. We have to include all the terms that contribute to
the observed final state. We impose the condition that the total of
emitted electrons be $q$, which is expressed by $q = \sum_{m=1}^{N_{m}}
q_{m}$ \autocite{Mcguire1991AAMOP_MEIp217,Kirchne2002PRAp42727},
\begin{equation}\label{Q:qionPrb}
\mathcal{P}_{q}(\bm{k},\rho) = \sum_{\left( q_{m} | \sum q_{m}=q
\right)}^{N_{1},\ldots,N_{m}} \;\prod_{i=1}^{m}
  \,\mathcal{P}_{qi}(\bm{k},\rho)
\end{equation}

The total electron emission production is given by
\begin{equation}\label{Q:TEEPrb}
\mathcal{P}_{Tee} \equiv \sum_{q} q \, \mathcal{P}_{q} = \sum_{m} P_{m}
\end{equation}


\section{Computation of the transition probability}
\label{S:Compu-trans-proba}

The transition amplitude and matrix are bi-dimensional Fourier
transform of each other
\begin{eqnarray}\label{Q:prob1}
\mathcal{A}_{if}(\bm{k},\brho) &=& \frac{1}{(2 \pi)} \int d
\bm{Q}_{\perp} \;
e^{- i \, \bm{Q} \cdot \brho } \; t_{if}(\bm{k},\bm{Q}) \\
t_{if}(\bm{k},\bm{Q}) &=& \frac{1}{(2 \pi)} \int d \brho \;
e^{i \, \bm{Q} \cdot \brho } \;\mathcal{A}_{if}(\bm{k},\brho)
\end{eqnarray}

Following several works \autocite{Rodrigu1996JPBp275,Galassi2002JPBp1727} we Fourier expand the transition matrix in the azimuthal angle
\[
t_{if}(\bm{k},\bm{Q}) = \frac{1}{\sqrt{2 \pi}} \sum_{m=-\infty}^{\infty} t^{m}_{if}(\bm{k},Q_{\perp}) \, e^{i m \varphi_{k Q}} \, .
\]
where each term of the series is given by
\[
t^{m}_{if}(\bm{k},Q_{\perp}) = \frac{1}{\sqrt{2 \pi}} \int_{0}^{2 \pi}
d \varphi_{k Q}\; e^{- i \,m\, \varphi_{k Q} } \;
t_{if}(\bm{k},\bm{Q}) .
\]
Introducing this expansion in the equation \ref{Q:prob1} we obtain
\[
\mathcal{A}_{if}(\bm{k},\brho) = \sum_{m=-\infty}^{\infty} \left\{
\int_{0}^{\infty} Q_{\perp} d Q_{\perp} \;
t^{m}_{if}(\bm{k},Q_{\perp}) \; \left[ \int_{0}^{2 \pi} d \varphi \,
e^{- i \, Q_{\perp} \rho \cos{\varphi}}\ e^{i m \varphi}
\right] \right\}\, \frac{e^{i m \varphi_{\rho k}}}{(2 \pi)^{3/2}}
\]

Before we continue we note that the transition matrix contains an
additional symmetry, such that it only depends on the relative
azimuthal angle. Moreover,
\[
t_{if}(k,\theta_{k},Q_{\perp},\varphi_{k Q}) =
t_{if}(k,\theta_{k},Q_{\perp},2 \pi - \varphi_{k Q})
\]

For every function with this symmetry ($f(\varphi)=f(2 \pi - \varphi)$)
the one-dimensional Fourier transform can be written as
\begin{eqnarray} \label{Q:cFT-tm}
\frac{1}{\sqrt{2 \pi}} \int_{0}^{2 \pi} d \varphi \; e^{- i
\,m \, \varphi } \; f(\varphi) &=& \frac{1}{\sqrt{2 \pi}} \left(
\int_{0}^{\pi} d \varphi \; e^{- i \,m \, \varphi} \;
f(\varphi) + \int_{0}^{\pi} d \varphi \; e^{- i \,m \, (2 \pi
- \varphi) } \;
f(\varphi) \right) \nonumber \\
&=& \frac{1}{\sqrt{2 \pi}} \int_{0}^{\pi} d \varphi \; \left(
e^{- i \,m \, \varphi} + e^{i \,m \, \varphi} \right) \;
f(\varphi)   \nonumber \\
&=& \frac{2}{\sqrt{2 \pi}} \int_{0}^{\pi} d \varphi \; \cos{(m \,
\varphi) } \; f(\varphi)
\end{eqnarray}

The use of the symmetry shows two important results: first, the domain
of integration is only half of the angular range and secondly that
$t^{m}_{if} = t^{-m}_{if}$, while the cosine is an even function. The
same is true for the integral over $\varphi$ in the above expression of
the impact parameter probability. In fact it can be written in terms of
the Bessel functions using (\ref{S:Bess-n})
\begin{align}
  \int_{0}^{2 \pi} d \varphi \,
e^{- i \, Q_{\perp} \rho \cos{\varphi}}\ e^{i m \varphi} \nonumber
&=2
 \int_{0}^{\pi} d \varphi \, e^{- i \, Q_{\perp} \rho
\cos{\varphi}}\ \cos{(m \varphi)}
\\
&= 2 \pi \, (-i)^{|m| } \, J_{|m| }(Q_{\perp} \rho) \,.
\end{align}
%
Thus, the probability can be written
\begin{subequations}
  \begin{align}
    \mathcal{A}_{if}(\bm{k},\brho) &= \sum_{m=-\infty}^{\infty} \, (-i)^{|m| } S_{|m|
    }(\bm{k}, \rho) \, e^{i m \varphi_{\rho k}}
    \\
    S_{m}(\bm{k}, \rho) &= \frac{1}{\sqrt{2 \pi}} \int_{0}^{\infty} Q_{\perp} \; t^{|m|
    }_{if}(\bm{k},Q_{\perp}) \; J_{|m| }(\rho \, Q_{\perp}) \, d
    Q_{\perp} \\
    t^{m}_{if}(\bm{k},Q_{\perp}) &= \frac{2}{\sqrt{2 \pi}} \int_{0}^{2 \pi}
    t_{if}(\bm{k},\bm{Q}) \; \cos{(m \, \varphi_{k Q})} \; d \varphi_{k Q}
  \end{align}
\end{subequations}
where $J_{m}(z)$ is the integer order $m$ Bessel function.

Integrating over the azimuthal angle of the impact parameter we obtain
the probability of ionizing one electron with momentum $\bm{k}$
\begin{eqnarray}\label{Q:prob2}
p_{i} (\bm{k}, \rho) &=& \frac{1}{2 \pi} \int_{0}^{2 \pi} d
\varphi_{\rho k} \left|  \mathcal{A}_{if}(\bm{k},\brho) \right| ^{2} =
\sum_{m=-\infty}^{\infty} \left|  S_{|m| }(\bm{k}, \rho) \right| ^{2}
\nonumber
\\
&=& \left|  S_{0}(\bm{k},\rho) \right| ^{2} + 2 \, \sum_{m=1}^{\infty}
\left| S_{m}(\bm{k},\rho) \right| ^{2}
\end{eqnarray}
%
which coincides with previous results \cite{Galassi2002JPBp1727}.

\subsection{Normalization of the transition probability}
\label{S:Norma-trans-proba}

The probability transition is defined such that the probability of
ionize on electron of the $m$ shell
\[
\sigma_{Tee}(m) = \int_{0}^{\infty} \rho \, P_{m}(\rho) \, d \rho \,
,
\]
has to be equal to
\[
\int \frac{d \sigma}{d \bm{k}} \, d \bm{k} = \frac{(2
\pi)^{5}}{v^{2}} \int |t_{if}(\bm{k},\bm{Q})|^{2} d Q_{\perp}
\]

Then, the transition matrix and the probability are related, within our
definitions, by
\begin{equation}\label{Q:Relat-trans-proba-matri}
\frac{d \sigma}{d \bm{k}} = \int_{0}^{\infty} \rho \, p_{i}(\rho)
\, d \rho = \frac{(2 \pi)^{4}}{v^{2}}\int_{0}^{\infty} Q_{\perp}
|t_{if}(Q_{\perp})|^{2} \, d Q_{\perp} \, .
\end{equation}
%
Then, the transition matrix is related to the quantity found in the
literature $R_{if}$ by
\begin{equation}\label{Q:RelRifTif}
  R_{if}(Q_{\perp}) = \frac{(2 \pi)^{5/2}}{v} \ t_{if}(Q_{\perp})
\end{equation}


%%%%%%%%%%%%%%%%%%%%%%%%%%%%%%%%%%%%%%%%%%%%%%%%%%%%%%%%%%%%%%%%%%%%%%%%
%%%%%%%%%%%%%%%%%%%%%%%%%%%%%%%%%%%%%%%%%%%%%%%%%%%%%%%%%%%%%%%%%%%%%%%%
\chapter{Simultaneous ionization from different centers}

\section{Hamiltonian of the problem}

We start by considering a collision between two atomic species, each of
them having one or more electrons. The corresponding Hamiltonian is
given by \autocite{Fiol2001JPBpL503}
\begin{eqnarray}\label{Q:simH1}
H &=& H_{0} + V_{N_{P},e_{P}} + V_{N_{T},e_{T}} +  V_{N_{P},N_{T}} +
V_{N_{T},e_{P}} + V_{N_{P},e_{T}} + V_{e_{P}, e_{T}} \nonumber \\
&=& H_{P} + H_{T} + V_{N_{P},N_{T}} + V_{N_{T},e_{P}} + V_{N_{P},e_{T}}
+
V_{e_{P}, e_{T}} \nonumber \\
&=& H_{P} + H_{T} + V_{N_{P},N_{T}}
\\
&+& \left[ V_{N_{T},e_{P}} + \langle V_{e_{P}, e_{T}} \rangle_{T}
\right] + \left[ V_{N_{P},e_{T}} + \left\langle V_{e_{P}, e_{T}}
\right\rangle_{P} \right] + \left[  V_{e_{P}, e_{T}} - \langle
V_{e_{P}, e_{T}} \rangle_{T} - \langle  V_{e_{P}, e_{T}} \rangle_{P}
\right] \nonumber
\end{eqnarray}
Here $H_{P,T}$ are the hamiltonian of each center, $\langle
V_{e_{T},e_{P}} \rangle $  is the interaction of projectile electrons
with the mean field produced by the target electrons. The first term
between square brackets model the screened nuclear interaction felt by
the projectile electrons. A similar interpretation can be made of the
second set of square brackets. It is the average potential felt by the
target electrons due to the screened projectile. The last term accounts
by the difference between mean fields and the real interactions. It can
be interpreted as the dynamical contribution of the e-e interaction,
since the static part is contained in the two first brackets.

We model the screened interaction of each center with each electrons in
the other center through a single potential $\mathcal{V}_{T,P}$. Thus
the hamiltonian is
\[
H = H_{P} + H_{T} + V_{N_{T},N_{P}} + \sum_{i_{P}}
\mathcal{V}_{T,e_{P}} + \sum_{i_{T}} \mathcal{V}_{P,e_{T}} + \left[
V_{e_{P}, e_{T}} - \langle V_{e_{P}, e_{T}} \rangle_{T} - \langle
V_{e_{P}, e_{T}} \rangle_{P} \right]
\]

\section{Modelling of simultaneous target and projectile ionization}


We are interested in the simultaneous target and projectile ionization
by the collision of two atomic/ionic species at large relative
velocities. In this range of impact energy the First Born approximation
(FBA) is valid for describing single ionization in three-body problems.

In the three-body FBA the internuclear potential can be safely
switched-off for describe cross sections related to the electron
momentum. The final state in this approximation include the (free)
motion of the projectile, while the electron does not interact with the
projectile in the final state.

Now we will try to generalize the FBA for the process of interest
including, at least, four bodies. Additionally we will incorporate some
of the interactions neglected in the FBA calculations. We consider
first the simplest case in which the target and projectile contain only
one electron. The transition matrix element for the simultaneous
ionization in a perturbative approach is given by $t_{if} = \langle
\Psi^{*-}_{f} \mid V\mid \Phi_{i} \rangle$

\begin{eqnarray}\label{Q:Sim1}
\Psi^{\pm}_{f} &=& e^{i (\bm{k}_{T} \cdot \bm{r}_{T} + \bm{k}_{T,2}
\cdot \bm{r}_{T,2} + \bm{K}_{T} \cdot \bm{R}_{T})} \,
D^{\pm}_{\nu_{T,1}}\, D^{\pm}_{\nu_{P,2}}\,\left[  D^{\pm}_{\nu_{1,2}}
\right] \, \left[ D^{\pm}_{\nu_{T,P}}\, D^{\pm}_{\nu_{T,2}}\,
D^{\pm}_{\nu_{P,1}} \right]
\nonumber \\
\Phi_{i} &=& e^{i \bm{P}_{T} \cdot \bm{R}_{T}} \;
\phi^{f}_{T}(\bm{r}_{T,1}) \phi^{f}_{P}(\bm{r}_{P,2})
\nonumber \\
\\
V &=&  V_{N_{T},e_{P}} + V_{N_{P},e_{T}} + V_{e_{P}, e_{T}} + \left[
V_{N_{P},N_{T}} \right] \, . \nonumber
\end{eqnarray}

The terms between square brackets will be neglected in the following
calculations. This approximation is the equivalent to the FBA for four
bodies. While the final velocity of the projectile electron (electron
2) is expected to be close to its initial velocity (projectile incident
velocity) and the target electron  (electron 1) velocity is expected to
be small, the interaction of each electron with the opposite center is
neglected in the final state.


In the case of atoms with more than one electron, the final state has
to be modified by the inclusion of wavefunctions describing the
internal state of the projectile and target (discarding the terms
between brackets)
\[
\Psi^{\pm}_{f} = e^{i (\bm{k}_{T} \cdot \bm{r}_{T} + \bm{k}_{T,2}
\cdot \bm{r}_{T,2} + \bm{K}_{T} \cdot \bm{R}_{T})} \,
D^{\pm}_{\nu_{T,1}}\, D^{\pm}_{\nu_{P,2}}\, D^{\pm}_{\nu_{1,2}} \;
\phi^{f}_{T} \; \phi^{f}_{P}
\]


\subsection{Evaluation of the transition matrix}

The transition matrix for a process in which the projectile and target
change from initial states $\mid \Phi_{P,\,i} \rangle $, $\mid
\Phi_{T,\,i} \rangle $ to final states $\mid \Phi_{P,\,f} \rangle$ and
$\mid \Phi_{T,\,f} \rangle$ is given by
\begin{eqnarray}\label{Q:Sim2}
t_{if} = \langle \Phi_{P,\,f} \mid
\end{eqnarray}



\section{Stripping cross sections at high energy}

\subsection{First Born Approximation (FBA)}

The simplest approximation in the evaluation of total cross sections is
given by the FBA. In a first order we will neglect most of multi
electron effects together with the effect of dynamical
electron-electron interactions.

Within this model the problem simplifies to the projectile ionization
by the collision with a structureless target. The interaction between
the two projectile particles and the target is given by a screened
potential modelling the average effect of the nucleus and all target
electrons.

The transition matrix is given by

\[
t_{if} = \langle \mid
\]


%%% Local Variables: 
%%% mode: latex
%%% TeX-master: "mainxs"
%%% End: 
