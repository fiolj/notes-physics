\chapter{Three-body kinematics}

\section{Coordinate system}
\label{S:coord-syste}


Let us consider the collision of a projectile of mass $M_P$, charge $Z_P$ and velocity $\bm{v}$ (impulse $\bm{P} = M_P \bm{v}$) with a two-particle target of mass $M_T + m$ initially at rest in the laboratory reference system. Without any loss of generality we shall consider that $m \leq M_T$. The total energy is given by $E_{\rm Lab} = M_P v^2 / 2 + \varepsilon_i$, where $\varepsilon_i$ is the internal energy of the target. By extracting the constant energy $M v_{CM}^2 / 2$ associated with the movement with velocity $\bm{v}_{CM} = M_P \bm{v} / M$ of the whole system of mass $M = M_T + M_P + m$, we obtain the energy in the center-of-mass reference frame $E_i = \mu_T v^2 / 2 + \varepsilon_i$. Here $\mu_T = (m + M_T) M_P / (M_T + M_P + m)$ is the reduced mass of the initial projectile-target configuration.

In the center-of-mass reference system, the three-body problem can be described by any of three possible sets of Jacobi coordinates (Macek and Shakeshaft 1980),
\[
\vec{x}_T = \left( \begin{matrix} \bm{r}_T \cr \bm{R}_T \cr
  \end{matrix}
\right) \, , \, \vec{x}_P = \left( \begin{matrix} \bm{r}_P \cr \bm{R}_P \cr
  \end{matrix}
\right) \, , \, \vec{x}_N = \left( \begin{matrix} \bm{r}_N \cr \bm{R}_N \cr
  \end{matrix}
\right) \; ,
\]
as shown in \ref{f:jacobi}. In our notation, $\bm{r}_T$, $\bm{r}_P$ and $\bm{R}_N$ are the position vectors of the particle of mass $m$ relative to the recoiling target fragment T, the projectile P and the center of mass of T + P, respectively. $\bm{R}_P$ is the position vector of the centre of mass of $m$ + P relative to T; and $\bm{r}_N$ and $\bm{R}_T$ are the coordinates of P relative to T and the centre of mass of $m$ + T, respectively. These Jacobi coordinates are related by $\vec{x}_j = {\cal M}_{j \ell}\ \vec{x}_\ell$, for $j , \ell = T$, $P$ or $N$, with
\begin{eqnarray*} {\cal M}_{PT} = \left(\begin{matrix}\alpha & -1 \cr 1- \alpha \beta & \beta \cr
    \end{matrix}
  \right) & & \; {\cal
    M}_{TP} = \left( \begin{matrix} \beta & 1 \cr \alpha \beta - 1 & \alpha \cr 
    \end{matrix}
  \right) \\  {\cal
    M}_{NT} = \left(\begin{matrix} 1 - \alpha & 1 \cr \alpha + \gamma - \alpha \gamma & \gamma - 1 \cr
    \end{matrix}
  \right) \qquad & & \; {\cal M}_{TN} = \left(\begin{matrix}1 - \gamma & 1 \cr \alpha + \gamma - \alpha
      \gamma & \alpha - 1 \cr 
    \end{matrix}
  \right) \\  {\cal M}_{NP} = \left( \begin{matrix} \beta - 1 & 1 \cr 1-
      \gamma (1- \beta) & \gamma \cr 
    \end{matrix}
  \right) & & \; {\cal M}_{PN} = \left( \begin{matrix} -\gamma & 1 \cr 1
      - \gamma (1- \beta) & 1 - \beta \cr 
    \end{matrix}
  \right) \; ,
\end{eqnarray*}
where
\[
\qquad \alpha= \frac{M_T}{m + M_T} \qquad , \qquad \beta=\frac{M_P}{m + M_P} \qquad \mbox{and} \qquad \gamma = \frac{M_T}{M_P + M_T} \; .
\]
These pairs of coordinates diagonalize the free Hamiltonian $H_o$ in the center-of-mass reference frame.

In momentum space, the system is described by the associated pairs $(\bm{k}_T, \bm{K}_T)$, $(\bm{k}_P, \bm{K}_P)$ and $(\bm{k}_N,\bm{K}_N)$ which are related by
\[
\left( \begin{matrix} \bm{k}_j \cr \bm{K}_j \cr \end{matrix} \right) = {\cal M}^t_{\ell j} \left( \begin{matrix} \bm{k}_\ell \cr \bm{K}_\ell \cr
  \end{matrix}
\right) \; ,
\]
where the superscript $t$ indicates the transposition of the matrix elements.

Switching back to the Laboratory reference frame, $\bf k$, $\bm{K}_R$ and $\bm{K}$ are the final momenta of the electron of mass $m$, the (recoil) target fragment of mass $M_T$ and the projectile of mass $M_P$.

In the mass-restricted problems, it is assumed that the target nucleus remains motionless throughout the collision. Additionally, in the Poincar\'{e} restricted problem the projectile momentum $\bf K$ is bound to remain fixed at its initial value $M_{P} \bm{v}$. As a consequence, the electron momentum is identified either with the Jacobi electron-target momentum, $\bm{k} \equiv \bm{k}_{T}$, or with the electron-projectile momentum, $\bm{k} \equiv \bm{k}_{P}+ m \bm{v}$. In particular, the electron double-differential cross-section $d \sigma / d \bm{k}$ is assumed to be equal either to $d \sigma / d \bm{k}_T$ or $d \sigma / d \bm{k}_P$. However, this is not generally valid for any mass configuration. One of the main features of the present formulation is that we are not neglecting any $\alpha$, $\beta$ or $\gamma$ terms.  In our approach these parameters can be changed arbitrarily from very
small to very large numbers in order to cope with any possible mass configuration. However, each momentum in the laboratory reference frame can still be written in terms of a given Jacobi impulse $\bm{K}_j$ as
\begin{equation}\label{Q:1}
  \qquad \bm{k} = m \bm{v}_{CM} + \bm{K}_N \quad , \quad \bm{K} =
  M_P \bm{v}_{CM} + \bm{K}_T \quad \mbox{and} \quad \bm{K}_R = M_T
  \bm{v}_{CM} - \bm{K}_P \; ;
\end{equation}
so that, for instance, $d \sigma / d \bm{k} = d \sigma / d \bm{K}_N$. In general, the multiple-differential cross-section for ionization collisions in the impulse of the particles can be written in terms of the Jacobi impulses ($j=T$, $P$ o $N$),
\begin{equation}\label{Q:2}
  \frac
  {d \sigma} {d \bm{k} \, d \bm{K} \,d \bm{K}_R} = \delta
  \left( \bm{k} + \bm{K} + \bm{K}_R - M \bm{v}_{CM}\right) \, \frac {d
    \sigma} {d \bm{k}_j \, d \bm{K}_j} \; .
\end{equation}
Here (atomic units are used throughout)
\begin{equation}\label{Q:3}
  \frac {d \sigma} {d \bm{k}_j \, d \bm{K}_j}= \,
  \frac{(2\pi)^4}{v} |t_{if}|^2 \, \delta \left(E_i - \frac{1}{2 m_j}
    k_j^2 - \frac{1}{2 \mu_j} K_j^2 \right) \; ,
\end{equation}
where $t_{if}$ is the transition matrix element. We have defined the reduced masses of each two- and three-body system
\begin{eqnarray*}
  m_T = \alpha m = \frac{m M_T}{m + M_T} & & \; \mu_T = \frac {\beta m} {1 - \alpha \beta} =
  \frac{(m + M_T)M_P}{m + M_T + M_P} \\ m_P = \beta m = \frac{m M_P}{m + M_P} & & \; \mu _P = \frac
  {\alpha m} {1 - \alpha \beta} = \frac{(m + M_P) M_T}{m + M_T + M_P} \\ m_N=\gamma M_P = \frac{M_T
    M_P}{M_T + M_P} \qquad & & \; \mu_N = \frac{(M_P + M_T) m} {m + M_T + M_P} \; .
\end{eqnarray*}
such that
\[
m_j \mu_j = \frac {m M_T M_P} {m + M_T + M_P} \qquad \mbox{for $j$ = $N$, $T$ or $P$.}
\]

In terms of the laboratory momenta $\bm{k}, \bm{Q}$ the Jacobi momenta are written:
\begin{align*}
  \bm{k}_{T}&= \bm{k} - \frac{m_{T}}{M_{T}}\, \bm{Q} \,, &
  \bm{k}_{P}&= \frac{m_{P}}{m}\,\bm{k} + \frac{m_{P}}{M_{P}}\, \bm{Q} - m_{P}\bm{v} \, , &
  \bm{k}_{N}&= \frac{m_{N}}{M_{T}}\,\bm{k} -  \bm{Q} + m_{N}\bm{v} &
\end{align*}

%%% Local Variables: 
%%% mode: latex
%%% TeX-master: "mainxs"
%%% End: 
