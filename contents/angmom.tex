
\chapter{Angular momentum}
\label{C:angular-momentum}


\section{Clebsch-Gordan coefficient}
\label{S:C-G}

Definition of the Clebsch-Gordan coefficients
[\cite[(B.41)]{Galindo1990_QMvI}, \cite[(3.18,3.19)]{Rose1967_ETO} and also
\cite[(3.6.11)]{Edmonds1960_AMI}] for the sum of two angular momenta $j_{1}, j_{2}$ giving $j_{3}$
%
\begin{align}\label{Q:C-G-definition}
C(j_{1} \, j_{2} \, j_{3}; m_{1} \, m_{2} \, m_{3}) \equiv & \langle j_{3} m_{3}|j_1 m_1 j_2 m_2\rangle  \equiv \langle j_1 m_1 j_2 m_2|j_{3} m_{3}\rangle \nonumber \\ 
&= \delta_{m_{1} +m_{2},m_{3}} \, \Delta(j_{1},j_{2},j_{3}) \, \sqrt{2
j_{3}+1}  \qquad\qquad
\\
&{\times} \left[ \frac{(j_{1}+j_{2}-j_{3})!\,
(j_{3}+j_{2}-j_{1})!\,(j_{1}+j_{3}-j_{2})!\,(j_{3}+m_{3})!\,(j_{3}-
m_{3})! }{(j_{1} +j_{2} + j_{3} +1)! \,(j_{1}- m_{1})! \,(j_{1}+
m_{1})! \,(j_{2}-
m_{2})! \,(j_{2}+ m_{2})!} \right]^{1/2} \nonumber \\
&{\times} \sum_{s} \frac{(-1)^{s + j_{2} + m_{2}}}{s!}\frac{(j_{2} + j_{3} +
m_{1} - s)! \, (j_{1} - m_{1} + s)! \,}{(j_{2} + j_{3} - j_{1} - s)! \,
(j_{3} + m_{3} - s)! \, (j_{1} - j_{2} - m_{3} + s)! \,} \nonumber
\\
\nonumber \\
&= \delta_{m_{1} +m_{2},m_{3}} \, \Delta(j_{1},j_{2},j_{3}) \, \sqrt{2
j_{3}+1} \label{Q:spf1b}
\\
{\times}& \left[ \frac{(j_{1}+j_{2}-j_{3})!\, (j_{3}+j_{2}-j_{1})!\,
(j_{1}+j_{3}-j_{2})!}{(j_{1} +j_{2} + j_{3} +1)!} \right]^{1/2}
\nonumber \\
& {\times}  \,(j_{1}- m_{1})! \,(j_{1}+ m_{1})! \,(j_{2}- m_{2})! \,(j_{2}+
m_{2})! \,(j_{3}+m_{3})!\,(j_{3}- m_{3})! \nonumber
\\
&{\times} \sum_{s} \frac{(-1)^{s}}{s!} \left[ \frac{1}{(j_{1} + j_{2} + j_{3}
- s)! \, (j_{1} - m_{1} - s)! \, (j_{2} + m_{2} - s)! } \right.
\nonumber
\\
& \qquad \qquad \quad \left. \frac{1}{\,(j_{3} + j_{2} + m_{1} + s)!
\, (j_{3} - j_{1} - m_{2} + s)!} \right] \nonumber
\end{align}
%
where the triangular function is
\[
\Delta(j_{1},j_{2},j_{3}) = 
\begin{cases}
1 & \text{if $ \mid j_{1}-j_{2} \mid$ $ < j_{3}<j_{1}+j_{2} $} \\
0 & \text{other case}
\end{cases}
\]
%


\subsection{Symmetries}
\label{S:C-G-symm}

\begin{subequations}
  \begin{align}
    \langle j_1 m_1 j_2 m_2|J M\rangle
    \label{Q:C-G-symmetries-1} & = (-1)^{j_1+j_2-J} \langle j_1\, {-m_1} j_2 \, {-m_2}|J \, {-M}\rangle \\
    \label{Q:C-G-symmetries-2} & = (-1)^{j_1+j_2-J} \langle j_2 m_2 j_1 m_1|J M\rangle \\
    \label{Q:C-G-symmetries-3} & = (-1)^{j_1 - m_1} \sqrt{\frac{2 J +1}{2 j_2 +1}}  \langle j_1 m_1 J \, {-M}| j_2\,{-m_2} \rangle \\
    \label{Q:C-G-symmetries-4} & = (-1)^{j_1 - m_1} \sqrt{\frac{2 J +1}{2 j_2 +1}} \langle
    J M j_1 \, {-m_1} | j_2 m_2 \rangle
  \end{align}
\end{subequations}

\subsection{Orthogonality relations}
%
\begin{subequations}
  \begin{align}\label{Q:C-G-orthogonality-0}
    \sum_{m_{1}=-j_{1}}^{j_{1}} C(j_{1} \, j_{2} \, j_{3};m_{1} \, m_{2} \, m_{3}) \,
    C(j_{1} \, j_{2} \, j'_{3};m_{1} \, m_{2} \, m_{3}) =
    \delta_{j_{3},j'_{3}} \\
\label{Q:C-G-orthogonality-1}
    \sum_{J=|j_1-j_2|}^{j_1+j_2} \sum_{M=-J}^{J} \langle j_1 m_1 j_2 m_2|J M\rangle\langle
    J M|j_1 m_1' j_2 m_2'\rangle =
    \delta_{m_1,m_1'}\delta_{m_2,m_2'} \\
    \label{Q:C-G-orthogonality-2} \sum_{m_1, m_2} \langle J M|j_1 m_1 j_2 m_2\rangle \,
    \langle j_1 m_1 j_2 m_2|J' M'\rangle = \langle J M | J' M'\rangle =
    \delta_{J,J'}\delta_{M,M'}
  \end{align}
\end{subequations}


\subsection{Particular values}
\label{S:C-G particular-values}
%
When one of the angular momenta are zero:
\begin{align}
\label{Q:spf1a}
C(j_{1} \, 0 \, j_{3}; m_{1} \,0 \, m_{3}) = \delta_{j_{1},j_{3}} \,
\delta_{m_{1},m_{3}}  &
\\
  \label{Q:C-G-part-val-1}
  C(j_{1} \, 1 \, j_{3}; 0 \,0 \, 0) = (j_{3}-j_{1})\, \sqrt{\frac{J_{+}}{2 j_{1} + 1}} \, \delta_{j_{1},j_{3} \pm 1}& & \text{where} && J_{+} = \max{(j_{1},j_{3})} 
\end{align}
Note that $j_{3}-j_{1} = \pm 1$.

\section{Wigner 3$-j$ symbols}

Closely related to the Clebsch-Gordan coefficient are the Wigner 3-$j$
symbols

\begin{equation}\label{Q:spf2}
C(j_{1} \, j_{2} \, j_{3}; m_{1} \, m_{2} \, m_{3}) =
(-1)^{j_{1}-j_{2}+m_{3}} \, \sqrt{2 j_{3} + 1} \left( \begin{matrix}j_{1} &
j_{2} & j_{3} \cr m_{1} & m_{2} & -m_{3}
\end{matrix}
\right)
\end{equation}
%

\subsection{Symmetries}
\label{S:3j-symmetries}

They present a highly symmetric form in the three pairs
$(j_{i},m_{i})$. In particular, they verify
\begin{equation} \label{Q:spf3}
(-1)^{j_{1}+j_{2}+j_{3}} \, \left( \begin{matrix}j_{1} & j_{2} & j_{3} \cr
m_{1} & m_{2} & m_{3}
\end{matrix}
\right) =  \left( \begin{matrix}j_{2} & j_{1} & j_{3}
\cr m_{3} & m_{1} & m_{3}
\end{matrix}
\right) = \left( \begin{matrix}j_{1} & j_{3} &
j_{2} \cr m_{1} & m_{3} & m_{2}
\end{matrix}
\right)\, .
\end{equation}
Moreover,

\begin{equation}\label{Q:spf4}
\left( \begin{matrix}j_{1} & j_{2} & j_{3} \cr 0 & 0 & 0
  \end{matrix}
\right)=0 \qquad
\mbox{if} \qquad j_{1}+j_{2} + j_{3}  \quad \mbox{is an odd number}
\end{equation}

  \noindent
The equation equivalent to \ref{Q:spf1a} is
\[
\left( \begin{matrix}j_{1} & j_{2} & 0 \cr  m_{1} & m_{2} & 0
  \end{matrix}
\right)=(-1)^{j_{1} + m_{2}} \sqrt{2 j_{1} +1} \left( \begin{matrix}j_{1} &
j_{2} & j_{3} \cr 0 & 0 & 0
\end{matrix}
\right) \delta_{j_{1},j_{2}} \,
\delta(m_{1},m_{2})
\]

\subsection{Orthogonality relations}

From equation \ref{Q:spf1b} it is easy to find the orthogonality
relations for 3-$j$ symbols

\begin{equation}\label{Q:spf1b2}
\sum_{m_{1}=-j_{1}}^{j_{1}}  \left( \begin{matrix}j_{1} & j_{2} & j_{3} \cr
m_{1} & m_{2} & -m_{3}
\end{matrix}
\right) \,  \begin{pmatrix}j_{1} & j_{2} &
j'_{3} \cr m_{1} & m_{2} & -m_{3}
\end{pmatrix}
 = ( 2 j_{3} + 1 ) \, \delta_{j_{3},j_{3}'}
\end{equation}

\subsection{Particular values}
\label{S:3j-particular-values}

\begin{equation}
  \label{Q:3j-part-val-1}
  \begin{pmatrix}  j & 1 & j' \cr 0 & 0 & 0 \end{pmatrix} 
  = (-1)^{j-1}\,  \sqrt{\frac{j_{>}}{(2j'+1)(2j+1)}} \, \delta_{j',j\pm 1}
\end{equation}
where $j_{>} = \max{(j,j')}$.
\section{Wigner-Eckart theorem}
\label{S:wign-eckart-theor}

The Wigner-Eckart theorem establishes that the matrix element of a irreducible tensor of k-rank between states $\ket{JM}$ and $\ket{J'M'}$ is proportional to a Clebsh-Gordan coefficient:
\begin{equation*}
  \langle JM|\mathrm{T}^{(k)}_q|J'M'\rangle = \alpha \langle J'M';kq|JM \rangle 
\end{equation*}
with $\alpha$ independent of $M, M', q$.

We use the definition of \autocite[appendix B]{Galindo1990_QMvI}:
\begin{equation}\label{Q:W-E-theorem-2}
  \langle JM|\mathrm{T}^{(k)}_q|J'M'\rangle = \langle J\|\mathrm{T}^{(k)} \|J' \rangle \, \langle J'M';kq|JM \rangle 
\end{equation}
This definition differs from the --most usual-- \autocite[see for instance][ch.~XIII]{Messiah1966_QMvII}:
\begin{equation}\label{Q:W-E-theorem-usual}
  \langle JM|\mathrm{T}^{(k)}_q|J'M'\rangle = \frac{\langle J\|\mathrm{T}^{(k)} \|J' \rangle}{\sqrt{2 J +1}}\, \langle J'M';kq|JM \rangle 
\end{equation}

\section{Vector spherical harmonics}

The Vector spherical harmonic $\boldsymbol{\mathcal{Y}}^{L}_{JM}$ is
the eigenvector of the $\mathbf{L}^{2}$, $\mathbf{J}^{2}$,
\[
\boldsymbol{\mathcal{Y}}^{L}_{JM}.
\]

%%% Local Variables: 
%%% mode: latex
%%% TeX-master: "main"
%%% End: 
