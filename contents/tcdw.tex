\chapter{CDW Approximations}

\section{CDW wavefunction}
We will use the final state wavefunction $C_{3}$ o CDW:
%
\begin{equation}\label{Q:C3}
\Psi^{\pm}_{C3} (\vect{r},\vect{R}) = \frac{e^{i (\vect{k}_j\cdot
\vect{r}_j+\vect{K}_j\cdot \vect{R}_j)}}{(2 \pi)^3} \,
D^{\pm}(\nu_T,\vect{k}_T,\vect{r}_T) \, D^{\pm}(\nu_P,\vect{k}_P,\vect{r}_P)
\,D^{\pm}(\nu_N,\vect{k}_N,\vect{r}_N)
\end{equation}
%
where the distortion factor $D^{\pm}$ are defined in terms of the two body
wave functions as
\[
 \psi^{\pm}_{\vect{k}_j}(\vect{r}_j)  = (2\pi)^{-3/2}e^{i \vect{k}_{j}
 \cdot \vect{r}_{j}} D^{\pm}(\nu_j,\vect{k}_{j},\vect{r}_{j})
\]
%
For a continuum state with Coulomb interactions the distortion factor
is given by
%
\begin{equation}\label{Q:DFactCoul}
 D^{\pm}(\nu_j,\vect{k}_{j},\vect{r}_{j})= N^{\pm}(\nu_{j}) \,{_1F_1}\left(
\mp i \nu_{j};1; {\pm} i (k_{j} r_{j} \mp \vect{k}_{j} \cdot\vect{r}_{j}
) \right) \, ,
\end{equation}
%
$N^{\pm}(\nu_j)= \Gamma(1 {\pm} i\nu_j) e^{-\pi \nu_j/2}$ and the
Sommerfeld's parameter $\nu_j = m_j Z_j/ k_j$.

In the general case, for arbitrary interaction between particles it can
be defined as
\begin{equation}\label{Q:DFact}
D^{\pm}(\nu_j,\vect{k}_j,\vect{r}_j)  = (2\pi)^{3/2}e^{ - i\vect{k}_j \cdot
\vect{r}_j} \psi^{\pm}_{\vect{k}_j}(\vect{r}_j)
\end{equation}
where $\psi^{\pm}_{\vect{k}_j}(\vect{r}_j)$ is the exact two-body wavefunction.


\section{CDW-Born approximation}

We use the final state $\Psi_{C3}$ and an initial undistorted Born wave
function $\Psi_{B1}=\phi_{T} \, \exp{ \left( i \mu_{T} \vect{v}_{T}
\cdot \vect{R}_{T} \right)}$. The wavefunction is the C3
\begin{equation} \label{Q:FCC1}
\Psi^{\pm}_{C3} (\vect{r},\vect{R}) = \frac{e^{i {\bf k}_j\cdot {\bf r}_j+{\bf
K}_j\cdot {\bf R}_j}}{(2 \pi)^3} \, D^{\pm}(\nu_T,{\bf k}_T,{\bf r}_T) \,
D^{\pm}(\nu_P,{\bf k}_P,{\bf r}_P) \,D^{\pm}(\nu_N,{\bf k}_N,{\bf r}_N)
\end{equation}

Here the Coulomb wave function is
\begin{equation}\label{Q:FCC} 
\psi^{\pm}_{{\bf k}_j}({\bf r}_j)  = (2\pi)^{-3/2}e^{i{\bf
k}_j \cdot {\bf r}_j} D^{\pm}(\nu_j,{\bf k}_j,{\bf r}_j)
\end{equation}
%
with
%
\begin{eqnarray}\label{Q:NNorm}
D^{\pm}(\nu_j,{\bf k}_j,{\bf r}_j) &=& N(\nu_j) \,{_1F_1}\left( \mp i
\nu_j;1; {\pm} i(k_j r_j \mp {\bf k}_j\cdot{\bf r}_j ) \right) \, ,\\
\nonumber \\ 
 N^{\pm}(\nu_j)&=& \Gamma(1 {\pm} i\nu_j)\, e^{-\pi \nu_j/2}
\end{eqnarray}
%
and the Sommerfeld's parameter $\nu_j = m_j Z_j/ k_j$.

\subsection{T-matrix in prior-form}
In the prior-form the final distortion, given by the channel potential
is equal to $V_{P} + V_{N}$. We can separate the T-matrix in the form:
$t_{if} = t_{P} + t_{N}$, with $t_{j} = \langle \Psi_{f}^{-}\mid
V_j\mid \Psi_{i} \rangle $

\begin{eqnarray} \label{QA2:6}
t_{P} = \frac{Z_{P}}{(2 \pi)^{3}} && \int d \vect{r}_{P} \,
   \frac{e^{- i \, \vect{Q} \cdot \vect{r}_{P}}}{\vect{r}_{P}}
D^{-\,\ast}(\nu_{P},\vect{k}_{P},\vect{r}_{P}) {\times}
\\
& & \int d \vect{r}_{T} \, e^{i \, (m_{T}/m) \, \vect{Q} \cdot
\vect{r}_T} \, \phi_{i}(\vect{r}_T) \, \phi^{-\, \ast}_{f}(\vect{r}_T) \,
D^{-\,\ast}(\nu_{N}, \vect{k}_{N}, \vect{r}_{T} - \vect{r}_{P} ) \;,
\nonumber
\end{eqnarray}
%
for the projectile-electron interaction and
%
\begin{eqnarray}\label{QA2:7}
t_{N} = \frac{Z_{N}}{(2 \pi)^{3}} & & \int d \vect{r}_{N} \,
   \frac{e^{ i \, \vect{Q} \cdot \vect{r}_{N}}}{\vect{r}_{N}}
D^{-\,\ast}(\nu_{N},\vect{k}_{N},\vect{r}_{N}) {\times}
\\
 & & \int d \vect{r}_T \,
e^{- i \, (m_{T}/M_{T}) \, \vect{Q} \cdot \vect{r}_T} \,
\phi_{i}(\vect{r}_{T})\, \phi^{-\, \ast}_{f}(\vect{r}_T) \,
D^{-\,\ast}(\nu_{P}, \vect{k}_{P}, \vect{r}_{T} - \vect{r}_{N} ) \nonumber
\end{eqnarray}
for the internuclear perturbation.

We define the atomic form factor $\langle \phi_{f}\mid e^{i
\vect{p}\cdot \vect{r}}\mid \phi_{i} \rangle$
%
\begin{equation}\label{QA2:8}
  F_{if}(\vect{p}) = \frac{1}{(2 \pi)^{3/2}} \int d \vect{r}_{T} \;
e^{\, i \vect{p} \cdot \vect{r}_{T}} \, \phi_{i}(\vect{r}_{T}) \,
\phi^{-\, \ast}_{f}(\vect{r}_T)
\end{equation}
%
such that
\[
\phi_{i}(\vect{r}) \, \phi^{-\, \ast}_{f}(\vect{r}) = \frac{1}{(2
\pi)^{3/2}} \int d \vect{p} \, e^{- i \vect{p} \cdot \vect{r}} \,
F_{if}(\vect{p}) \,.
\]
%
Replacing in \ref{QA2:6} and \ref{QA2:7} we get
%
\begin{eqnarray}\label{QA2:9}%
t_{P} &=& -\frac{Z_{P}\, \mathcal{N}_{P} \mathcal{N}_{N}}{(2\pi)^{9/2}}
\; \lim_{\lambda_{1},\lambda_{2} \to 0^{+}} \mathcal{H} \left(
\lambda_{1},- \nu_{P},\vect{k}_{P} ;  \lambda_{2},-
\nu_{N},\vect{k}_{N};\vect{Q} , -(m_{T}/M_{T}) \, \vect{Q} \right)
  \nonumber \\
t_{N} &=& \frac{Z_{N} \, \mathcal{N}_{P} \mathcal{N}_{N}}{(2\pi)^{9/2}}
\; \lim_{\lambda_{1},\lambda_{2} \to 0^{+}} \mathcal{H} \left(
\lambda_{1}, - \nu_{N},\vect{k}_{N} ; \lambda_{2},- \nu_{P},\vect{k}_{P} ;
- \vect{Q} ,  (m_{T}/m) \, \vect{Q} \right) \; ,
\end{eqnarray}
%
with
%
\begin{equation}\label{QA2:10}%
 \mathcal{H} \left( \lambda_{1},a_{1},\vect{k}_{1} ;
  \lambda_{2},a_{2},\vect{k}_{2}; \vect{Q}, \vect{p} \right)  =
 \int d \vect{q} \; F_{if}( \vect{p} - \vect{q} ) \;
 J_{0}( \lambda_{2}, \vect{q} + \vect{Q}, a_{2}, \vect{k}_{2} ) \;
 J_{1}( \lambda_{1}, \vect{q}, a_{1}, \vect{k}_{1} )
\end{equation}
%
where $J_{0}$ y $J_{1}$ are the Nordsieck-like integrals
\parencite*{Nordsie1954PRp785} given by eq.~\ref{Q:Jint}.

Now, because the $J_{o}$ is more peaked in the origin than $J_{1}$,
will be useful to change variables $\vect{q} \to \vect{q}' = \vect{q} +
\vect{Q}$ and get (dropping the prime)
\begin{equation}\label{QA2:12}%
 \mathcal{H} \left( \lambda_{1},a_{1},\vect{k}_{1} ;
  \lambda_{2},a_{2},\vect{k}_{2}; \vect{Q}, \vect{p} \right)  =
 \int d \vect{q} \; F_{if}( \vect{p} + \vect{Q} - \vect{q} ) \;
 J_{0}( \lambda_{2}, \vect{q} , a_{2}, \vect{k}_{2} ) \;
 J_{1}( \lambda_{1}, \vect{q} - \vect{Q}, a_{1}, \vect{k}_{1} )
\end{equation}

We can also write simply
\begin{align*}
  t_{if}&= \frac{1}{(2 \pi)^{9/2}} \int d \vect{p}\, F_{if}(\vect{p})\,\Big[ Z_{N} J_{1}(0^{+}, \vect{k}_{1},-\nu_{N}, \vect{k}_{N}) J_{0}(0^{+}, \vect{k}_{2},-\nu_{P}, \vect{k}_{P}) - Z_{P}\,  J_{0}(0^{+}, \vect{k}_{1},-\nu_{N}, \vect{k}_{N}) J_{1}(0^{+}, \vect{k}_{2},-\nu_{P}, \vect{k}_{P}) \Big]
\intertext{with}
\vect{k}_{1}&= \frac{m_{T}}{m}\vect{Q} - \vect{p} \\
\vect{k}_{2}&= - \left( \frac{m_{T}}{M_{T}}\vect{Q} + \vect{p}  \right)
\end{align*}

\subsection{Series expansion of the T-matrix}

We expand the atomic form factor $F_{if}$ in a Taylor series around the
point $\vect{q}= \vect{Q}$, where $J_{1}$ present its maximum,

\begin{equation}\label{QTaylorFif}
F_{if}(\vect{p} + \vect{Q} - \vect{q}) = F_{if}(\vect{p}) + \left( \vect{Q} -
\vect{q} \right) \cdot \left. \nabla F_{if} \right| _{\vect{p}} +
\Delta^{2} F_{if}
\end{equation}

We obtain a similar expansion for $\mathcal{H} = \mathcal{H}_{o} +
\mathcal{H}_{1} + \Delta^{2} \mathcal{H}$.

\subsection*{Evaluation of $\mathcal{H}_{o}$}
We evaluate the first term as:
%
\begin{equation}\label{Q:mathc-lambd-a_1-1}
 \mathcal{H}_{o}\left( \lambda_{1},a_{1},\vect{k}_{1} ;
\lambda_{2},a_{2},\vect{k}_{2}; \vect{Q}, \vect{p} \right) = F_{if}(\vect{p})
\int J_{o}(\lambda_{2},\vect{q},a_{2},\vect{k}_{2}) \, J_{1}(\lambda_{1},
\vect{q}- \vect{Q} ,a_{1},\vect{k}_{1} ) \, d \vect{q}
\end{equation}

By using the relation \ref{Q-N1-Jb} we obtain
\begin{equation}\label{Q:fbeq-mathcalh_o}
  \fbeq{
\mathcal{H}_{o} \left( \lambda_{1},a_{1},\vect{k}_{1} ; \lambda_{2},
a_{2}, \vect{k}_{2} ; \vect{Q}, \vect{p} \right) = (2 \pi)^{3} \;
F_{if}(\vect{p}) \; N_{1} \left(\lambda_{1}+\lambda_{2}, \vect{Q} ; a_{1},
- \vect{k}_{1} ; a_{2}, \vect{k}_{2} \right) \, ,
  }
\end{equation}
%
where $N_{1}$ is the Nordsieck's Integral defined in \ref{Q:N1}.

\subsection*{The order 1}
%
\begin{eqnarray}\label{Q:order1}
 \mathcal{H}_{1}\left( \lambda_{1},a_{1},\vect{k}_{1} ;
\lambda_{2}, a_{2}, \vect{k}_{2}; \vect{Q}, \vect{p} \right) &=&  \nabla
F_{if} ({\vect{p}}) \cdot  \left[ \int \left( \vect{Q} - \vect{q} \right)
J_{o}(\lambda_{2},\vect{q},a_{2},\vect{k}_{2}) \,  J_{1}(\lambda_{1},
\vect{q}- \vect{Q} ,a_{1},\vect{k}_{1} ) \, d \vect{q} \right]
\end{eqnarray}

  \noindent
Now, while $\nabla F_{if} ({\vect{p}}) = i \, \vect{L}_{if}(\vect{p})$ and
using $\vect{I}_{1}$ as in eq. \ref{Q-N1-Jc} we obtain

\begin{eqnarray*}
&&\mathcal{H}_{1}\left( \lambda_{1},a_{1},\vect{k}_{1} ; \lambda_{2},
a_{2}, \vect{k}_{2}; \vect{Q}, \vect{p} \right)= \\
&&\qquad i \vect{L}_{if} ({\vect{p}})
\cdot \left[(2 \pi)^{3} \vect{Q} N_{1}(\lambda,\vect{Q} ; a_{1},-\vect{k}_{1}
; a_{2},\vect{k}_{2}) - \frac{(2 \pi)^{3}}{-i}
\vect{I}_{1}(\lambda,\vect{Q}; a_{1},-\vect{k}_{1}; a_{2},\vect{k}_{2} )
\vstretch \right]
\end{eqnarray*}

The first order is then given by
\begin{eqnarray}\label{Q:mathc-lambd-a_1}
&&\mathcal{H}_{1}\left( \lambda_{1},a_{1},\vect{k}_{1} ; \lambda_{2},
a_{2}, \vect{k}_{2}; \vect{Q}, \vect{p} \right)= \\
&& \qquad (2 \pi)^{3} \, \vect{L}_{if}
({\vect{p}}) \cdot  \left[ \vect{I}_{1}(\lambda,\vect{Q}; a_{1},-\vect{k}_{1};
a_{2},\vect{k}_{2} ) + i \vect{Q} N_{1}(\lambda,\vect{Q} ;
a_{1},-\vect{k}_{1} ; a_{2},\vect{k}_{2}) \vstretch \right] \nonumber
\end{eqnarray}
%
where $\lambda = \lambda_{1}+\lambda_{2}$.



\section{CDW Final Distortion potential}
The wavefunction \ref{Q:C3}, with $D$ defined as in \ref{Q:DFactCoul},
has been employed as an approximated solution of the three-body Coulomb
continuum state. Although our interest is focused in the Coulomb case,
this wavefunction can be defined for general interactions between the
particles through \ref{Q:C3} and \ref{Q:DFact}.

To determine the distortion potential we write the wavefunction as
\[
\Psi^{\pm}_{C3} (\vect{r},\vect{R}) = \frac{e^{i \vect{k}_j\cdot
\vect{r}_j+\vect{K}_j\cdot \vect{R}_j}}{(2 \pi)^3} D_{T} \,
\pounds(\vect{r},\vect{R})
\]
where, in order to simplify the notation, we write $D^{\pm}_{j} \equiv
D^{\pm}(\nu_j,\vect{k}_{j},\vect{r}_{j})$.

\begin{equation}\label{Q:C3-final-disto}
\left( H - E \right) \Psi(\vect{r}_{j}, \vect{R}_{j}) = \left[ H_{0} +
V_{T}(\vect{r}_{T}) + V_{P}(\vect{r}_{P}) + V_{N}(\vect{r}_{N}) - E \right]
\left[ \frac{e^{i (\vect{k}_j\cdot \vect{r}_j+\vect{K}_j \cdot
\vect{R}_j)}}{(2 \pi)^3} \, D_{T} \, \pounds(\vect{r},\vect{R}) \right]
\end{equation}

\noindent Now, using that
\begin{eqnarray*}
\nabla^{2}_{\vect{r}}(f \, g) &=& f \, \nabla^{2}_{\vect{r}}(g) + g \,
\nabla^{2}_{\vect{r}}(f) + 2 \overrightarrow{\nabla}_{\vect{r}}(g) \cdot
\overrightarrow{\nabla}_{\vect{r}}(f) \, , \\
\overrightarrow{\nabla}_{\vect{r}} e^{i \vect{k}\cdot \vect{r}} &=&
i \,
\vect{k} \, e^{i \vect{k}\cdot \vect{r}} \, , \\
\overrightarrow{\nabla}_{\vect{R}_{T}} D_{T}(\vect{r}_{T}) &=& 0
\end{eqnarray*}
 we obtain
%
\begin{eqnarray*}
\left( H - E \right) \Psi &=& \left( \frac{k_{T}^{2}}{2m_{T}} +
\frac{K_{T}^{2}}{2 \mu_{T}} - E \right) \Psi + \frac{e^{ i
(\vect{k}_{T}\cdot \vect{r}_{T} + \vect{K}_{T} \cdot \vect{R}_{T})}}{(2 \pi)^3}
{\times}
\\
&&{\times} \left\{ D_{T} \, \left[ -\frac{\nabla^{2}_{\vect{r}_{T}}}{2 m_{T}}
-\frac{\nabla^{2}_{\vect{R}_{T}}}{2 \mu_{T}}  + V_{N} + V_{P} \right]
\pounds \right. + \pounds \left[-\frac{\nabla^{2}_{\vect{r}_{T}}}{2m_{T}}
+ V_{T} \right] D_{T}
\\
&&\left.-  \frac{ \overrightarrow{\nabla}_{\vect{r}_{T}}D_{T} \cdot
\overrightarrow{\nabla}_{\vect{r}_{T}} \pounds }{m_{T}} - \frac{i
\vect{k}_{T}}{m_{T}} \cdot \left( \pounds \,
\overrightarrow{\nabla}_{\vect{r}_{T}} D_{T} +  D_{T} \,
\overrightarrow{\nabla}_{\vect{r}_{T}}\pounds \right) - \frac{i D_{T}
\vect{K}_{T}}{\mu_{T}} \cdot \overrightarrow{\nabla}_{\vect{R}_{T}} \pounds
  \right\} \, .
\end{eqnarray*}
%
The first term vanishes due to conservation of the energy
$E=k_{T}^{2}/2m_{T} + K_{T}^{2}/2\mu_{T}$. Moreover, as it is easy to
prove from the definition \ref{Q:DFact}, the distortion factors verify
the differential equation
%
\begin{equation}\label{Q:DEDFact}
\left[-\frac{\nabla^{2}_{\vect{r}_{j}}}{2m_{j}} + V_{j} - \frac{ i \,
\vect{k}_{j}}{m_{j}} \cdot
\overrightarrow{\nabla}_{\vect{r}_{j}}\right]D_{j} = 0 \, .
\end{equation}
%
Thus, we can write
\begin{eqnarray}\label{Q:D-pot1}
\left( H - E \right) \Psi &=& \Phi \left\{ D_{T} \, \left[ -
\frac{\nabla^{2}_{\vect{r}_{T}}}{2 m_{T}}
-\frac{\nabla^{2}_{\vect{R}_{T}}}{2 \mu_{T}}  + V_{N} + V_{P} -
\frac{i \vect{k}_{T} \cdot \overrightarrow{\nabla}_{\vect{r}_{T}}
}{m_{T}} - \frac{i \vect{K}_{T} \cdot
\overrightarrow{\nabla}_{\vect{R}_{T}}}{\mu_{T}} \right] \pounds
 \right. \nonumber
\\
&&- \left. \frac{ \overrightarrow{\nabla}_{\vect{r}_{T}}D_{T} \cdot
\overrightarrow{\nabla}_{\vect{r}_{T}} \pounds }{m_{T}}
  \right\} \, .
\end{eqnarray}
where we have defined $\Phi = (2 \pi)^{-3}\ \rme^{i \,(\vect{k}_j\cdot \vect{r}_j + \vect{K}_j \cdot \vect{R}_{j})} $.
The differential operators can be written in terms of any other pair of
Jacobi coordinates. In particular, it can be shown that they transform
as the  Jacobi momenta \parencite{Fiol2002JPBp149}. We obtain
\begin{equation}\label{Q:Rel-Opp}
\left( \begin{matrix} \vect{\nabla}_{\vect{r}_{T}} \cr
\vect{\nabla}_{\vect{R}_{T}} \cr
\end{matrix}
\right) = \left(\begin{matrix}m_{T}
/ m & m_{T}/M_{T} \cr -1 & 1 \cr
\end{matrix}
\right) \left( \begin{matrix}
\vect{\nabla}_{\vect{r}_{P}} \cr
\vect{\nabla}_{\vect{r}_{N}} \cr
\end{matrix}
\right)  \, .
\end{equation}

In this coordinates the free hamiltonian $H_{0}$ is given by
\[
H_{0}=-\frac{ \nabla^{2}_{\vect{r}_{T}}}{2 m_{T}}
-\frac{\nabla^{2}_{\vect{R}_{T}}}{2 \mu_{T}} \, = \,
    -\frac{\nabla^{2}_{\vect{r}_{P}}}{2 m_{P}}
  -\frac{ \nabla^{2}_{\vect{r}_{N}}}{2 m_{N}}
  +\frac{\nabla_{\vect{r}_{P}} \cdot \nabla_{\vect{r}_{N}}}{M_{P}} \, .
\]

Employing these relations in eq. \ref{Q:D-pot1} and writing $\pounds =
D_{P} D_{T} $ we obtain
%
\begin{eqnarray}\label{Q:D-Pot2}
\left( H - E \right) \Psi \!\!\!&=& \Phi \left\{ D_{T} \, \left[
-\frac{\nabla^{2}_{\vect{r}_{P}}}{2 m_{P}}+ V_{P} -\frac{
\nabla^{2}_{\vect{r}_{N}}}{2 m_{N}} + V_{N} - \frac{i \vect{k}_{T} \cdot
\overrightarrow{\nabla}_{\vect{r}_{T}} }{m_{T}} - \frac{i \vect{K}_{T}
\cdot \overrightarrow{\nabla}_{\vect{R}_{T}}}{\mu_{T}} \right] D_{P}
D_{N}
 \right.
\\
&&- \left. \frac{ \overrightarrow{\nabla}_{\vect{r}_{T}} D_{T} \cdot
\overrightarrow{\nabla}_{\vect{r}_{T}} D_{P} D_{N} }{m_{T}} + D_{T}
\frac{\overrightarrow{\nabla}_{\vect{r}_{P}}D_{P} \cdot
\overrightarrow{\nabla}_{\vect{r}_{N}}D_{N}}{M_{P}} \right\} \nonumber
\\
&=& \Phi \left\{ D_{T} \,  D_{N} \frac{i \vect{k}_{P}\cdot 
\overrightarrow{\nabla}_{\vect{r}_{P}}}{m_{P}}D_{P} +
D_{P}D_{T}\frac{i \vect{k}_{N} \cdot
\overrightarrow{\nabla}_{\vect{r}_{N}}}{m_{N}} D_{N} \right.\nonumber
 \\
&&- D_{T} \,\left[\frac{i \vect{k}_{T} \cdot \left( \frac{m_{T}}{m}
\overrightarrow{\nabla}_{\vect{r}_{P}} + \frac{m_{T}}{M_{T}}
\overrightarrow{\nabla}_{\vect{r}_{N}}\right) }{m_{T}} - \frac{i
\vect{K}_{T} \cdot \left( - \overrightarrow{\nabla}_{\vect{r}_{P}} +
\overrightarrow{\nabla}_{\vect{r}_{N}}\right) }{\mu_{T}} \right]  D_{P}
D_{N}\nonumber
\\
&&- \left. \frac{ \overrightarrow{\nabla}_{\vect{r}_{T}} D_{T} \cdot
\left( \frac{m_{T}}{m} \overrightarrow{\nabla}_{\vect{r}_{P}} +
\frac{m_{T}}{M_{T}} \overrightarrow{\nabla}_{\vect{r}_{N}}\right) D_{P}
D_{N} }{m_{T}} + D_{T} \frac{\overrightarrow{\nabla}_{\vect{r}_{P}}D_{P}
\cdot \overrightarrow{\nabla}_{\vect{r}_{N}}D_{N}}{M_{P}} \right\}
\nonumber
\\
&=& \Phi \, \left\{ D_{T} \,  D_{N} \frac{i \vect{k}_{P}\cdot
\overrightarrow{\nabla}_{\vect{r}_{P}}D_{P}}{m_{P}} +
D_{P}D_{T}\frac{i \vect{k}_{N} \cdot
\overrightarrow{\nabla}_{\vect{r}_{N}} D_{N}}{m_{N}} - D_{T} D_{N} \,
\frac{i \vect{k}_{T} \cdot \overrightarrow{\nabla}_{\vect{r}_{P}}
D_{P}}{m} \right. \nonumber
 \\
&&+ D_{T} D_{P} \frac{i \vect{k}_{T} \cdot
\overrightarrow{\nabla}_{\vect{r}_{N}} D_{N}}{M_{T}} + D_{T} D_{N}
\frac{i \vect{K}_{T} \cdot \overrightarrow{\nabla}_{\vect{r}_{P}}
D_{P}}{\mu_{T}} - D_{T} D_{P} \frac{i \vect{K}_{T} \cdot
\overrightarrow{\nabla}_{\vect{r}_{N}} D_{N}}{\mu_{T}} \nonumber
\\
&&- \left. \overrightarrow{\nabla}_{\vect{r}_{T}} D_{T} \cdot \left(D_{N}
\frac{\overrightarrow{\nabla}_{\vect{r}_{P}}D_{P}}{m} + D_{P} \frac{
\overrightarrow{\nabla}_{\vect{r}_{N}} D_{N}}{M_{T}} \right)+ D_{T}
\frac{\overrightarrow{\nabla}_{\vect{r}_{P}} D_{P} \cdot
\overrightarrow{\nabla}_{\vect{r}_{N}}D_{N}}{M_{P}} \right\} \nonumber
\\
&=& \Phi \, \left\{i \, D_{T} \,  D_{N} \left[ \frac{
\vect{k}_{P}}{m_{P}} - \frac{ \vect{k}_{T} }{m} + \frac{
\vect{K}_{T}}{\mu_{T}} \right] \cdot \overrightarrow{\nabla}_{\vect{r}_{P}}
D_{P} \right.\nonumber
\\
&&+ \, i \, D_{P} \, D_{T}\left[ \frac{\vect{k}_{N} }{m_{N}} +
\frac{\vect{k}_{T} }{M_{T}} - \frac{\vect{K}_{T} }{\mu_{T}} \right] \cdot
\overrightarrow{\nabla}_{\vect{r}_{N}} D_{N} \nonumber
\\
&& \left. - D_{N} \,\frac{\overrightarrow{\nabla}_{\vect{r}_{T}} D_{T}
\cdot \overrightarrow{\nabla}_{\vect{r}_{P}} D_{P}}{m} - D_{P}
\,\frac{\overrightarrow{\nabla}_{\vect{r}_{T}} D_{T} \cdot
\overrightarrow{\nabla}_{\vect{r}_{N}} D_{N}}{M_{T}} + D_{T}
\frac{\overrightarrow{\nabla}_{\vect{r}_{P}}D_{P} \cdot
\overrightarrow{\nabla}_{\vect{r}_{N}}D_{N}}{M_{P}} \right\} .\nonumber
\end{eqnarray}

The expression in the last member of (\ref{Q:D-Pot2}) can be further
simplified because the Jacobi momenta are related by \ref{Q:Rel-Opp},
such that the expressions between square brackets vanish. We finally
obtain
%
\begin{equation}\label{Q:D-Pot}
\left( H - E \right) \Psi(\vect{r}_{j}, \vect{R}_{j}) = \mathbb{W} \,
\Psi(\vect{r}_{j}, \vect{R}_{j}) \qquad \qquad \mbox{with}\qquad
\mathbb{W}= \frac{ \mathbb{K}_{P} \mathbb{K}_{N}}{M_{P}}
-\frac{\mathbb{K}_{P} \mathbb{K}_{T}}{m} - \frac{\mathbb{K}_{T}
\mathbb{K}_{N}}{M_{T}} \;.
\end{equation}
%
Here, we have defined the multiplicative operators
\[
\mathbb{K}_{j} = \frac{\nabla_{\vect{r}_{j}}
D(\nu_{j};\vect{k}_{j},\vect{r}_{j})}{D(\nu_{j};\vect{k}_{j},\vect{r}_{j})}
\]


\section{CDW approximations in post form}
\label{S:appro-post-form}

\subsection{Evaluation of the T-matrix}
\label{S:Evalu-T-mat}

Now we evaluate the transition matrix in different approximation for
the initial state. We use a Born (B1) initial state, a CDW initial
wavefunction, or its asymptotic limit given by the \emph{eikonal} wave
function,
\begin{eqnarray} \label{Q:wf-i}
\Psi^{\pm}_{C3} (\vect{r},\vect{R}) = \frac{e^{i \mu_{T} \vect{v} \cdot
\vect{R}_T}} {(2 \pi)^{3/2}} \, \phi(\vect{r}_T) \,
D^{\pm}(\nu^{\circ}_P,\vect{k}_{P}^{\circ},\vect{r}_{P}^{\circ})
\,D^{\pm}(\nu_{N}^{\circ},\vect{k}_{N}^{\circ},\vect{r}_N)  \\
\Psi^{\pm}_{E} (\vect{r},\vect{R}) = \frac{e^{i \mu_{T} \vect{v} \cdot
\vect{R}_T}} {(2 \pi)^{3/2}} \, \phi(\vect{r}_T) \,
E^{\pm}(\nu^{\circ}_P,\vect{k}_{P}^{\circ},\vect{r}_{P}^{\circ})
\,E^{\pm}(\nu_{N}^{\circ},\vect{k}_{N}^{\circ},\vect{r}_N)
\end{eqnarray}
%
where $E^{\pm}(\nu,\vect{k},\vect{r})$ is given by the asymptotic behavior of
the Coulomb distortion factor
\begin{equation} \label{Q:Eikonal-wf}
E^{\pm}(\nu,\vect{k},\vect{r}) = \left. D^{\pm}(\nu,\vect{k},\vect{r}) \vstretch
\right|_{r \to \infty} = e^{{\pm} i \nu \ln{\left( kr \mp \vect{k}\cdot
\vect{r} \right)}}
\end{equation}


The T-matrix element can be decomposed in three terms $t_{if}=
\sum_{j=T,P,N} t^{j}_{if}$. The terms are given by:

\subsubsection{The $t^{N}_{if}$ term}
%
\begin{eqnarray} \label{Q:tifN1}
&t^{N}_{if}&\!\!= \frac{-1}{(2 \pi)^{9/2}}  \int d \vect{r}_{T} \,
  e^{i \, (m_{T}/m) \, \vect{Q} \cdot \vect{r}_T}
\, \phi_{i}(\vect{r}_T) \, e^{-i \vect{k}_{T} \cdot \vect{r}_T}
\,\nabla_{\vect{r}_{T}}\left[ D^{-*}(\nu_{T},\vect{k}_{T},\vect{r}_T) \right]
{\times}
\\
&& \int d \vect{r}_{P} \,
   e^{- i \, \vect{Q} \cdot \vect{r}_{P}}
\nabla_{\vect{r}_{P}}\left[ D^{-*}(\nu_{P},\vect{k}_{P},\vect{r}_{P}) \right]
\, D^{+}(\nu_{P}^{\circ},\vect{k}_{P}^{\circ},\vect{r}_{P}) \, D^{-\,\ast}(\nu_{N},
\vect{k}_{N}, \vect{r}_{N} ) \, D^{+}(\nu_{N}^{\circ},\vect{k}_{N}^{\circ},\vect{r}_{N})\;
,\nonumber
\end{eqnarray}

By defining the vector (see \ref{S:FGKL-if})
\begin{equation}\label{Q:Kif}
  \vect{K}_{if}(\vect{p}) = \frac{1}{(2 \pi)^{3/2}} \int d \vect{r}_{T} \;
e^{\, i \vect{p} \cdot \vect{r}_{T}} \, \phi_{i}(\vect{r}_{T}) \left\{
\frac{e^{- i \vect{k}_{T} \cdot \vect{r}_{T}}}{(2
\pi)^{3/2}}\,\nabla_{\vect{r}_{T}} \left[
D^{-*}(\nu_{T},\vect{k}_{T},\vect{r}_{T}) \right] \right\}
\end{equation}
%
such that
\begin{equation}\label{Q:Fif-inv}
\phi_{i}(\vect{r}_{T}) \frac{ e^{- i \vect{k}_{T} \cdot \vect{r}_{T}}}
{(2 \pi)^{3/2}} \,\nabla_{\vect{r}_{T}} \left[
D^{-*}(\nu_{T},\vect{k}_{T},\vect{r}_{T}) \right]  = \frac{1}{(2
\pi)^{3/2}} \int d \vect{p} \; e^{- i \vect{p} \cdot \vect{r}_{T}}
\,\vect{K}_{if}(\vect{p})
\end{equation}
%
we can write
\begin{eqnarray*}
t^{N}_{if}&=& \frac{-1}{(2 \pi)^{9/2}} \int d \vect{q}
\vect{K}_{if}(\vect{q}) \int d \vect{r}_{T} \, e^{i \left(
(m_{T}/m)\vect{Q} - \vect{q} \right) \cdot \vect{r}_{T}} {\times}
  \\
&& \int d \vect{r}_{P} \, e^{- i \, \vect{Q} \cdot \vect{r}_{P}}
\nabla_{\vect{r}_{P}}\left[ D^{-*}(\nu_{P},\vect{k}_{P},\vect{r}_{P}) \right]
\, D^{+}(\nu_{P}^{\circ},\vect{k}_{P}^{\circ},\vect{r}_{P}) \, D^{-\,\ast}(\nu_{N},
\vect{k}_{N}, \vect{r}_{N} ) \, D^{+}(\nu_{N}^{\circ}, \vect{k}_{N}^{\circ}, \vect{r}_{N}) \;
,\nonumber
\end{eqnarray*}

Changing from $\vect{r}_{T} \to \vect{r}_{N}= \vect{r}_{T} - \vect{r}_{P}$
\begin{eqnarray} \label{Q:tifcdws1}
t^{N}_{if}&=& \frac{-1}{(2 \pi)^{9/2}} \int d \vect{q}
\vect{K}_{if}(\vect{q}) \int d \vect{r}_{N} \, e^{i \left(
(m_{T}/m)\vect{Q} - \vect{q} \right) \cdot \vect{r}_{N}} \,
 D^{-\,\ast}(\nu_{N},
\vect{k}_{N}, \vect{r}_{N} ) \, D^{+}(\nu_{N}^{\circ},\vect{k}_{N}^{\circ},\vect{r}_{N})\; {\times}
  \\
&& \int d \vect{r}_{P} \, \,e^{ -i \left((m_{T}/M_{T})\vect{Q}+
\vect{q} \right) \cdot \vect{r}_{P}} \,
D^{+}(\nu_{P}^{\circ},\vect{k}_{P}^{\circ},\vect{r}_{P}) \, \nabla_{\vect{r}_{P}}\left[
D^{-*}(\nu_{P},\vect{k}_{P},\vect{r}_{P}) \right] ,\nonumber
\end{eqnarray}

\subsubsection{The $t^{P}_{if}$ term}

The evaluation of the $t^{P}_{if}$ term is similar to the one of the
$t^{N}_{if}$ term. The equivalent of equation \ref{Q:tifN1} is
\begin{eqnarray*}
&t^{P}_{if}&\!\!= \frac{-1}{(2 \pi)^{9/2}\, M_{T}}  \int d
\vect{r}_{T} \,
  e^{i \, (m_{T}/m) \, \vect{Q} \cdot \vect{r}_T}
\, \phi_{i}(\vect{r}_T) \, e^{-i \vect{k}_{T} \cdot \vect{r}_T}
\,\nabla_{\vect{r}_{T}}\left[ D^{-*}(\nu_{T},\vect{k}_{T},\vect{r}_T) \right]
{\times}
\\
&& \int d \vect{r}_{P} \,
   e^{- i \, \vect{Q} \cdot \vect{r}_{P}}
 D^{-*}(\nu_{P},\vect{k}_{P},\vect{r}_{P})
\, D^{+}(\nu_{P}^{\circ},\vect{k}_{P}^{\circ},\vect{r}_{P}) \,\nabla_{\vect{r}_{N}}\left[
D^{-\,\ast}(\nu_{N}, \vect{k}_{N}, \vect{r}_{N} )  \right]\,
D^{+}(\nu_{N}^{\circ},\vect{k}_{N}^{\circ},\vect{r}_{N})\; .\nonumber
\end{eqnarray*}

A completely equivalent procedure to the followed above leads
\begin{eqnarray}\label{Q:tifcdw2}
t^{P}_{if}&=& \frac{-1}{(2 \pi)^{9/2}\, M_{T}} \int d \vect{q}
\vect{K}_{if}(\vect{q}) \\
&& \int d \vect{r}_{N} \, e^{i \left( (m_{T}/m)\vect{Q} - \vect{q}
\right) \cdot \vect{r}_{N}} \, \nabla_{\vect{r}_{N}}\left[
D^{-\,\ast}(\nu_{N}, \vect{k}_{N}, \vect{r}_{N} ) \right]\,
D^{+}(\nu_{N}^{\circ},\vect{k}_{N}^{\circ},\vect{r}_{N})\; {\times}
  \\
&& \int d \vect{r}_{P} \, \,e^{ -i \left((m_{T}/M_{T})\vect{Q}+
\vect{q} \right) \cdot \vect{r}_{P}} \,
D^{-*}(\nu_{P},\vect{k}_{P},\vect{r}_{P} ) \, D^{+}(\nu_{P}^{\circ},\vect{k}_{P}^{\circ},
\vect{r}_{P}) ,\nonumber
\end{eqnarray}

\subsubsection{The $t^{T}_{if}$ term}
The remaining term is given by
%
\begin{eqnarray} \label{Q:tifT1}
&t^{T}_{if}&\!\!= \frac{(2 \pi)^{-9/2}}{M_{P}}  \int d \vect{r}_{T} \,
e^{i \, (m_{T}/m) \, \vect{Q} \cdot \vect{r}_T} \, \phi_{i}(\vect{r}_T)
\, e^{-i \vect{k}_{T} \cdot \vect{r}_T}
D^{-*}(\nu_{T},\vect{k}_{T},\vect{r}_T) {\times}
\\
&& \int d \vect{r}_{P} \, e^{- i \, \vect{Q} \cdot \vect{r}_{P}}
\nabla_{\vect{r}_{P}}\left[ D^{-*}(\nu_{P},\vect{k}_{P},\vect{r}_{P}) \right]
\,
D^{+}(\nu_{P}^{\circ},\vect{k}_{P}^{\circ},\vect{r}_{P}) \nonumber \\
&&\,\,\nabla_{\vect{r}_{N}}\left[ D^{-\,\ast}(\nu_{N}, \vect{k}_{N},
\vect{r}_{N} ) \right] \, D^{+}(\nu_{N}^{\circ},\vect{k}_{N}^{\circ},\vect{r}_{N})\;
,\nonumber
\end{eqnarray}

By introducing the form factor $F_{if}$ (defined by \ref{QA2:8}) we
obtain
\begin{eqnarray}\label{Q:tt_if}
t^{T}_{if}&=& \frac{(2 \pi)^{-9/2}}{M_{P}} \int d \vect{q}
F_{if}(\vect{q}) \int d \vect{r}_{T} \, e^{i \left(
(m_{T}/m)\vect{Q} - \vect{q} \right) \cdot \vect{r}_{T}} {\times}
  \\
&& \int d \vect{r}_{P} \, e^{- i \, \vect{Q} \cdot \vect{r}_{P}}
\nabla_{\vect{r}_{P}}\left[ D^{-*}(\nu_{P},\vect{k}_{P},\vect{r}_{P}) \right]
\,
D^{+}(\nu_{P}^{\circ},\vect{k}_{P}^{\circ},\vect{r}_{P}) \, \\
&& \nabla_{\vect{r}_{N}} \left[ D^{-\,\ast}(\nu_{N}, \vect{k}_{N},
\vect{r}_{N} ) \right] \, D^{+}(\nu_{N}^{\circ}, \vect{k}_{N}^{\circ}, \vect{r}_{N}) \;
,\nonumber
\end{eqnarray}
%
and changing $\vect{r}_{T}\to \vect{r}_{N}$
\begin{eqnarray} \label{Q:tifcdws2}
t^{T}_{if}&=& \frac{(2 \pi)^{-9/2}}{M_{P}} \int d \vect{q}
F_{if}(\vect{q})
\\
&& \int d \vect{r}_{N} \, e^{i \left( (m_{T}/m)\vect{Q} - \vect{q}
\right) \cdot \vect{r}_{N}} \, \nabla_{\vect{r}_{N}} \left[
D^{-\,\ast}(\nu_{N}, \vect{k}_{N}, \vect{r}_{N} ) \right] \,
D^{+}(\nu_{N}^{\circ},\vect{k}_{N}^{\circ},\vect{r}_{N})\; {\times}
  \\
&& \int d \vect{r}_{P} \, \,e^{ -i \left((m_{T}/M_{T})\vect{Q}+
\vect{q} \right) \cdot \vect{r}_{P}} \,
D^{+}(\nu_{P}^{\circ},\vect{k}_{P}^{\circ},\vect{r}_{P}) \, \nabla_{\vect{r}_{P}}\left[
D^{-*}(\nu_{P},\vect{k}_{P},\vect{r}_{P}) \right] ,\nonumber
\end{eqnarray}

\subsection{Results for different approximations}


\subsubsection{CDW-CDW}

From equation \ref{Q:tifcdws1} we obtain for the CDW-CDW transition
matrix
\begin{eqnarray}\label{Q:tn_if}
t^{N}_{if}&=& -\frac{\left( N^{+}(\nu_{P}^{\circ})\,N^{-*}(\nu_{P}) \right)
\left( N^{+}(\nu_{N}^{\circ})\,N^{-*}(\nu_{N}) \right)}{m \; (2 \pi)^{9/2}}
\int d \vect{q} \vect{K}_{if}(\vect{q}) \cdot
  \\
&& N_{0}(0^{+}, (m_{T}/m)\vect{Q} - \vect{q}; -\nu_{N}, \vect{k}_{N};
-\nu_{N}^{\circ}, -\vect{k}_{N}^{\circ} ) \, \vect{I}_{0}( 0^{+} ,
-\vect{q}-(m_{T}/M_{T})\vect{Q};-\nu_{P},\vect{k}_{P}; -\nu_{P}^{\circ},
-\vect{k}_{P}^{\circ}) \nonumber
\\
\nonumber \\
t^{P}_{if}&=& -\frac{\left( N^{+}(\nu_{P}^{\circ})\,N^{-*}(\nu_{P}) \right)
\left( N^{+}(\nu_{N}^{\circ})\,N^{-*}(\nu_{N}) \right)}{M_{T} \; (2
\pi)^{9/2}} \int d \vect{q} \vect{K}_{if}(\vect{q}) \cdot
  \\
&& \vect{I}_{0}(0^{+}, (m_{T}/m)\vect{Q} - \vect{q}; -\nu_{N}, \vect{k}_{N} ;
-\nu_{N}^{\circ}, -\vect{k}_{N}^{\circ}) \, N_{0}( 0^{+} ,
-\vect{q}-(m_{T}/M_{T})\vect{Q};-\nu_{P},\vect{k}_{P}; -\nu_{P}^{\circ},
-\vect{k}_{P}^{\circ}) \nonumber
\\
\nonumber \\
t^{T}_{if}&=& \frac{\left( N^{+}(\nu_{P}^{\circ})\,N^{-*}(\nu_{P}) \right)
\left( N^{+}(\nu_{N}^{\circ})\,N^{-*}(\nu_{N}) \right)}{M_{P} \; (2
\pi)^{9/2}} \int d \vect{q} F_{if}(\vect{q}) \cdot
  \\
&& \vect{I}_{0}(0^{+}, (m_{T}/m)\vect{Q} - \vect{q}; -\nu_{N}, \vect{k}_{N};
-\nu_{N}^{\circ}, -\vect{k}_{N}^{\circ} ) \, \vect{I}_{0}( 0^{+} , -\vect{q} -
(m_{T}/M_{T})\vect{Q} ; - \nu_{P}, \vect{k}_{P}; -\nu_{P}^{\circ}, - \vect{k}_{P}^{\circ} )
\nonumber
\end{eqnarray}

\subsubsection{CDW-EIS}

With an Eikonal initial state (replacing $D^{+} \to E^{+}$) we get:
\begin{eqnarray}\label{Q:tn_if-eikon}
t^{N}_{if}&=& -\frac{\left(N^{-*}(\nu_{P}) \,N^{-*}(\nu_{N}) \right)}{m
\; (2 \pi)^{9/2}} \int d \vect{q} N^{E}_{0}(0^{+}, (m_{T}/m)\vect{Q} -
\vect{q}; -\nu_{N}, \vect{k}_{N}; -\nu_{N}^{\circ}, -\vect{k}_{N}^{\circ} ) \,
  \\
&& \vect{K}_{if}(\vect{q}) \cdot \vect{I}^{E}_{0}( 0^{+} ,
-\vect{q}-(m_{T}/M_{T})\vect{Q} ; -\nu_{P}, \vect{k}_{P} ; -\nu_{P}^{\circ},
-\vect{k}_{P}^{\circ}) \nonumber
  \\
\nonumber \\
t^{P}_{if} &=& -\frac{\left(N^{-*}(\nu_{P}) \,N^{-*}(\nu_{N})
\right)}{M_{T} \; (2 \pi)^{9/2}} \int d \vect{q} \;
\vect{K}_{if}(\vect{q}) \cdot
  \\
&& \vect{I}^{E}_{0}(0^{+}, (m_{T}/m)\vect{Q} - \vect{q}; -\nu_{N},
\vect{k}_{N}; -\nu_{N}^{\circ}, -\vect{k}_{N}^{\circ} ) \,  N^{E}_{0}( 0^{+} ,
-\vect{q}-(m_{T}/M_{T})\vect{Q};-\nu_{P},\vect{k}_{P}; -\nu_{P}^{\circ},
-\vect{k}_{P}^{\circ}) \nonumber
\\
\nonumber \\
t^{T}_{if}&=& \frac{\left(N^{-*}(\nu_{P}) \,N^{-*}(\nu_{N})
\right)}{M_{P} \; (2 \pi)^{9/2}} \int d \vect{q} \; F_{if}(\vect{q})
\cdot
  \\
&& \vect{I}^{E}_{0}(0^{+}, (m_{T}/m)\vect{Q} - \vect{q}; -\nu_{N},
\vect{k}_{N}; -\nu_{N}^{\circ}, -\vect{k}_{N}^{\circ} ) \,  \vect{I}^{E}_{0}( 0^{+} ,
-\vect{q}-(m_{T}/M_{T})\vect{Q};-\nu_{P},\vect{k}_{P}; -\nu_{P}^{\circ},
-\vect{k}_{P}^{\circ}) \nonumber
\end{eqnarray}

\subsubsection{CDW-B1}

In the case of an unperturbed Born initial state, we must set the
distortion factors in the initial channel to 1,
\begin{eqnarray}\label{Q:tn_if-born}
t^{N}_{if}&=& -\frac{\left(N^{-*}(\nu_{P}) \,N^{-*}(\nu_{N}) \right)}{m
\; (2 \pi)^{9/2}} \int d \vect{q} \, J_{0}(0^{+}, (m_{T}/m)\vect{Q} -
\vect{q}; -\nu_{N}, \vect{k}_{N} ) \,
  \\
&& \vect{K}_{if}(\vect{q}) \cdot \vect{G}_{0}( 0^{+} ,
-\vect{q}-(m_{T}/M_{T})\vect{Q};-\nu_{P},\vect{k}_{P}) \nonumber
  \\
\nonumber \\
t^{P}_{if}&=& -\frac{\left(N^{-*}(\nu_{P}) \,N^{-*}(\nu_{N})
\right)}{M_{T} \; (2 \pi)^{9/2}} \int d \vect{q} \, J_{0}( 0^{+} ,
-\vect{q}-(m_{T}/M_{T})\vect{Q};-\nu_{P},\vect{k}_{P})  \,
  \\
&& \vect{K}_{if}(\vect{q}) \cdot \vect{G}_{0}(0^{+}, (m_{T}/m)\vect{Q} -
\vect{q}; -\nu_{N}, \vect{k}_{N} )\nonumber
  \\
\nonumber \\
t^{T}_{if}&=& \frac{\left(N^{-*}(\nu_{P}) \,N^{-*}(\nu_{N})
\right)}{M_{P} \; (2 \pi)^{9/2}} \int d \vect{q} \,   F_{if}(\vect{q})
  \\
&&  \vect{G}_{0}( 0^{+} ,
-\vect{q}-(m_{T}/M_{T})\vect{Q};-\nu_{P},\vect{k}_{P})\cdot \vect{G}_{0}(0^{+},
(m_{T}/m)\vect{Q} - \vect{q}; -\nu_{N}, \vect{k}_{N} )\nonumber
\end{eqnarray}

%%%%%%%%%%%%%%%%%%%%%%%%%%%%%%%%%%%%%%%%%%%%%%%%%%%%%%%%%%%%%
\subsection{Evaluation form}

In order to put the above expressions in useful form to evaluate the
integrals we change the integration variable to $\vect{q}' =
-\vect{q}+\vect{p}_{o}$, with
\[
\fbeq{ \vect{p}_{o}=-(m_{T}/M_{T})\vect{Q}}
\]
  and replace:
\[
  \vect{q}= - \vect{q}' + \vect{p}_{o}\, , \qquad
  -\vect{q}-(m_{T}/M_{T})\vect{Q}=\vect{q}' \qquad \mbox{and} \qquad
  (m_{T}/m)\vect{Q}-\vect{q}=\vect{Q}+\vect{q}' \,.
\]
%
Thus, discarding the prime symbol, we obtain
%
\subsubsection{CDW-CDW}

\begin{eqnarray}\label{Q:tn_if-eval-form}
t^{N}_{if}&=& -\frac{\left( N^{+}(\nu_{P}^{\circ})\,N^{-*}(\nu_{P}) \right)
\left( N^{+}(\nu_{N}^{\circ})\,N^{-*}(\nu_{N}) \right)}{m \; (2 \pi)^{9/2}}
\int d \vect{q} \vect{K}_{if}(\vect{p_{o}}- \vect{q}) \cdot
  \\
&& N_{0}(0^{+}, \vect{Q} + \vect{q}; -\nu_{N}, \vect{k}_{N}; -\nu_{N}^{\circ},
-\vect{k}_{N}^{\circ} ) \, \vect{I}_{0}( 0^{+} , \vect{q};-\nu_{P},\vect{k}_{P};
-\nu_{P}^{\circ}, -\vect{k}_{P}^{\circ}) \nonumber
\\
\nonumber \\
t^{P}_{if}&=& -\frac{\left( N^{+}(\nu_{P}^{\circ})\,N^{-*}(\nu_{P}) \right)
\left( N^{+}(\nu_{N}^{\circ})\,N^{-*}(\nu_{N}) \right)}{M_{T} \; (2
\pi)^{9/2}} \int d \vect{q} \vect{K}_{if}(\vect{p_{o}}- \vect{q}) \cdot
  \\
&& \vect{I}_{0}(0^{+}, \vect{Q} + \vect{q}; -\nu_{N}, \vect{k}_{N} ; -\nu_{N}^{\circ},
-\vect{k}_{N}^{\circ}) \, N_{0}( 0^{+} , \vect{q};-\nu_{P},\vect{k}_{P}; -\nu_{P}^{\circ},
-\vect{k}_{P}^{\circ}) \nonumber
\\
\nonumber \\
t^{T}_{if}&=& \frac{\left( N^{+}(\nu_{P}^{\circ})\,N^{-*}(\nu_{P}) \right)
\left( N^{+}(\nu_{N}^{\circ})\,N^{-*}(\nu_{N}) \right)}{M_{P} \; (2
\pi)^{9/2}} \int d \vect{q} F_{if}(\vect{p_{o}}- \vect{q}) \cdot
  \\
&& \vect{I}_{0}(0^{+}, \vect{Q} + \vect{q}; -\nu_{N}, \vect{k}_{N}; -\nu_{N}^{\circ},
-\vect{k}_{N}^{\circ} ) \, \vect{I}_{0}( 0^{+} , -\vect{q} - (m_{T}/M_{T})\vect{Q} ; -
\nu_{P}, \vect{k}_{P}; -\nu_{P}^{\circ}, - \vect{k}_{P}^{\circ} ) \nonumber
\end{eqnarray}

\subsubsection{CDW-EIS}

With an Eikonal initial state (replacing $D^{+} \to E^{+}$) we get:
\begin{eqnarray}\label{Q:tn_if-cdw-eis}
t^{N}_{if}&=& -\frac{\left(N^{-*}(\nu_{P}) \,N^{-*}(\nu_{N}) \right)}{m
\; (2 \pi)^{9/2}} \int d \vect{q} N^{E}_{0}(0^{+}, \vect{Q} + \vect{q};
-\nu_{N}, \vect{k}_{N}; -\nu_{N}^{\circ}, -\vect{k}_{N}^{\circ} ) \,
  \\
&& \vect{K}_{if}(\vect{p_{o}}- \vect{q}) \cdot \vect{I}^{E}_{0}( 0^{+} , \vect{q}
; -\nu_{P}, \vect{k}_{P} ; -\nu_{P}^{\circ}, -\vect{k}_{P}^{\circ}) \nonumber
  \\
\nonumber \\
t^{P}_{if} &=& -\frac{\left(N^{-*}(\nu_{P}) \,N^{-*}(\nu_{N})
\right)}{M_{T} \; (2 \pi)^{9/2}} \int d \vect{q} \;
\vect{K}_{if}(\vect{p_{o}}- \vect{q}) \cdot
  \\
&& \vect{I}^{E}_{0}(0^{+}, \vect{Q} + \vect{q}; -\nu_{N}, \vect{k}_{N};
-\nu_{N}^{\circ}, -\vect{k}_{N}^{\circ} ) \,  N^{E}_{0}( 0^{+} ,
\vect{q};-\nu_{P},\vect{k}_{P}; -\nu_{P}^{\circ}, -\vect{k}_{P}^{\circ}) \nonumber
\\
\nonumber \\
t^{T}_{if}&=& \frac{\left(N^{-*}(\nu_{P}) \,N^{-*}(\nu_{N})
\right)}{M_{P} \; (2 \pi)^{9/2}} \int d \vect{q} \; F_{if}(\vect{p_{o}}-
\vect{q}) \cdot
  \\
&& \vect{I}^{E}_{0}(0^{+}, \vect{Q} + \vect{q}; -\nu_{N}, \vect{k}_{N};
-\nu_{N}^{\circ}, -\vect{k}_{N}^{\circ} ) \,  \vect{I}^{E}_{0}( 0^{+} ,
\vect{q};-\nu_{P},\vect{k}_{P}; -\nu_{P}^{\circ}, -\vect{k}_{P}^{\circ}) \nonumber
\end{eqnarray}

\subsubsection{CDW-B1}

In the case of an unperturbed Born initial state, we must set the
distortion factors in the initial channel to 1,
\begin{eqnarray}\label{Q:tn_if-cdw-b1}
t^{N}_{if}&=& -\frac{\left(N^{-*}(\nu_{P}) \,N^{-*}(\nu_{N}) \right)}{m
\; (2 \pi)^{9/2}} \int d \vect{q} \, J_{0}(0^{+}, \vect{Q} + \vect{q};
-\nu_{N}, \vect{k}_{N} ) \,
  \\
&& \vect{K}_{if}(\vect{p_{o}}- \vect{q}) \cdot \vect{G}_{0}( 0^{+} ,
\vect{q};-\nu_{P},\vect{k}_{P}) \nonumber
  \\
\nonumber \\
t^{P}_{if}&=& -\frac{\left(N^{-*}(\nu_{P}) \,N^{-*}(\nu_{N})
\right)}{M_{T} \; (2 \pi)^{9/2}} \int d \vect{q} \, J_{0}( 0^{+} ,
\vect{q};-\nu_{P},\vect{k}_{P})  \,
  \\
&& \vect{K}_{if}(\vect{p_{o}}- \vect{q}) \cdot \vect{G}_{0}(0^{+}, \vect{Q} +
\vect{q}; -\nu_{N}, \vect{k}_{N} )\nonumber
  \\
\nonumber \\
t^{T}_{if}&=& \frac{\left(N^{-*}(\nu_{P}) \,N^{-*}(\nu_{N})
\right)}{M_{P} \; (2 \pi)^{9/2}} \int d \vect{q} \, F_{if}(\vect{p_{o}}-
\vect{q})
  \\
&&  \vect{G}_{0}( 0^{+} , \vect{q};-\nu_{P},\vect{k}_{P})\cdot
\vect{G}_{0}(0^{+}, \vect{Q} + \vect{q}; -\nu_{N}, \vect{k}_{N} )\nonumber
\end{eqnarray}

%%%%%%%%%%%%%%%%%%%%%%%%%%%%%%%%%%%%%%%%%%%%%%%%%%%%%%%%%
\subsection{Evaluation without the internuclear interaction}

If we set the internuclear interaction to 0 in the initial and final
states, then $\nu_{N}$ and $\nu_{N}^{\circ}$ vanishes and the distortion
factors are equal to 1. In this case, the $N_{0}$ function is
proportional to the delta function $N_{0}= (2 \pi)^{3}
\delta((m_{T}/m)\vect{Q} - \vect{q})$ and the transition matrix is given by

\begin{equation}\label{Q:tif-noN}
t_{if} = \frac{N^{+}(\nu_{P}^{\circ})\,N^{-*}(\nu_{P})}{(2 \pi)^{3/2}}\,
\vect{K}_{if}((m_{T}/m)\vect{Q})  \cdot \vect{I}_{0}( 0^{+} , -\vect{Q};
-\nu_{P}^{\circ}, -\vect{k}_{P}^{\circ};-\nu_{P},\vect{k}_{P})
\end{equation}

\subsubsection*{Eikonal initial state}

In the eikonal approximation for the initial state, the $D$ factors are
replaced by the $E$ factors. The transition matrix is then
\begin{equation}\label{Q:tif-eis}
t_{if} = \frac{{N^{-}}^{*}(\nu_{P})}{(2 \pi)^{3}}\,
\vect{K}_{if}((m_{T}/m)\vect{Q}) \cdot \vect{I}^{E}_{0}( 0^{+} , -\vect{Q};
-\nu_{P}^{\circ}, -\vect{k}_{P}^{\circ};-\nu_{P},\vect{k}_{P})
\end{equation}

\subsection{Details of calculations}

\subsection*{Case of an hydrogenic $1s$ state}
For an hydrogenic initial atomic state $1s$
\[
\phi_{1s}(\vect{r}) = \frac{(m_{T} Z_{T})^{3/2}}{\sqrt{\pi}}
e^{-(m_{T} Z_{T} r)}
\]
%
the factor $\vect{K}$ in the transition matrix element is given by
\begin{eqnarray*}
\lefteqn{\vect{K}_{if}((m_{T}/m)\vect{Q}) = \frac{(m_{T}
Z_{T})^{3/2}}{\sqrt{\pi}} \frac{N^{-*}(\nu_{T})}{(2 \pi)^{3/2}}} \\
&&  \int d \vect{r}_{T} \; \frac{e^{i ((m_{T}/m)\vect{Q} -
\vect{k}_{T}) \cdot \vect{r}_{T}}}{(2 \pi)^{3/2}} e^{-
(m_{T}Z_{T})r_{T}}\, \nabla_{\vect{r}_{T}} \left[\,{_{1}F_{1}}( -i
\nu_{T}; 1; i(k_{T} r_{T} + \vect{k}_{T} \cdot \vect{r}_{T}) ) \vstretch
\right]
\end{eqnarray*}

\section{Semiclassical inclusion of internuclear interaction}

In the usual mass' approximation the internuclear interaction can be
included by means of a phase factor in a semiclassical approximation.
For Coulomb interactions this factor is given by $e^{i
\nu_{N}\ln(\rho)}$ and the transition matrix can be written as
%
\begin{eqnarray}\label{Q:Npot-semic}
t_{if}(\vect{k},\vect{Q}) &=& \frac{1}{(2 \pi)^{2}} \int d \brho \;
e^{i \, \vect{Q} \cdot \brho } \rho^{2 \, i \, \nu} \int d
\vect{Q}_{\perp}' \; e^{- i \, \vect{Q}' \cdot \brho } \;
t^{\circ}_{if}(\vect{k},\vect{Q}) \nonumber \\
  &=&
\frac{1}{(2 \pi)^{2}} \int d \vect{Q}_{\perp}' \;
t^{\circ}_{if}(\vect{k},\vect{Q}) \; \int_{0}^{\infty}  \rho^{ 1 + 2 \, i \,
\nu_{N}} \; d \rho \int_{0}^{2 \pi} e^{i \, (\vect{Q}- \vect{Q}')
\cdot \brho} \; d \varphi_{\rho}.
\end{eqnarray}
%
where $t^{\circ}_{if}$ is evaluated neglecting the internuclear interaction.
The integral on the impact parameter can be carried out analytically
\begin{eqnarray}\label{Q:Npot-semic1}
I_{1}=\int_{0}^{\infty} d \rho \; \rho^{ 1 + 2 \, i \, \nu}
\int_{0}^{2 \pi} e^{i \, \eta\, \rho \cos{\varphi_{\rho}}} \;
d \varphi_{\rho} &=& 2 \pi \int_{0}^{\infty} d \rho \; \rho^{1 +
2 \, i \, \nu} J_{0}(\rho \,\eta)
  \\ &=&
\frac{{4^{1 + i\,\nu_{N}}}\,\pi \, \eta^{-2 - 2\, i\,\nu_{N}}\,
\Gamma(1 + i\,\nu_{N})} {\Gamma\left(- i\,\nu_{N} \right)}
\nonumber
\end{eqnarray}
where $J_{0}$ and $\Gamma$ are the order zero-Bessel and gamma
functions, respectively. We have defined the two-dimensional vector
$\vect{\eta}=\vect{Q}_{\perp}-\vect{Q}_{\perp}'$. The first integration is
valid provided than $\textrm{Im}\left(\eta \, \rho \right)=0$. The
second equality requires that $\textrm{Im}\left(\sqrt{\eta^{2}}
\right)=0$, $\eta^{2} > 0$ and $\textrm{Re}\left( i \nu_{N} \right)
> -1$.

The transition matrix can finally be written as
\begin{eqnarray}\label{Q:tifNsemic}
t_{if}(\vect{k},\vect{Q}) &=& \frac{2^{1 + 2\,i\,\nu_{N}}}{2 \pi}
\frac{\Gamma(1 + i\,\nu_{N})} {\Gamma\left(- i\,\nu_{N} \right)}
\int d \vect{Q}'_{\perp} \; \frac{\; t^{\circ}_{if}(\vect{k},\vect{Q}')} {\mid
\vect{Q}_{\perp}-\vect{Q}'_{\perp}\mid ^{2 + 2\, i\,\nu_{N}}}\,
\nonumber
\\
&=& \frac{2^{1 + 2\,i\,\nu_{N}}}{2 \pi} \frac{\Gamma(1 +
i\,\nu_{N})} {\Gamma\left(- i\,\nu_{N} \right)} \int d
\vect{\eta} \; \frac{\; t^{\circ}_{if}(\vect{k},(\vect{\eta}+\vect{Q},Q_{\parallel}))}
{\eta^{2 + 2\, i\,\nu_{N}}}\,
\end{eqnarray}

This expression presents the difficulty that in the case of null
internuclear interaction does not converge to the identity relation
$t_{if}=t^{\circ}_{if}$.

\subsection{Regularization of the integrals}

In order to define the Fourier transforms we use a regularization based
on an exponential decreasing of the integrand. Equation
\ref{Q:Npot-semic1} can be evaluated as the limit $\alpha \to 0$ of the
integral
\begin{eqnarray}\label{Q:Npot-sreg}
I_{1} &=&\int_{0}^{\infty} d \rho \; \rho^{ 1 + 2 \, i \, \nu} \,
e^{- \alpha \rho} \int_{0}^{2 \pi} e^{i \, \eta\, \rho
\cos{\varphi_{\rho}}} \; d \varphi_{\rho} \nonumber
  \\
&=& 2 \pi \int_{0}^{\infty} d \rho \; \rho^{1 + 2 \, i \, \nu}\,
e^{- \alpha \rho} \, J_{0}(\rho \,\eta) \nonumber
  \\
&=& 2 \pi \, \frac{\Gamma \left( 2 + 2 i \nu_{N} \right)}{ \alpha^{2
+ 2 i \nu_{N}}} \; _{2}F_{1}\left[ 1 + i \nu_{N}, \frac{3}{2} +
i \nu_{N} ; 1; -  \frac{\eta^{2}}{\alpha^{2}} \right] \nonumber
  \\
&=& 2 \pi \, \frac{\Gamma \left( 2 + 2 i \nu_{N} \right)}{ \alpha^{2
+ 2 i \nu_{N}}} \; \left[ 1 + \left( \frac{\eta}{\alpha} \right)^{2}
\right]^{-(1+i \nu)} P^{0}_{1 + 2 i \nu} \left[
\frac{\alpha}{\sqrt{\alpha^{2}+ \eta^{2}}} \right] \nonumber
  \\
&=&  \frac{2 \pi \,\Gamma \left( 2 + 2 i \nu_{N} \right)}{
\left({\alpha^{2} + \eta^{2}} \right)^{1+i \nu}} \;  P^{0}_{1 + 2
i \nu} \left[ \frac{\alpha}{\sqrt{\alpha^{2}+ \eta^{2}}} \right]
\end{eqnarray}
%
where $_{2}F_{1}$ and $P^{\mu}_{\lambda}(x)$ are the hypergeometric and
Legendre functions, respectively.

The transition matrix is evaluated as
\begin{eqnarray}\label{Q:tifNsreg}
t_{if} &=&   \frac{\Gamma \left( 2 + 2 i \nu_{N} \right)}{2 \pi} \;
\int d \vect{\eta} \; P^{0}_{1 + 2 i \nu} \left(
\frac{\alpha}{\sqrt{\alpha^{2}+ \eta^{2}}}\right) \;
\frac{t^{\circ}_{if}(\vect{k},(\vect{\eta}+\vect{Q}_{\perp},Q_{\parallel}))} {
\left( {\alpha^{2} + \eta^{2}} \right)^{1+i \nu}} \nonumber \\ \\
&=&  \frac{\Gamma \left( 2 + 2 i \nu_{N} \right)}{2 \pi} \;
\int_{0}^{\infty} P^{0}_{1 + 2 i \nu} \left(
\frac{\alpha}{\sqrt{\alpha^{2}+ \eta^{2}}} \right) \; \frac{\eta \,
d \eta}{\left( {\alpha^{2} + \eta^{2}} \right)^{1+i \nu}}
\;\int_{0}^{2 \pi} d \varphi_{\eta} \;
t^{\circ}_{if}(\vect{k},(\vect{\eta}+\vect{Q}_{\perp},Q_{\mid }))\nonumber
\\
&=&  \frac{\Gamma \left( 2 + 2 i \nu_{N} \right)}{4 \pi} \;
\int_{0}^{\infty} P^{0}_{1 + 2 i \nu} \left(
\frac{\alpha}{\sqrt{\alpha^{2}+ \eta^{2}}} \right) \; \frac{d
\eta^{2}}{ \left( {\alpha^{2} + \eta^{2}} \right)^{1+i \nu}}
\;\int_{0}^{2 \pi} d \varphi_{\eta} \;
t^{\circ}_{if}(\vect{k},(\vect{\eta}+\vect{Q}_{\perp},Q_{\mid }))\nonumber
\end{eqnarray}
%
where the limit $\alpha \downarrow 0$ must be understood. If we suppose
that the region $\eta \gg \alpha $ most contributes to the above
integral, we can use the expansion of the Legendre function around
$x=0$
%
\[
P_{1 + 2 i \nu} (x) \bajo{\approx}{x \to 0}
\frac{\sqrt{\pi}}{\Gamma\left(\frac{3}{2} + i \nu \right) \,
\Gamma\left(-i \nu\right)} + \frac{(1 + 2 i \nu) (1 + i \nu)
\sqrt{\pi}}{\Gamma\left(\frac{1}{2} - i \nu \right) \,
\Gamma\left(2+ i \nu \right)}\, x + \mathcal{O}(x^{2}) \, .
\]


Equivalently to \ref{Q:tifNsreg} we can write the transition matrix in
terms of the hypergeometric function
%
\begin{eqnarray}\label{Q:t_if-Nint-hyperg}
t_{if} & =&\frac{\Gamma \left( 2 + 2 i \nu_{N} \right)}{2 \pi \,
\alpha^{2 + 2 i \nu_{N}}} \int_{0}^{\infty} \,{_{2}F_{1}}\left[ 1 +
i \nu_{N}, \frac{3}{2} + i \nu_{N} ; 1; -
\frac{\eta^{2}}{\alpha^{2}} \right] \eta \, d \eta \nonumber
\\
&&\int_{0}^{2 \pi} d \varphi_{\eta} \;
t^{\circ}_{if}(\vect{k},(\vect{\eta}+\vect{Q}_{\perp},Q_{\mid  \mid }))
\end{eqnarray}

\subsection*{Limit without internuclear interaction $\nu_{N}=0$}

In the case that the internuclear interaction is negligible
($\nu_{N}=0$) the Legendre function is $P_{0}(x)=x$ and the expression
\ref{Q:tifNsreg} can be rewritten as

\begin{eqnarray*}
t_{if} &=& \frac{1}{2 \pi} \int_{0}^{\infty} \frac{\alpha \eta \, d
\eta}{\sqrt[3]{\alpha^{2}+ \eta^{2}}}  \;\int_{0}^{2 \pi} d
\varphi_{\eta} \;
t^{\circ}_{if}(\vect{k},(\vect{\eta}+\vect{Q}_{\perp},Q_{\mid  \mid })) \\
(u = \alpha^{2}+ \eta^{2}) \quad &=&  \frac{\alpha}{4 \pi} \int_{0}^{2
\pi} d \varphi_{\eta} \; \int_{\alpha^{2}}^{\infty} \, \frac{d u
}{\sqrt[3]{u}} \;\, t^{\circ}_{if}(\vect{k},(\vect{\eta}(u,\alpha) +
\vect{Q}_{\perp},Q_{\mid  \mid }))
\\
&\approx&  \frac{\alpha}{4 \pi} \int_{0}^{2 \pi} d \varphi_{\eta} \;
t^{\circ}_{if}(\vect{k},\vect{Q}) \int_{\alpha^{2}}^{\infty} \, \frac{d u
} {\sqrt[3]{u}} \\
&=&  \frac{\alpha}{4 \pi} \; t^{\circ}_{if}(\vect{k}, \vect{Q}) \, \int_{0}^{2
\pi}
d \varphi_{\eta} \left[ - 2 u^{-1/2}\right]_{\alpha^{2}}^{\infty} \\
&=&  \frac{\alpha}{4 \pi} \; t^{\circ}_{if}(\vect{k}, \vect{Q}) \; 2 \pi \,
\frac{2}{\alpha} \\
&=& t^{\circ}_{if}
\end{eqnarray*}
%
where the natural expression is recovered.

\[
P_{1 + 2 i \nu} (x) \bajo{\approx}{x \to 1} 1 + (1+ 2 i \nu)(1 +
i \nu) (x-1) + \mathcal{O}(x^{2})
\]

%
\section{Capture to bound states of the projectile in the CDW
approximation}

In the case of final states where the electron is bound to the
projectile, the distortion $D_{P}$ \ref{Q:DFact} in the $C_{3}$
wavefunction \ref{Q:C3} must be defined in terms of a bound state
similar to the initial state \ref{Q:wf-i}, $D_{P}=(2 \pi)^{3/2}\
\exp{\left(- i \vect{k}_{P} \cdot \vect{r}_{P} \right)} \, \varphi_{f}
(\vect{r}_{P})$.

The evaluation of the transition matrix is carried out in similar form
that in section \ref{S:appro-post-form} for ionization. For instance,
the only term that contributes in the approximation of masses for heavy
ions
\begin{eqnarray}\label{Q:tif-N-cpt}
t^{N}_{if} &=& \frac{-1}{(2 \pi)^{3}}  \int d \vect{r}_{T} \,d
\vect{r}_{P} \, e^{i \, ( \mu_{T}\vect{v} \cdot \vect{R}_{T} -
\vect{K}_{P}  \cdot \vect{R}_{P} ) }\, \phi_{i}(\vect{r}_T) \,\left[
\nabla_{\vect{r}_{T}} D^{-*}_{T}(\vect{k}^{\circ}_{T},\vect{r}_T) \right] {\times}
\\
&&  \left[ \vstretch \nabla_{\vect{r}_{P}} \phi_{f}(\vect{r}_{P}) \right]
\, D^{\circ +}_{P} (\vect{k}_{P}^{\circ},\vect{r}_{P}) \, \left[ D^{-\,\ast
\circ}_{N}(\vect{k}_{N}, \vect{r}_{N} ) \, D^{+ \circ }_{N}
(\vect{k}_{N}^{\circ},\vect{r}_{N}) \right] \; , \nonumber
\end{eqnarray}
where
\[
\vect{k}_{P}^{\circ} = - m_{P} \vect{v} \; , \qquad \vect{k}^{\circ}_{T} = m_{T} \vect{v}
\]


Now, expressing coordinates and momenta in terms of $\vect{r}_{P}$,
$\vect{r}_{T}$ and $\vect{K}_{R}=\vect{Q}$, the argument of the exponential
is
\[
\vect{K}_{R} \cdot \vect{r}_{T} - \left( m_{P} \vect{v} + \frac{m_{P}}{m}
\vect{K}_{R} \right)\vect{r}_{P}
\]


In the simplest case, the internuclear interaction can be expressed as
an exponential factor which is switched-off in total cross sections
calculations
\begin{eqnarray}\label{Q:tif-cpt-noNN-f}
t_{if}^{N} &=& \frac{-1}{(2 \pi)^{3/2}}  \int d \vect{r}_{T} \,
e^{i \, \vect{K}_{R} \cdot \vect{r}_T} \, \phi_{i}(\vect{r}_T) \,\left[
\nabla_{\vect{r}_{T}} D^{-*}_{T}(\vect{k}^{\circ}_{T},\vect{r}_T) \right]  \cdot
\nonumber \\
&& \int d \vect{r}_{P} \, e^{- i \, (m_{P} \vect{v} +
\frac{m_{P}}{m} \vect{K}_{R} + \vect{k}_{P}^{\circ}) \cdot \vect{r}_{P}}
\left[\nabla_{\vect{r}_{P}} \phi_{f}(\vect{r}_{P}) \right] \,
\psi^{+}_{\vect{k}_{P}^{\circ}} (\vect{r}_{P}) \nonumber
\\
&=& (2 \pi)^{3} \vect{K}^{(T)}_{if}\left[ \vect{K}_{R} + \vect{k}_{T}^{\circ}
\right] \cdot \vect{G}^{(P)}_{if}\left[-\vect{K}_{R},-\vect{k}^{\circ}_{P}\right]
\end{eqnarray}
where we have used that
\[
\frac{m_{P}}{m} \vect{K}_{R} + m_{P}\vect{v} + \vect{k}_{P}^{\circ} =
\frac{m_{P}}{m} \vect{K_{R}} \approx \vect{K}_{R}
\]

Note that this result must be equal to the obtained by applying the
perturbation on the initial state,
%
\begin{eqnarray}\label{Q:tif-cpt-noNN-i}
t_{if}^{N} &=&  \int d \vect{r}_{T} \, e^{i \,(
\vect{K}_{R}+\vect{k}_{T}^{\circ}) \cdot \vect{r}_T} \,\left[ \nabla_{\vect{r}_{T}}
\phi_{i}(\vect{r}_T) \right] \, \psi^{-*}_{\vect{k}^{\circ}_{T}}(\vect{r}_T)   \cdot
\nonumber \\
&&\frac{-1}{(2 \pi)^{3/2}} \int d \vect{r}_{P} \, e^{- i \,
\left( \frac{m_{P}}{m} \vect{K}_{R} + \vect{k}_{P}^{\circ} \right) \cdot
\vect{r}_{P}} \phi_{f}(\vect{r}_{P}) \, \left[\nabla_{\vect{r}_{P}}
D^{+}_{P}(\vect{r}_{P},\vect{k}_{P}^{\circ}) \right]  \nonumber
\\
&=& (2 \pi)^{3} \vect{G}^{(T)}_{if}\left[ \vect{K}_{R} + \vect{k}_{T}^{\circ}
\right] \cdot \vect{K}^{(P)}_{if}\left[-\vect{K}_{R},-\vect{k}^{\circ}_{P}\right]
\end{eqnarray}

\subsection{Eikonal initial or final states}

The eikonal state is defined as the solution of the exact two-body
problem asymptotically. Within this approximation the distortion factor
$D^{\pm}$ must be replaced by its asymptotic limit as expressed in
\ref{Q:Eikonal-wf}
\[
E^{\pm}_{\alpha}(\vect{k},\vect{r}) = \lim_{r \to \infty}
D^{\pm}_{\alpha}(\vect{k},\vect{r})
\]

In the case of Coulomb interactions, lets write the limit
\begin{eqnarray*}
\lim_{\lambda \to \infty} D^{\pm}(\nu_{k}, \vect{k},\lambda \vect{r}) &=&
e^{{\pm} i \, \nu_{k} \ln{\lambda \left( k r \mp \vect{k}\cdot \vect{r}
\right)}}
\\
&=& \lim_{\lambda \to \infty} e^{\pm i \, \nu_{k} \ln{\lambda}}
\, e^{ \pm i \, \nu_{k} \ln{\left( k r \mp \vect{k} \cdot \vect{r}
\right)}}
\end{eqnarray*}
%
Then, we can evaluate the eikonal state as
%
\begin{equation}\label{Q:Eikonal-wf-Coul}
E^{\pm} (\vect{k}_{\alpha}, \vect{r}) = \lim_{\lambda \to \infty} e^{\mp
i \nu \ln{ \lambda}} \, D^{\pm}(\nu , \lambda \vect{k}_{\alpha}, \vect{r})
\end{equation}
where we have used that $D^{\pm} (\vect{k},\lambda \vect{r})= D^{\pm} (\lambda
\vect{k}, \vect{r})$.
%
The factors $\vect{G}_{if}$ in \ref{Q:tif-cpt-noNN-f} and
\ref{Q:tif-cpt-noNN-i} must be computed with the replacement $D \to E$.

%%% Local Variables: 
%%% mode: latex
%%% TeX-master: "main"
%%% End: 
